\documentclass{standalone}

\usepackage{tikz}
\usepackage{verbatim}

\usetikzlibrary{arrows}
\usetikzlibrary{calc}

\begin{document}

\begin{tikzpicture}

  % Colors
  % Styles
  \tikzstyle{defaultRec} = [font=\small, text centered, inner sep = 1pt,
    draw=black, shape=rectangle, fill=white, minimum height = .5cm];
  \tikzstyle{smallJob} = [defaultRec, minimum width = .5cm, anchor=west];
  \tikzstyle{largeJob} = [defaultRec, minimum width = 1.3cm, anchor=west];
  \tikzstyle{cacheHeavyJob} = [defaultRec, minimum width = , anchor=west];

  \tikzstyle{defaultCirc} = [font=\small, text centered, inner sep = 1pt,
    draw=none, shape=circle];
  \tikzstyle{core} = [defaultCirc, minimum width = 1cm];
  
  \tikzstyle{timeArrow} = [->, thin, black];

  % Variables
  \newcommand\Ymax{3};
  \newcommand\Xmax{7};
  \newcommand\fstLvl{1};
  \newcommand\sndLvl{2};


% Drawing

  % Run Queue balancing
  \begin{scope}
    \draw[timeArrow] (0,0) -- (7.5,0);
    \foreach \x in {0,...,\Xmax} 
    \draw[black, thin] (\x cm,.1cm) -- (\x cm, -.1cm);
    \foreach \x in {0,...,\Xmax} 
    \node[draw=none] at (\x cm, -.4cm) {\x};
    \foreach \x in {1,...,\Xmax} 
    \draw[dashed, black, very thin, black!20] (\x, 0) -- (\x,\Ymax cm);

    \draw[thin, black] (0,0) -- (0,\Ymax);

    \node(c1) [defaultCirc, minimum width = .5cm] at (-.5, \fstLvl) {C0};
    \node(c2) [defaultCirc, minimum width = .5cm] at (-.5, \sndLvl) {C1};

    %core 1
    \node[largeJob] at (1,\sndLvl) {B};
    \node[largeJob] at (2.4,\sndLvl) {B};
    \node[smallJob] at (3.8,\sndLvl) {A};

    %core 0
    \node[smallJob] at (1,\fstLvl) {A};
    \node[smallJob] at (1.6,\fstLvl) {A};
    \node[smallJob] at (2.2,\fstLvl) {A};
  \end{scope}

  % balanced by time usage
  \begin{scope} [xshift = 10cm]
    \draw[timeArrow] (0,0) -- (7.5,0);
    \foreach \x in {0,...,\Xmax} 
    \draw[black, thin] (\x cm,.1cm) -- (\x cm, -.1cm);
    \foreach \x in {0,...,\Xmax} 
    \node[draw=none] at (\x cm, -.4cm) {\x};
    \foreach \x in {1,...,\Xmax} 
    \draw[dashed, black, very thin, black!20] (\x, 0) -- (\x,\Ymax cm);

    \draw[thin, black] (0,0) -- (0,\Ymax);

    \node(c1) [defaultCirc, minimum width = .5cm] at (-.5, \fstLvl) {C0};
    \node(c2) [defaultCirc, minimum width = .5cm] at (-.5, \sndLvl) {C1};
    
    %core 1
    \node[largeJob] at (1,\sndLvl) {B};
    \node[smallJob] at (2.4,\sndLvl) {A};
    \node[smallJob] at (3,\sndLvl) {A};

    % core 0
    \node[smallJob] at (1,\fstLvl) {A};
    \node[smallJob] at (1.6,\fstLvl) {A};
    \node[largeJob] at (2.2,\fstLvl) {B};
  \end{scope}

  % Fiasco default
  \begin{scope} [yshift = -5cm]
    \draw[timeArrow] (0,0) -- (7.5,0);
    \foreach \x in {0,...,\Xmax} 
    \draw[black, thin] (\x cm,.1cm) -- (\x cm, -.1cm);
    \foreach \x in {0,...,\Xmax} 
    \node[draw=none] at (\x cm, -.4cm) {\x};
    \foreach \x in {1,...,\Xmax} 
    \draw[dashed, black, very thin, black!20] (\x, 0) -- (\x,\Ymax cm);

    \draw[thin, black] (0,0) -- (0,\Ymax);

    \node(c1) [defaultCirc, minimum width = .5cm] at (-.5, \fstLvl) {C0};
    \node(c2) [defaultCirc, minimum width = .5cm] at (-.5, \sndLvl) {C1};
    
    %core 1
    \node[largeJob] at (1,\fstLvl) {B};
    \node[smallJob] at (2.4,\fstLvl) {A};
    \node[largeJob] at (3,\fstLvl) {B};
    \node[smallJob] at (4.4,\fstLvl) {A};
    \node[smallJob] at (5,\fstLvl) {A};
    \node[smallJob] at (5.6,\fstLvl) {A};

    % core 0
  \end{scope}
  
  % balanced by cache usage
%  \begin{scope} [xshift = 10cm, yshift = -5cm]
%    \draw[timeArrow] (0,0) -- (7.5,0);
%    \foreach \x in {1,...,7} 
%    \draw[black, thin] (\x cm,.1cm) -- (\x cm, -.1cm);
%    \foreach \x in {1,...,7} 
%    \node[draw=none] at (\x cm, -.4cm) {\x};
%    \foreach \x in {1,...,7} 
%    \draw[dashed, black, very thin, black!20] (\x, 0) -- (\x,3cm);
%
%    \draw[thin, black] (0,0) -- (0,3);
%
%    \node(c1) [defaultCirc, minimum width = .5cm] at (-.5, 1) {C0};
%    \node(c2) [defaultCirc, minimum width = .5cm] at (-.5, 2) {C1};
%    
%    %core 1
%    \node[smallJob] at (2.4,2) {A};
%    \node[smallJob] at (3,2) {A};
%    \node[largeJob] at (1,2) {B};
%
%    % core 0
%    \node[smallJob] at (1,1) {A};
%    \node[smallJob] at (1.6,1) {A};
%    \node[largeJob] at (1,1) {B};
%  \end{scope}
\end{tikzpicture}

\end{document}
