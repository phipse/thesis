\section{Fiasco.OC \& L4Re}
\label{state:env}
\todo{Fiasco.OC \& L4Re}

\textbf{Fiasco.OC}
\begin{itemize}
  \item Kernel scheduler does no balancing, assigns thread to the first
    core specified in the affinity descriptor
  \item affinity descriptor: core(s) a thread should run on
  \item Syscall via run\_thread() to pass affinity descr to kernel scheduler
  \item interface to query execution time for each thread
  \item capability system -- to derive communication relationships from
  \item	Kernel feature wishes derived from related work: Performance counters
    and per thread accounting
\end{itemize}

\textbf{L4Re}
\begin{itemize}
  \item provides scheduler proxy interface, including affinity descriptor,
    scheduling parameters
  \tiem syscall interface
\end{itemize}


% -----------------------------------------------------------------------------

\section{\gls{intel} Haswell Architecture}
\label{state:haswell}

As the \ref{state:related}
\includegraphics[width=0.8\textwidth]{images/haswell_architecture_by_intel_large}

\includegraphics[width=0.8\textwidth]{images/haswell_core_layout}


\begin{itemize}
  \item diagram of architecture: four cores with L1I/D \& L2 cache; two smt/ht
    cores per physical core; L3 cache shared and sliced, ring buffer for
    access; mem controler in uncore package;
  \item 4 prefetcher per cache; 2 for L1D, 2 for L2 cache
  \item L1D \& L2 cache data is present in L3 cache to be able to redirect
    requests from other cores to the correct cache. --> Issue with security; no
    cache side attack surface reduction
  \item IF security: dual socket system with one core dedicated to security
    tasks for an interval
  \item L3 cache slices corespond to number of cores
  \item logic portion and data array portion; access, coherency, memory
    ordering, LLC misses, writeback to memory; cache lines
  \item hash function uniformly distributes addresses
  \item access times to L3 cache varies depending on travel distance on the
    bi-directional ring buffer
  \item system agent receives memory requests not serviced by cache and
    redirects to IMC
\end{itemize}

\paragraph{Issues with security core idea:}
The idea to reduce attack surface for e.g. openssh cache attacks was to provide
a dedicated core for security critical applications.
This idea spawned from the assumption that L1D \& L2 cache was solely present
on a core, but the L3 cache content is a superset of all cores L1D \& L2 cache.
Hence, cache side channel attacks are still possible, although more
complicated, as the attacker not present on the core, must determine, which
cache lines correspond to the L1D \& L2 cache of the ``security core''.
Attacks presented in \cite{yarom_recovering_2014} and
\cite{bernstein_cache-timing_2005} rely on the fact, that only one application
is using cache lines on the core.
On a multicore, only observing the \gls{llc}, other running applications are
expected to disturb the observations, effectivly reducing the side channel
throughput.
\todo{This should be part of the architecture chapter, as the security core
idea wasn't presented yet.}

% -----------------------------------------------------------------------------
