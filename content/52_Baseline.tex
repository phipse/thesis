
\paragraph{Baseline}

A baseline is necessary to assess the improvements through a behaviour aware
load balancing.
Previous work on Linux, uses the default Linux scheduler.
However, as neither Fiasco.OC's nor L4Re's scheduler does thread distribution,
I have to write a simple distribution mechanism to provide a baseline.

The aforementioned algorithmcore simple logic distribution is implemented in a
simple load balancer and provides this baseline.
It has no knowledge about thread behaviour and relies on system side
knowledge about core utilization and thread count.
\\

Each of the selected SPEC programs runs solo with the simple balancer as its
scheduler 101 times.
Figures 1-4 show histograms of the runtimes together with the average runtime.

\begin{figure}[h!]
  \begin{subfigure}{.49\textwidth}
    \includegraphics[width=\textwidth]{images/finalPlots/gcc_simple_balancer.pdf}
    \caption{SPEC GCC}
    \label{baseline:gcc}
  \end{subfigure}
  \begin{subfigure}{.49\textwidth}
    \includegraphics[width=\textwidth]{images/finalPlots/gamess_simple_balancer.pdf}
    \caption{SPEC GAMESS}
    \label{baseline:gamess}
  \end{subfigure}
  \begin{subfigure}{.49\textwidth}
    \includegraphics[width=\textwidth]{images/finalPlots/lbm_simple_balancer.pdf}
    \caption{SPEC LBM}
    \label{baseline:gcc}
  \end{subfigure}
  \begin{subfigure}{.49\textwidth}
    \includegraphics[width=\textwidth]{images/finalPlots/mcf_simple_balancer.pdf}
    \caption{SPEC MCF}
    \label{baseline:gamess}
  \end{subfigure}
  \caption{Histograms of runtimes with simple balancer}
\end{figure}
