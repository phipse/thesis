% vim:set ft=tex:
\pagebreak
\section{Task Configuration}
\label{impl:config}

In sections \ref{design:isolation} and \ref{design:comm}, I discussed
task properties relevant to a load balancer.
The result of these discussions was to provide the load balancer with a static
configuration of a task's properties.
In th following section I show how the configuration is used and implemented.

\paragraph{Configuration Parameters.}
Listing \ref{impl:config:parameters} shows the implemented parameters:
\texttt{sec} and \texttt{rt} define security and real-time groups,
\texttt{distr} and \texttt{clsvr} configure \gls{bsp} and client-server
communication groups.
The \texttt{\_C} or \texttt{\_S} flag marks a task as client or server.

\begin{lstlisting}[language={[5.2]Lua}, caption={Implemented set of
configuration parameters.},label={impl:config:parameters}]
min_prio  = <number>
max_prio  = <number>
sec       = <name>
rt        = <name>
distr     = <name>
clsvr     = <name>_{C|S}
\end{lstlisting}

The system designer must not define orthogonal parameters for
a single task.
Either a task is in a security or in a real-time group, never in both.
The same goes for \texttt{clsvr\_C} and \texttt{clsvr\_S}:
A task cannot be client and server within the same group.
Also, \texttt{distr} and \texttt{clsvr} may not be combined, as the
\texttt{distr} property is assumed if a client task runs more than one thread.
Additionally, the system designer may not define more real-time and security
groups than physical cores available.
As mentioned in \ref{design:isolation}, a real-time group could have more than
one core assigned, but this is currently not implemented.
In the future the name could be followed by a core count, to state the number
of cores necessary.
Then a combination of \texttt{clsvr} and \texttt{rt} makes sense, which is not
the case at the current state.

To configure the task a create call to the factory service of the load
balancer must be issued, containing the scheduler protocol as first parameter
and then a comma separated list of strings stating the configuration
parameters.
The factory will then return a capability on a channel to its proxy scheduler
encapsulating the task configuration.
\begin{comment}
Listing \ref{impl:config:proxy} shows an example for the described create call.

\begin{lstlisting}[language={[5.2]Lua},caption={Create call to the load
balancer factory containing protocol, priority band, and a client-server
group.},label={impl:config:proxy}]
tmFactory:create(
  L4.Proto.Scheduler,
  "min_prio = 0",
  "max_prio = 5",
  "clsvr = A_S")
\end{lstlisting}
\end{comment}

The capability received from the above create call replaces the
inherited scheduler in the task's environment.
As the scheduler interface has not changed, the task accesses the scheduler via
\texttt{L4Re::Env::env()->scheduler()}.
Listing \ref{impl:config:taskcreation} displays a complete task creation in
the Ned configuration script.

\begin{lstlisting}[language={[5.2]Lua},caption={Task creation with replaced
scheduler in Ned startup script.},label={impl:config:taskcreation}]
L4.default_loader:start(
  {
    caps = {},
    scheduler = tmFactory:create(
      L4.Proto.Scheduler,
      "min_prio = 0", "max_prio = 5", "distr = A")
  },
  "rom/app_matrixmul"
)
\end{lstlisting}

Therefore, it is only necessary to change the the application startup
configuration parameters, which are encapsulated in the Ned startup script,
shown in listing \ref{config:ned_full}.
The load balancing service is completely transparent to the application.

\lstinputlisting[language={[5.2]Lua}, firstline=3, caption={Complete Benchmark
configuration.}, label={config:ned_full}]{app_restart_mmul_parallel.cfg}


\paragraph{Configuration Groups.}

The configuration arguments stated in the startup script are parsed by the
factory service and stored in a \texttt{config\_t} type
(listing \ref{config:config_t}).
The parser strips all white spaces from the configuration string and splits it
in two parts: before and after the equality sign.
The front part evaluates to priority or group type, whereas the rear part
evaluates to either an integer or string.
%The front part evaluates to minimum or maximum priority, or group type and the
%rear part evaluates to either an integer or a string identifier for the group.
If the group type is a client-server type, the last character is matched
against \texttt{C} or \texttt{S} to recognize client and server tasks.

Note, this configuration is per task, hence all threads of this task possess
the same configuration.
A configuration group must not necessarily span across several tasks.
The \texttt{distr} parameter, for example, can group all threads of one task
and ensures that all threads of this task will be assigned to the maximal
number of available cores.
\\
\newline

\lstinputlisting[language=c++, caption={Configuration Type},
label={config:config_t}]{codeSamples/config.cc}
