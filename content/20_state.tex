% vim:set ft=tex:
\chapter{Technical Background}
\label{sec:state}

% Hier werden zwei wesentliche Aufgaben erledigt:

% 1. Der Leser muß alles beigebracht bekommen, was er zum Verständnis
% der späteren Kapitel braucht. Insbesondere sind in unserem Fach die
% Systemvoraussetzungen zu klären, die man später benutzt. Zulässig ist
% auch, daß man hier auf Tutorials oder Ähnliches verweist, die hier auf
% dem Netz zugänglich sind.

% 2. Es muß klar werden, was anderswo zu diesem Problem gearbeitet
% wird. Insbesondere sollen natürlich die Lücken der anderen klar
% werden. Warum ist die eigene Arbeit, der eigene Ansatz wichtig, um
% hier den Stand der Technik weiterzubringen? Dieses Kapitel wird von
% vielen Lesern übergangen (nicht aber vom Gutachter ;-), auch später
% bei Veröffentlichungen ist "Related Work" eine wichtige Sache.

% Viele Leser stellen dann später fest, daß sie einige der Grundlagen
% doch brauchen und blättern zurück. Deshalb ist es gut,
% Rückwärtsverweise in späteren Kapiteln zu haben, und zwar so, daß man
% die Abschnitte, auf die verwiesen wird, auch für sich lesen
% kann. Diese Kapitel kann relativ lang werden, je größer der Kontext
% der Arbeit, desto länger. Es lohnt sich auch! Den Text kann man unter
% Umständen wiederverwenden, indem man ihn als "Tutorial" zu einem
% Gebiet auch dem Netz zugänglich macht.

% Dadurch gewinnt man manchmal wertvolle Hinweise von Kollegen. Dieses
% Kapitel wird in der Regel zuerst geschrieben und ist das Einfachste
% (oder das Schwerste weil erste).


In this chapter I provide an overview over terminology and techniques I use
during this thesis.
I begin with an introduction to basic terminology in section \ref{state:term}.

Different information sources can enhance load balancing decisions regarding
\gls{smt} performance, resource usage, and communication relationships.
Besides these three performance oriented topics, a load balancer can also
improve isolation for security critical tasks and consider overall energy
usage.
I present state of the art research from these areas in section \ref{state:related}.

In section \ref{state:cfs}, I give an overview over Linux's completely fair
scheduler.
I discuss benefits and drawbacks of Fiasco.OC and L4Re in section
\ref{state:env}, before I introduce the targeted CPU architecture in section
\ref{state:haswell}.


\section{Terminology}
\label{state:term}

The following makes the reader familiar with the terminology of the field.
Readers may continue with section \ref{state:related} and use this section
as reference, if a term is ambiguous or unknown.

% todo Item sorting

\begin{description}
  \item[\Gls{utcb}] A \gls{utcb} is an user-level memory section containing
    information about a thread, the thread's communication registers and
    communication buffers. It is primarily used for message transfer.

  \item[\Gls{sc}] A \gls{sc} contains information for the scheduler consisting
    of a priority level, execution time, and remaining execution time.

  \item[Thread] A thread is a unit of execution, consisting of a kernel and
    \gls{utcb}, a \gls{sc}, and further interfaces.

  \item[Task] A task is a unit of isolation. It manages different address
    spaces; for example memory, capability, and I/O port spaces.
    Threads run within tasks, but tasks do not necessarily contain a thread.

  \item[Capability] A capability is a rights management concept.
    The system allows access to an object only if the caller provides a
    capability containing appropriate rights.
    It is orthogonal to access control lists, where the object knows, who may
    access it.

  \item[\Gls{ipc}] \Gls{ipc} is one way for threads in L4Re/Fiasco.OC
    to communicate with one another.
    It is synchronous communication with the following modes of operation:
    send, open wait, closed wait, reply \& open wait, and call.
    A thread initiating an open wait, waits for an arbitrary thread to contact
    him, whereas a closed wait, expects a communication intent of a specific
    thread.
    A call is normally used for communication with a service, where a reply is
    awaited, hence it combines a send and a closed wait operation.
    To initiate an \gls{ipc} in L4Re/Fiasco.OC the sending thread needs to hold
    a capability on a channel going to its communication partner.

  \item[\Gls{instpc}] \Gls{instpc} describe the number of instructions executed
    per processor cycle.

% todo Make context specific description lists?

  \item[Core] A physical CPU core is referred to as core.

  \item[Cache] Generally, a CPU cache is used to store frequently accessed data
    close to the CPU core to reduce access times.
    As a cache has size restrictions, depending on the type of cache different
    evocation policies can be applied, e.g. \gls{fifo} or \gls{lru}.
    There are multiple levels of caches in modern CPUs, creating a hierarchy;
    three levels are common.
    If modified data is evicted from the highest level of the hierarchy, it is
    written back to memory.
% todo device memory caches: fully-cached, not-cached, buffered

  \item[Migration] The process of moving a thread from one core to another core
    is called migration.
    The cores do not necessarily need to be on the same chip, nor do they need
    to be in the same machine. Only the thread is migrated, cached data remains
    on the old core.

  \item[Simultaneous Muli-Threading (SMT)] \gls{smt} is a technique to multiplex several hardware
    threads on the same physical core.
    Thus, sharing all the core's resources -- integer and floating point units,
    caches, prefetcher, etc. -- among the core's hardware threads.
    The goal is to improve the utilization of the execution units by increasing
    the number of instructions available to them.
    \gls{smt} transforms thread level parallelism to instruction level
    parallelism.
    A \gls{smt}-core is referenced as logical core or hardware thread.

  \item[\Gls{ht}] \gls{ht} is the \gls{intel} specific implementation of the
    \gls{smt}-concept.

  \item[\Gls{cmp}  \& \Gls{smp}] Regarding cache coherence, \gls{cmp} and
    \gls{smp} architectures are equal.
    However, \gls{cmp} style CPUs consists of several CPU chips soldered on one
    die.
    This leads to physically separated caches, limiting the amount of cache
    available to a single thread.
    In contrast, modern \gls{smp} architectures share the \gls{llc} among all
    cores on one die.

  \item[\Gls{isa}]\gls{isa} describes the set of hardware instructions executed
    by a processor.

  \item[Intervall] Describes a pre-defined amount of time.
  \item[Wall clock time] Describes the time passed in real life.
  \item[Execution time] Describes the time a thread spends executing on a
    core.

%  \item[Job] In scheduling theory a job is a specific unit of work.
%    In real-time systems, the execution time of a job is normally known beforehand.
%    \todo{difference between jobs and threads?}

  \item[Turnaround time] Time between submission of a job and its completion.

%  \item[Makespan] Time to execute a set of jobs from beginning to end.




\end{description}


\section{Related Work}
\label{state:related}

This section presents prior work and puts emphasis on the parts pertinent to
this work.
One restriction applies to this review: mechanism proposing hardware changes
are not in the focus of this work.
This restriction excludes interesting ideas to gather behavioural information,
e.g. observation of cache coherency traffic proposed in
\cite{cruz_dynamic_2014}.

\begin{comment}
  Structure for the description of related work:
    * Assumptions
    * Concept
    * Relevant contribution
    * Result
    * Deficits

  Alternative: Aspects of the thesis in related work;
\end{comment}

\begin{itemize}
  \item Assumptions: Single ISA, no heterogeneous cores, SMP, one specific CPU
    architecture
  \item Goals: Load Balancer providing isolation, stable execution times for the
    same workload, reduced congestion on shared ressources, two balancing
    strategies: Performance and low energy usage;
  \item introduce different approaches: speed balancing, symbiotic scheduling,
    congestion aware scheduling

    \begin{itemize}
      \item concicely introduce the work
      \item highlight the key aspects, goals, achievements
      \item relevant achievements I will use
      \item drawbacks of the work
    \end{itemize}
  \item symbiotic scheduling:
    \begin{itemize}
      \item technique for SMT to improve performance of threads scheduled on
	two hardware sharing SMT cores
      \item term coined by \citeauthor{snavely_symbiotic_2002} in 2000 using a
	simulator
      \item also used for CMP systems, but less shared HW (later)
      \item runs threads on two corresponding SMT cores and measures their
	performance;
      \item explain performance, throughput; how measured?
      \item sampling phase needs time, but can be done, while a default
	scheduler runs
      \item restriction on changes in workload to ease sampling phase
      \item summarize all restrictions/assumptions
      \item select thread pairs/co-schedules for smt-core-pairs with most throughput for
	all threads
      \item try to combine workloads that use different cpu hardware
	units, e.g. fp workload and integer workload; heavy cache usage, little
	cache usage
      \item Meausrements in their \citeyear{snavely_symbiotic_2000} paper use perf counters not present in
	real HW, e.g. conflics inside FP units, FPqueue full.
      \item  \citeauthor{snavely_symbiotic_2002}: Simulator, no real hardware, but priority
	sensitive (2002)
      \item priority is perceived as ``guarantee a fraction of machine
	proportional to priority'' or `` guarantee a fraction of
	single-threaded performance proportional to priority''
      \item \citetitle{banikazemi_pam_2008}: Use real CMP system: 8 core two socket IBM Blade
      \item relies on Linux scheduler as default
      \item only when better performing thread-to-core mapping is predicted, the
	default scheduler is overruled for a defined count of scheduling-cycles
      \item uses hardware performance counter evaluate and support decisions
      \item goodness metric to rate the thread-core-mapping
      \item mathematical model for the algorithm
      \item \citetitle{eyerman_revisiting_2015}: Symbiotic scheduling does only provide an average
	throughput gain of around 3%.
      \item reason: the most test case assumptions in literature are benifical
	for symbiotic scheduling, but not close to reality.
      \item First-come-first-served scheduler is close to the theoreticaly
	optimal scheduler
      \item theoretical optimal schedule knows the performance of all tasks in
	all possible combinations with all other tasks.
      \item use simulator to measure tasks on a 'reference core' and
	compute throughput and IPC values for different schedules.
      \item arrival time of tasks is critical for FCFS schedule throughput and
	hence determines, how close FCFS comes to the optimal scheduler
      \item processor utilization or empty time are better indicator of
	throughput improvement than turnaround time
      \item overload situations?
    \end{itemize}

  \item congestion aware scheduling
    \begin{itemize}
      \item goal: minimizing usage of shared hardware resources
      \item focus on shared caches, but also prefetching hardware and memory
	controller
      \item options: cache line reuse vs. miss rate
      \item miss rate expresses load on prefetching hardware and mem controller
      \item
    \end{itemize}

  \item security relevance of shared ressources
    \begin{itemize}
  \item cache side channel attacks undermining security of OpenSSL; how thread
    placement minimizes side channel surface;
    \end{itemize}

  \item what is relevant and will be used, where improves this work the
    previous (contribution)
  \end{itemize}

\paragraph{ \cite{sarma_smartbalance_2015} }
\citeauthor{sarma_smartbalance_2015} write about their approach to balance work
in a heterogeneous system.
They assume a system with a single ISA and several CPUs with different performance
characteristics and different hardware features.
\todo{check different hw features}
They introduce categories for each performance level and measure the
performance differences between each level.
This results in a matrix displaying the performance gain or degradation, when
a work package (e.g. thread) is migrated.
Their load balancing algorithm has three phases: sense, predict, balance.
During the sense phase the algorithm observes the CPU utilization for an intervall.
An intervall consists of configurable many clock ticks.
The sense-result is then used by the prediction phase to compute a load
expectation for the different performance levels.
Based on this forecast a balancing decision is made and enforced in the third
phase.
Their model can also select the best CPU size for a optimal performance per
joule ratio.

\textbf{relevant} load balancing algorithm: sense, prediciton, balance,
intervall;

\textbf{not} heterogeneous processors, gem5, big.LITTLE, performance
matrix;

\paragraph{ \cite{hofmeyr_load_2010} }

\textbf{relevant} notion of speed of a processor: $speed = t_{exec}/t_{real}$;
sched\_setaffinity to migrate threads; PIDs of all the threads of a task;
core speed computation and coparison algorithm;
distributed balancing thread with global synchronization;
assumptions \& drawbacks of kernel level load balancing;
in oversubscibed environments balance trumps locality;


\textbf{be aware} turbo boost -> different core speeds on same CPU;
sched\_yield and sleep difference: run queue;

\textbf{not} NUMA, MPI, OpenMP;

\textbf{shortcommings} Tigerton: No \gls{ht}, no turbo-boost, no l3 cache,
and a dual-die CMP architecture.


\paragraph{ \cite{zhuravlev_survey_2012} }
\citeauthor{zhuravlev_survey_2012} provide a survey over scheduling techniques,
aiming at better usage of shared resources in multicore processors.
They focus on \gls{cmp}s, which share caches between at
least two cores, but not between all cores.
Techniques like hyper-threading are also taken into account.
\todo{redo this section, cmp, smp already in terminology}
Simultaneous multiprocessors (SMPs) are not in the focus of their survey.
\gls{intel} newer architectures, e.g. Haswell, is a \gls{smp} architecture, where each
physical core has dedicated L1 \& L2 caches, but the L3 cache is shared between
all cores.
Their algorithmic focus lies on contention-aware schedulers, which consists of
four building blocks: objective, prediction, decision, and enforcement.


\paragraph{ \cite{knauerhase_using_2008} }
In \citetitle{knauerhase_using_2008} \citeauthor{knauerhase_using_2008} present
a mechanism to observe \gls{llc} misses and references, retried instructions,
core cycles, and reference cycles.
These information describe the behaviour of the threads in the system.
They assume a \gls{cmp} system and address three issues: cache interference,
migrating threads across caches and fairness between threads.

To reduce interference between cache heavy and light threads, they use the
following heuristics: cache miss per cycle, behaviour in the last time quantum,
and sum of the cache weights.
The goal is to run cache heavy threads on different \gls{llc}-groups, meaning
cores that share the same \gls{llc}.
The cache miss per cycle has shown to be a good heuristic, as it also displays
the memory access and hence the load on the memory bus.
Also, behavioural history longer than the last time quantum has proven
unnecessary, as the thread behaviour changes over time and longer history
clouds the prediction.
Finally, the cache weight is the sum of the weight of each thread running on
the same \gls{llc}-group.
A threads cache weight is the number of cache entries it uses. \todo{check
that}

To decide when and which threads to migrate to another \gls{llc}-group, the
thread weight and the cache load per \gls{llc}-group is computed.
New threads are assign tho the \gls{llc}-group with the smallest cache load.
To achieve better fairness between the \gls{llc}-groups, overweight threads are
migrated between groups periodically.
In general, migrations between \gls{llc}-groups are prevented, as the migrated
thread has to repopulate the cache on the new core.

To increase the fairness between cache light and cache heavy threads,
\citeauthor{knauerhase_using_2008} propose to increase the execution time of
cache light threads, as their performance suffers from running besides cache
heavy threads.
\todo{How is performance defined?}

The authors wrote a cachebuster and spinloop thread to simulate cache heave and
cache light threads. Additionally, application from the SPEC CPU 2000 benchmark
suite were used to provide further experimental prove to their claims.

\textbf{ shortcommings } CMP system


\paragraph{ \cite{yarom_recovering_2014} }
\textbf{motivation} FLUSH+RELOAD cache side channel attack;

\paragraph{ \cite{bernstein_cache-timing_2005} }
\textbf{motivation} timing side channel attack against cache;


\paragraph{ \cite{eyerman_revisiting_2015} }
\textbf{relevant} symbiotic scheduling on SMT systems;
list of assumptions;
definition of their experiments;
def. partially symmetric homogeneous multicores;
calculation of optimal throughput of a processor;


\paragraph{ \cite{fedorova_managing_2010} }
\textbf{assumptions:} no interaction between threads: no shared data, no
communication;

\begin{itemize}
  \item LLC miss rate of threads \textit{vs.} memory-reuse pattern
  \item more arguments for mem-reuse pattern
  \item cache sensitivit and intensity of threads --> Pain metric (offline)
  \item cache-miss and cache-access rate
  \item online metric to approximate pain metric -> approx-pain
  \item approx-pain uses perfcounters to measure LLC-miss-rate, as this showed
    to be the best predictor for sensitivity and intensity
  \item LLC-miss-rate predicts contention in other shared hardware
  \item LLC-miss-rate perf counter measures prefetching miss
  \item FURTHER: Distributed Intensity Online, Power Distributed Intensity
\end{itemize}

\textbf{Drawbacks}
\begin{itemize}
  \item no interactions between threads
  \item contention focus, cooperation ignored
  \item CMP processors used; SMP architecture fundamentally different;
\end{itemize}

\textbf{Gains}
\begin{itemize}
  \item overall completion time as scheduler performance metric
\end{itemize}

\paragraph{ \cite{zhuravlev_addressing_2010} }
\begin{itemize}
  \item cache aware scheduler needs: classification scheme \& scheduling policy
  \item classification: cache light/heavy; compute light/heavy;
  \item scheduling: assignment of threads to cores based on their classification
  \item miss rate is good estimator of contention for shared ressources, as it
    counts the LLC-misses for CPU accesses and prefetching accesses. Hence, it
    measures load on FSB, DRAM ctr. and prefetching HW besides the cache
    contention.
  \item centralized sort by LLC-miss-rate per million instructions
  \item used 8core, dual socket opteron, with a cache layout similar to Haswell
  \item thread count <= core count
  \item running average miss rate for scheduling decisions
\end{itemize}

\textbf{Gains}
\begin{itemize}
  \item better average performance
  \item high performance improvement for individual applications
  \item low performance variance (best/worst case) between different runs -->
    stable performance
  \item factors for performance degradations: memory controller, FSB,
    prefetching HW
  \item classification scheme evaluation
  \item 8core opteron machine is cache-layout comparable to Haswell
\end{itemize}

\textbf{Drawbacks}
\begin{itemize}
  \item no cooperation between threads assumed (shared mem, communication)
  \item no overload situation, at much as many threads as cores in the system
\end{itemize}


\paragraph{ \cite{liu_last-level_2015} }
\textbf{questions} the ability of reducing the surface for cache
side-channel attacks;


\paragraph{ \cite{ousterhout_scheduling_1982} }
\textbf{constrains} multiprocessor systems from the '80s.

\textbf{relevant} coscheduling introduced;

\paragraph{ \cite{watts_practical_1998} }
\textbf{constrains} discusses load balancing in a distributed network of
machines, e.g. cluster;

\textbf{relevant} definition of static and dynamic load balancing;
five phases for dynamic load balancing: load evaluation, profitability
determination, work transfer vector calculation, task selection ,task migration


\begin{itemize}
  \item Given a collection of tasks comprising a computation and a set of
computers on which these may be executed, find a mapping of tasks to computers
that results in each computer having an approximately equal amount of work.
  \item first determine that a load imbalance exists
  \item if the cost of the imbalance exceeds the cost of load balancing then
    load balancing should be initiated
  \item work transfer vector calculation, how much work shall be transfered
  \item task selection, constrained by locality and task size, cost function to
    take this into account;
  \item task migration, state \& communication integrity must be maintained
\end{itemize}


\paragraph{ \cite{banikazemi_pam_2008} }
\textbf{relevant} Cpusets to describe hardware architecture hierarchies;
\gls{llc}-sharing, \gls{llc}-separate, power/energy-aware cpusets;
only high-level task-to-cpuset mapping -- scheduler does the rest;
multi-level optimizations;
IDEA: security cpuset;

\textbf{keep in mind} Model for $n$-CPU system;
number of sched. choices;
measure occupancy and miss ratio and CPI;
estimate performance of scheduling choices;
benchmarking process;

\textbf{shortcommings} CMP system


\paragraph{ \cite{zhang_processor_2007} }
\textbf{relevant} bottlenecks, metrics, sched policies (IPC \& memory bus
accesses);


\section{Assumptions}
The mentioned research leads to the following assumptions for this work:


\begin{description}
  \item[Hardware]
  \item[Job types]
  \item[Job type distribution]
  \item[knowledge of execution time]
  \item[Energy efficency] In case of overload equal to performance algorithm

\end{description}

keywords:
online
load balancing
thread-to-core-mapping
meta-scheduler
user-land
on a $µ$-kernel

\paragraph{Tradeoff Table}
thread bahaviour history over several intervalls
\textbf{vs.}
thread behaviour only in last intervall


\paragraph{Contributions}
\begin{itemize}
  \item Current processor architectures don't use CMP processors any more, therefore
    the cache layout is different. My work evaluates the research results on the
    new HW layout. But AMD Opteron Barcelona had a similar cache layout.
  \item User level scheduling on a $µ$-kernel. But user level scheduling was
    done before.
  \item More scheduling parameters \textit{or} less assumptions about threads.
  \item No offline measurements, only online information gathering.
  \item Thread interaction possible (communication partner, security flag)
  \item Designated cores for security critical applications.
\end{itemize}


% vim:set ft=tex:
\section{Linux Completely Fair Scheduler}
\label{state:cfs}

\newacronym{cfs}{CFS}{Completely Fair Scheduler}

Since Linux version 2.6.23, Linux uses the\gls{cfs} (\cite{linux_cfs_doc}).
It aims at complete fairness, in regard to execution time provided to each task.
To achieve fairness, it measures so called virtual runtime of each task and
selects the task with the lowest account to run next.
Internally, \gls{cfs} uses a red-black tree to sort the tasks regarding virtual
runtime,

\gls{cfs} also supports group scheduling to provide fairness between tasks with one
thread and tasks with many threads.
If a task spawns many threads, they all execute on the account of the main
task, preventing one task to acquire more execution time than others.

Conceptually, \gls{cfs} runs the currently selected task until the virtual runtime is
higher than the virtual runtime of any other task.
Then a task switch occurs.
As this would lead to constant task switching, two measures are used to
counteract this: Target scheduling latency and minimum task runtime granularity.

The target scheduling latency (TSL) describes an amount of time shared between a
number of tasks.
If two tasks are runnable, and the TSL is 20ms, then each task runs for 10ms.
However, if more tasks become runnable, the amount of time per task shrinks and
eventually the switching overhead is higher than the time a task spends
executing.
To prevent this situation, the minimum task runtime granularity defines a
minimum amount of time a task needs to execute.
As the number of tasks grows, the TSL increases to be able to schedule each
runnable task once for the minimum task runtime.

Priorities are accommodated by weighting the virtual runtime accounted to each
task depending on the priority level.
The virtual runtime of a high priority task increases slower than the virtual
runtime of a low priority task, leading to more actual execution time for the
high priority task.

\paragraph{Domains, Groups, \& Load Balancing.}
In \gls{cfs} a domain represents a unit of physical hardware;
for example, a physical core is a domain and the two hyper-threads running
inside each constitute a group within this domain.
On each level of the hierarchy, the current level is the domain and the lower
hierarchy levels are groups.
Each group within a domain is equivalent to other groups in the same domain
(\cite{lwn_sched_domains}).

\gls{cfs} does active load balancing only between groups in the same domain.
Their load is compared and if imbalances are found, \gls{cfs} tries to migrate
a task from one group to another to establish balance again.
A cache affinity flag indicates, if the task ran recently and might
still have valid cache lines.
If the cache affinity flag is set, the task is not considered for migration.


% vim:set ft=tex:
\section{Fiasco.OC \& L4Re}
\label{state:env}
\todo{Fiasco.OC \& L4Re}

\textbf{Fiasco.OC}
\begin{itemize}
  \item Kernel scheduler does no balancing, assigns thread to the first
    core specified in the affinity descriptor
  \item affinity descriptor: core(s) a thread should run on
  \item Syscall via run\_thread() to pass affinity descr to kernel scheduler
  \item interface to query execution time for each thread
  \item capability system -- to derive communication relationships from
  \item	Kernel feature wishes derived from related work: Performance counters
    and per thread accounting
\end{itemize}

\textbf{L4Re}
\begin{itemize}
  \item provides scheduler proxy interface, including affinity descriptor,
    scheduling parameters
  \item syscall interface
\end{itemize}


% -----------------------------------------------------------------------------

\section{\gls{intel} Haswell Architecture}
\label{state:haswell}

In section \ref{state:related} several hardware features were mentioned.
The most important of all are performance counters for cache misses, cycles and
instructions executed.
Those and many more are present in the target processor: an \gls{intel} Core
i7-6440K.
The processor has ``architectural performance monitoring version 3''
capabilities, which consists among other features of three fixed-function
performance counters counting Instruction Retired, Unhalted Core Cycles, and
Reference Instruction Retired.
In contrast to the first two counters, the last counter is unaffected by
processor speed changes due to power saving or turbo boost features.
Besides these, four general purpose performance counters are available per
logical core, which can be programmed to count one specific event.
All \gls{intel} Core i generations support architectural performance monitoring
in different versions, but all versions are guaranteed to support the events
listed in table \ref{state:table:core_events}: Unhalted Core Cycles,
Instruction Retired, Unhalted Reference Cycles, LLC Reference, LLC Misses,
Branch Instruction Retired, and Branch Misses Retired.

Besides these central events, each hardware generation supports a different set
of so called ``non-architectural performance events''.
This non-architectural event set allows to monitor many specific events, e.g.
L2-misses/-hits, micro-ops per logical core executed per cycle, or unhalted
core cycles, while the logical core is in ring 0.

Another difference to the target processor, compared to most processors used in
earlier research is the \gls{smp} architecture. In contrast to \gls{cmp}
architectures no cache groups are defined by the hardware, because the
\gls{llc} is shared among all cores.
Figure \ref{state:fig:core_layout} shows examplary the cache hierarchy and core
components of the \gls{intel} Haswell processor.
Each core consists of two \gls{ht} cores, sharing L1- and L2-caches.
Besides the \gls{llc} the cores on one socket share nothing.
The caches are inclusive, meaning each cache line present in L1I or L1D
cache is also present in the corresponding L2 cache and each L2 cache line is
also present in the \gls{llc}.
This eases the lookup of cache lines in other cores on the same package.

%\includegraphics[width=0.8\textwidth]{images/haswell_architecture_by_intel_large}

\begin{figure}[h!]
  \centering
  \includegraphics[width=0.7\textwidth]{images/haswell_core_layout}
  \caption{Layout of a Core i7-4660K quad-core.
    Each physical core has two logical cores T0 and T1 sharing the cores L1I-,
    L1D-, and L2-cache.}
  \label{state:fig:core_layout}
\end{figure}


\begin{itemize}
  \item diagram of architecture: four cores with L1I/D \& L2 cache; two smt/ht
    cores per physical core; L3 cache shared and sliced, ring buffer for
    access; mem controler in uncore package;
  \item 4 prefetcher per cache; 2 for L1D, 2 for L2 cache
  \item L1D \& L2 cache data is present in L3 cache to be able to redirect
    requests from other cores to the correct cache. --> Issue with security; no
    cache side attack surface reduction
  \item IF security: dual socket system with one core dedicated to security
    tasks for an interval
  \item L3 cache slices corespond to number of cores
  \item logic portion and data array portion; access, coherency, memory
    ordering, LLC misses, writeback to memory; cache lines
  \item hash function uniformly distributes addresses
  \item access times to L3 cache varies depending on travel distance on the
    bi-directional ring buffer
  \item system agent receives memory requests not serviced by cache and
    redirects to IMC
\end{itemize}

\paragraph{Issues with security core idea:}
The idea to reduce attack surface for e.g. openssh cache attacks was to provide
a dedicated core for security critical applications.
This idea spawned from the assumption that L1D \& L2 cache was solely present
on a core, but the L3 cache content is a superset of all cores L1D \& L2 cache.
Hence, cache side channel attacks are still possible, although more
complicated, as the attacker not present on the core, must determine, which
cache lines correspond to the L1D \& L2 cache of the ``security core''.
Attacks presented in \cite{yarom_recovering_2014} and
\cite{bernstein_cache-timing_2005} rely on the fact, that only one application
is using cache lines on the core.
On a multicore, only observing the \gls{llc}, other running applications are
expected to disturb the observations, effectivly reducing the side channel
throughput.
\todo{This should be part of the architecture chapter, as the security core
idea wasn't presented yet.}

% -----------------------------------------------------------------------------



\section{Summary}
\label{state:summ}
\todo{rename to Summary, do summary of chapter}

\begin{itemize}
  \item introduction to relevant terminology
  \item SMT, symbiotic scheduling, co-schedule, measurement difficulties
  \item Congestion awareness of shared resources; Goal minimal slowdown for all
    applications; fairness, cache weights, pain metric, reliance on default
    scheduler, performance counters
  \item communication awareness: compute-communicate cycle, distribution of one
    application
  \item don't haves: No CMP architecture -> SMT-SMP combination; client server
    communication model;
  \item Fiasco.OC: no logic, just mechanisms in the kernel, hence no balancing;
  \item L4Re: no automatic, behaviour based balancing.
  \item do: communication models: HPC, CL-SVR; behaviour analysis: perf
    counters; SMT-SMP processor awareness, reduced conflicts in shared hardware
    resources, only on-line measurements;
  \item secure/real-time/exclusive core design.
\end{itemize}


\begin{itemize}
  \item Current processor architectures don't use CMP processors any more, therefore
    the cache layout is different. My work evaluates the research results on the
    new HW layout. But AMD Opteron Barcelona had a similar cache layout.
  \item User level scheduling on a $µ$-kernel. But user level scheduling was
    done before.
  \item More scheduling parameters \textit{or} less assumptions about threads.
  \item No off-line measurements, only on-line information gathering.
  \item Thread interaction possible (communication partner, security flag)
  \item Designated cores for security critical applications.
\end{itemize}


% -----------------------------------------------------------------------------


\cleardoublepage

%%% Local Variables:
%%% TeX-master: "diplom"
%%% End:
