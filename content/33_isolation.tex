% vim:set ft=tex:
\section{Isolation}
\label{design:isolation}

Fiasco.OC provides isolation and also real-time guarantees.
In the following, I show how a requirement-aware thread-to-core mapping of a
load balancer can support these guarantees.

\paragraph{Security.}
As presented in section \ref{state:related}, even in virtualized environments
the CPU cache is a valid side-channel to attack secret information.
To reduce the channel's throughput, security-critical applications should share
as little cache as possible with untrusted applications.
In Haswell's inclusive cache hierarchy, the \gls{llc} contains the lines
present in the core local L1 and L2 caches and is shared among all cores of a
chip.

As shown in \cite{inci_seriously_2015} and \cite{liu_last-level_2015},
the impact of placing security critical tasks on a seprate core is limited.
Still, a security core increases the difficulty for PRIME+PROBE attacks.
To completely mitigate this channel, no caches may be shared between the
security-critical task and the attacker.
Nowadays, only a multi-socket system or \gls{intel}'s Cache Allocation
Technology can enforce this (\cite{intel_cat}).
\\

The security requirements of a task are known at system design.
Therefore, a way to provide this static information to the load balancing
service is necessary.
It is also preferable to define specific security groups to separate different
security critical tasks or to create a group of related, equally trusted
security tasks.
Hence, the configuration scheme must allow to define groups and identify the
same group in the configurations of different tasks.

This can be achieved by a per task configuration parameter identifying
different groups by name.


\paragraph{Real-Time.}
A similar isolation guarantee is necessary for real-time tasks.
Real-time tasks need predictable execution times.
By using an exclusive core and cache, real-time tasks are less affected by the
influenced of best-effort workloads.
Due to the inclusive cache design, evictions from the \gls{llc} require
evicting the cache line in the core local caches as well.
Hence, this approach limits the possible disturbance, but does not eliminate them.
Also, several cores can be assigned to a real-time task group, depending on its
requirements.

As the requirements of real-time tasks are also known before run-time, a static
configuration can be used.
The configuration parameter must identify different groups by name
and display the physical core requirements.
