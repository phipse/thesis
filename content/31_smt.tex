% vim:set ft=tex
\section{SMT}
\label{design:smt}

\paragraph{co-schedule}
As mentioned in \ref{state:related} a co-schedule describes two threads running
in parallel on the same physical core, but in different \gls{smt}-threads.
A good co-schedule has the least performance degradation,
compared to executing each application without a co-runner.
Physical hardware resources are shared between co-runners, hence, two threads with
different resource needs are expected to impair each other the least, as they
compete little on the same resource.
So the main criteria for a good co-schedule is little conflict potential
between co-runners.

\paragraph{performance expectation}
But how can a performance expectation be formulated, without offline
measurements or an online measurement phase?
If a thread is scheduled the first time, no performance measurement is
available.
The thread can be placed on the core with the least threads or with the lowest
cache weight. \todo{did I explain cache weight already?}
At the next interval boundary, the thread is measured and can be placed by the
chosen scheme, e.g. cache weight.

\paragraph{performance measurement}
But does this measurement at interval boundaries actually describe the
performance of the co-schedule?
Three cases must be considered during an interval:
(A) a thread runs solo on a physical core,
(B) a thread runs together with one other thread on a physical core and both
threads execute on one logical core, and
(C) more than two threads run on the same physical core.
In case (A), the measurement describes the solo execution performance and
resource usage.
Case (B)'s measurement actually provides the co-schedule performance for both
threads.
And case (C) provides no reliable information at all.
If three threads are present, two of them, say \alpha{} and \beta{}, are assumed to
execute in parallel at each point in time.
During the interval the combination will change, so \alpha{} and \gamma{} will be
co-scheduled for some time and likewise \beta{} and \gamma{} and \alpha{} and
\beta{}.
The measurement takes place at the end of an interval, with no information
about how long which thread combination was executing together. So no
co-schedule performance statement for each pair of threads can be made.

This example is simplified to ease the argumentation, in reality two additional
problems must be considered.
First, the \gls{llc} of the target hardare is shared among all four physical cores,
hence the performance of each core local co-schedule is influenced by
co-schedules on other cores, too.
Second, the possible thread combinations are either (\alpha, \beta) and
(\gamma, \beta), (\alpha, \beta) and (\alpha, \gamma), or (\alpha, \gamma) and
(\beta, \gamma), because no scheduling decisions take place during an interval.
So one of the three threads can execute for the whole interval and the other
two execute for half an interval each.

The following three questions are still unanswered:
Which information regarding the performance of a group of threads on two
corresponding logical cores can be derived at the end of an interval?
How can this measurement be compared to other groups?
What measures need to be taken after the comparison to improve performance?

Instruction per cycle seems to be a good measure to rate the performance of a
co-schedule group.
\todo{ is it clear to the reader what a co-schedule group is, or do I need to
define this  explicitly}
If the co-runners at a point in time have a high conflict
rate on shared resources, the amount of instructions executed per cycle will
suffer.
For example, if two co-runners execute integer operations, the number of
instructions per cycle, will be the number of integer units physically available.
However, if one co-runner executes integer and the other floating point
operations, this number will change to the number of integer units plus
the number of floating point units, leading to a higher number of instructions
per cycle.

But is this number representative? No. Consider three threads, one executing
floating point operations and two executing integer operations. Obviously, the
one executing floating point operations should run alone on one logical core.
But this property is
unknown at runtime and cannot be derived from the measurements.
Hence, all three different combinations must be executed, before a concise
statement regarding the best possible performance achievable for this 
co-schedule group can be made.

\paragraph{abstraction}
\todo{after this discussion, it seems the abstraction needs to be part of the
implementation chapter}
