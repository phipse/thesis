% vim:set ft=tex:
\section{\gls{smt}/\gls{ht}}
\label{design:smt}

\newacronym{eu}{EU}{execution units}

At the lowest level, a load balancer must balance the work of two corresponding
\gls{ht} cores.
% todo improve connection

The performance gains of \gls{ht} stem from the conversion of thread level
parallelism to instruction level parallelism, by providing more instructions to
the \gls{eu} to fill their pipelines.
But if two threads issue the same group of instructions, no performance gain is
possible, as the number of \gls{eu} has not increased.
Hence, two threads with good \gls{ht} performance use different \gls{eu}s.

The term co-schedule, coined in \cite{snavely_symbiotic_2000}, describes two
threads running at the same time on two corresponding logical cores.
Two questions for this work are: how can the performance of a co-schedule be
measured? And: is such a measurement comparable to previous ones?

From a hardware perspective the question is: How high is the \gls{eu}
utilization of a co-schedule?
A performance counter observing all executed
micro-ops\footnote{UOPS\_RETIRED.ALL} could answer this question.
But it is questionable, if this measure reflects the performance
degradation each thread suffers due to sharing execution resources.
To get the degradation percentage, knowledge about solo execution is necessary,
which is either gathered offline beforehand, or by scheduling each thread alone
for some time and extrapolating the measurement to the thread's lifetime.
The former is excluded due to the scope of this thesis, the latter harms
overall system performance and is imprecise, if the thread's behaviour changes
heavily.

Besides fine granular performance measure and relying on estimations, a third
problem with rating \gls{ht} performance is the length of a balancing interval
and the possibility of running more than two threads during this interval on a
pair of logical cores.
If more than two threads are running during an interval, the load balancer can
only rate the group performance, because it is impossible to deduce which
of the
$2^n-1$\footnote{Stirling numbers of the second kind: S(n,k), n = \#objects, k =
\#partitions;\\special case for k = 2: $S(n,2) = 2^n-1$}
possible co-schedules ran for which amount of time.
Consequently, the group rating describes the performance of a physical core.

The same applies to performance comparison between hyper-threads. As the
performance of a single hyper-thread is influenced by its co-runner, it cannot
be compared to logical cores of different physical cores.

All this necessitates an abstraction from logical cores.
This abstraction allows for the comparison of physical core performance and
lets the load balancer act independently of the presence of \gls{ht}.
