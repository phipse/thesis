% vim:set ft=tex:
\chapter{Implementation}
\label{sec:implementation}

% Hier greift man einige wenige, interessante Gesichtspunkte der
% Implementierung heraus. Das Kapitel darf nicht mit Dokumentation oder
% gar Programmkommentaren verwechselt werden. Es kann vorkommen, daß
% sehr viele Gesichtspunkte aufgegriffen werden müssen, ist aber nicht
% sehr häufig. Zweck dieses Kapitels ist einerseits, glaubhaft zu
% machen, daß man es bei der Arbeit nicht mit einem "Papiertiger"
% sondern einem real existierenden System zu tun hat. Es ist sicherlich
% auch ein sehr wichtiger Text für jemanden, der die Arbeit später
% fortsetzt. Der dritte Gesichtspunkt dabei ist, einem Leser einen etwas
% tieferen Einblick in die Technik zu geben, mit der man sich hier
% beschäftigt. Schöne Bespiele sind "War Stories", also Dinge mit denen
% man besonders zu kämpfen hatte, oder eine konkrete, beispielhafte
% Verfeinerung einer der in Kapitel 3 vorgestellten Ideen. Auch hier
% gilt, mehr als 20 Seiten liest keiner, aber das ist hierbei nicht so
% schlimm, weil man die Lektüre ja einfach abbrechen kann, ohne den
% Faden zu verlieren. Vollständige Quellprogramme haben in einer Arbeit
% nichts zu suchen, auch nicht im Anhang, sondern gehören auf Rechner,
% auf denen man sie sich ansehen kann.

In this chapter, I explain implementation details of the load balancer.
I start with the description of the kernel implementation of the performance
counter, together with the user-land interface to query said counters.
Then, I describe the task configuration and processing of the configuration by
the load balancer.
I follow up with a description of internal data management during a balancing cycle.
Subsequently, I present load balancing algorithms.

% vim:set ft=tex
\section{Kernel Performance Counter}
\label{impl:perfcounter}

\gls{intel} provides hardware performance counter for various CPU
architectures.
Software developers should be enabled to observe the behaviour of their
software and examine bottlenecks or performance hotspots.

The performance monitoring capabilities of the targeted Haswell architecture
were presented in section \ref{state:haswell}.

The present kernel implementation for architectural performance monitoring is
extended to use the fixed-function counters for unhalted core cycles and
instructions retired.
Additionally, a general purpose counter for \gls{llc} misses is used,
leaving three counters for future use.

\paragraph{Configuring \& starting the counters}
Each general purpose performance counter consists of a counter register and a
configuration register.
The configuration register must be filled with the counter \texttt{UMask} and
\texttt{Event Select} to specify the event which should be counted.
Then user and operating system mode bits must be set, to count events happening
in both, privileged and unprivileged mode.
With the edge detect bit set, every new occurrence of the selected event is
counted.
If the event was present and happens again before it was counted, it counts as
a single event.
At last the enable bit must be set to start counting, but only if the counter
is also enabled in the global control register.

Fixed-function counters consist of a counter and a shared control register to
en- and disable each one.

But both counter types only start counting, if they are enabled in the global
performance monitoring control register.
This register contains a bit for each counter, which must also be set to enable
the counter.

Fiasco.OC already contains the boot parameter \texttt{-loadcnt} enabling the
unhalted core cycle counter on each core.
Although a fixed-function counter is available for this event, the
implementation still used a general purpose counter.
Using this flag, all mentioned events are configured and enabled on each core.

\paragraph{Accounting per thread}
% todo find out, where and when thread perf counting is enabled?
After performance counting is enabled for a thread,
Fiasco.OC adds the counter values on each context switch to the thread's
account and resets the hardware counter to zero.


\paragraph{user land interface}
Two new thread operations are introduced: \texttt{perf\_read} and
\texttt{perf\_reset}.
\texttt{Perf\_read} fills a passed struct with the values passed from the
kernel, whereas \texttt{perf\_reset} tells the kernel to set all counters to zero.

\begin{lstlisting}[language=c++]
enum L4_thread_ops {
  ...
  L4_THREAD_PERF_READ_OP	      = 10UL,
  /**< Read thread's performance counter */
  L4_THREAD_PERF_RESET_OP	      =	11UL,
  /**< Reset thread's performance counter */
  ...
}

struct perf_counter_read_t {
  l4_uint64_t core_cycle_cnt;
  l4_uint64_t instr_retired_cnt;
  l4_uint64_t llc_miss_cnt;
};
typedef struct perf_counter_read_t perf_counter_read_t;

L4_INLINE l4_msgtag_t
l4_thread_perf_read(l4_cap_idx_t thread,
                    perf_counter_read_t *cntr) L4_NOTHROW;

L4_INLINE l4_msgtag_t
l4_thread_perf_reset(l4_cap_idx_t thread) L4_NOTHROW;
\end{lstlisting}

% vim:set ft=tex:
\section{Task Configuration}
\label{impl:config}

In sections \ref{design:isolation} and \ref{design:comm}, I discussed
task properties relevant to a load balancer.
The result of these discussions was to provide the load balancer with a static
configuration of a task's properties.
In th following section I show how the configuration is used and implemented.

\paragraph{Configuration Parameters.}
Listing \ref{impl:config:parameters} shows the implemented parameters:
\texttt{sec} and \texttt{rt} define security and real-time groups,
\texttt{distr} and \texttt{clsvr} configure \gls{bsp} and client-server
communication groups.
The \texttt{\_C} or \texttt{\_S} flag marks a task as client or server.

\begin{lstlisting}[language={[5.2]Lua}, caption={Implemented set of
configuration parameters for priority range, isolation, and communication
groups.},label={impl:config:parameters}]
min_prio  = <number>
max_prio  = <number>
sec       = <name>
rt        = <name>
distr     = <name>
clsvr     = <name>_{C|S}
\end{lstlisting}

The system designer must not define orthogonal parameters for
a single task.
Either a task is in a security or in a real-time group, never in both.
The same goes for \texttt{clsvr\_C} and \texttt{clsvr\_S}:
A task cannot be client and server within the same group.
Also, \texttt{distr} and \texttt{clsvr} may not be combined, as the
\texttt{distr} property is assumed if a client task runs more than one thread.
Additionally, the system designer may not define more real-time and security
groups than physical cores available.
As mentioned in \ref{design:isolation}, a real-time group could have more than
one core assigned, but this is currently not implemented.
In the future the name could be followed by a core count, to state the number
of cores necessary.
Then a combination of \texttt{clsvr} and \texttt{rt} makes sense, which is not
the case at the current state.

To configure the task a create call to the factory service of the load
balancer must be issued, containing the scheduler protocol as first parameter
and then a comma separated list of strings stating the configuration
parameters.
The factory will then return a capability on a channel to its proxy scheduler
encapsulating the task configuration.
\begin{comment}
Listing \ref{impl:config:proxy} shows an example for the described create call.

\begin{lstlisting}[language={[5.2]Lua},caption={Create call to the load
balancer factory containing protocol, priority band, and a client-server
group.},label={impl:config:proxy}]
tmFactory:create(
  L4.Proto.Scheduler,
  "min_prio = 0",
  "max_prio = 5",
  "clsvr = A_S")
\end{lstlisting}
\end{comment}

The capability received from the above create call replaces the
inherited scheduler in the task's environment.
As the scheduler interface has not changed, the task accesses the scheduler via
\texttt{L4Re::Env::env()->scheduler()}.
Listing \ref{impl:config:taskcreation} displays a complete task creation in
the Ned configuration script.
\pagebreak

\begin{lstlisting}[language={[5.2]Lua},caption={Task creation with create call
  to the load balancer factory and replacement of the default scheduler in the
  Ned startup script.},label={impl:config:taskcreation}]
L4.default_loader:start(
  {
    caps = {},
    scheduler = tmFactory:create(
      L4.Proto.Scheduler,
      "min_prio = 0", "max_prio = 5", "distr = A")
  },
  "rom/app_matrixmul"
)
\end{lstlisting}

Therefore, it is only necessary to change the the application startup
configuration parameters, which are encapsulated in the Ned startup script,
shown in listing \ref{config:ned_full}.
The load balancing service is completely transparent to the application.

\lstinputlisting[language={[5.2]Lua}, caption={Example of the a benchmark
  startup using the load balancer as scheduler.},
  label={config:ned_full}]{app_restart_mmul_parallel.cfg}
\pagebreak


\paragraph{Configuration Groups.}
The configuration arguments stated in the startup script are parsed by the
factory service and stored in a \texttt{config\_t} type
(listing \ref{config:config_t}).
The parser strips all white spaces from the configuration string and splits it
in two parts: before and after the equality sign.
The front part evaluates to priority or group type, whereas the rear part
evaluates to either an integer or string.
%The front part evaluates to minimum or maximum priority, or group type and the
%rear part evaluates to either an integer or a string identifier for the group.
If the group type is a client-server type, the last character is matched
against \texttt{C} or \texttt{S} to recognize client and server tasks.

Note, this configuration is per task, hence all threads of this task possess
the same configuration.
A configuration group must not necessarily span across several tasks.
The \texttt{distr} parameter, for example, can group all threads of one task
and ensures that all threads of this task will be assigned to the maximal
number of available cores.
\\
\newline

\lstinputlisting[language=c++, caption={Implementation of the configuration
type stored by the task proxy.},
label={config:config_t}]{codeSamples/config.cc}

% vim:set ft=tex:
\section{Internal Data Management}
\label{impl:internal}

%\paragraph{Thread representation \& management}
If a thread calls \texttt{scheduler()->run\_thread(cap, sched\_params)}, passing
a capability and its scheduling parameters, the proxy scheduler
receives the call.
After adding the task's configuration, the proxy forwards the run call to the
central load balancer by invoking \texttt{run\_thread(cap, sched\_params, config)}.
The load balancer then asks the thread management to look up the capability in
its list of known threads.
If the thread is unknown, a new internal thread object of type
\texttt{Thread\_t} is created, encapsulating the capability, the scheduling
parameters, and the configuration.
Besides these three values, \texttt{Thread\_t} stores previous and current
performance values and execution times.

Next, the group configuration is evaluated and the \texttt{Thread\_t} instance
is assigned to all configured groups.

Besides creating and searching, the thread management also validates the
presence of threads and in the case of deletion, it removes the corresponding
\texttt{Thread\_t} instance from all groups, management lists, and deletes empty
groups, before the object is destroyed.


\paragraph{Interval Cycle.}
The interval cycle design introduced in section \ref{design:balancer} is
implemented as follows:
Measurements and prediction are stored in the thread's \texttt{Thread\_t}
instance.
The decision component evaluates the balance of the system and selects threads
to migrate.
The new scheduling parameters are also stored in the thread's \texttt{Thread\_t}
instance.
If the scheduling parameters changed, the enforce component informs the kernel.
\texttt{Thread\_t} is the central structure connecting all these components.

Threads in groups have the benefit to be placed separately, maintaining their
group properties.
Also, when the system is unbalanced, threads without group associations are
considered first for migration.
Isolation groups are not considered here, as their core assignment is static.
Then the threads of each \texttt{distr} group are distributed to all cores, before
\texttt{clsvr} type groups are placed.
The client-server groups are placed on the core with the least \gls{mpc}.
If a task provides several threads to the group, its threads are evenly
distributed to each core.

Afterwards, the load balancer assigns the threads without group affiliation.
First, the cache-heavy threads are distributed to each cache, depending on
\gls{mpc}; they are balanced in space.
Second, the remaining threads balance the system in time.
The load balancer tries to even out all remaining imbalances regarding
computation time between the cores.
The following section describes this in more detail.

% todo normally, after a fwd reference a topic switch

% vim:set ft=tex:
\section{Algorithms}
\label{impl:algos}

% SMT assignment algorithm
% core assignment algorithm
  % in space and in time
% group placement algorithms
% simple load distribution
% missing shared memory

The load balancer uses different algorithms for SMT placement, assignment of
groups and balancing of threads without group affiliation.
I present algorithms for these three places and also a simple behaviour-unaware
load balancing algorithm.
The simple load distribution algorithm serves as baseline to show the benefit of
behaviour analysis.

\paragraph{SMT distribution.}
After the physical core assignment is done, each physical core internally
distributes its set of assigned threads to the hyper-threads.
Different options arise: simple round robin, by \gls{mpc}, by \gls{instpc}, or
by load.

Another possibility consists of one core getting the high \gls{mpc}
threads and the other one the threads with high \gls{instpc}.
I assume this leads to a good symbiosis, as the high \gls{instpc} threads compute,
while the \gls{mpc} heavy threads wait for memory.

Superiority of one of these options above the others must be empirically evaluated.


\paragraph{Two Dimensional Balance.}
What is balance? This question takes the most effort to answer, because it
depends on several factors: isolation groups reduce the core count;
communication groups are excluded from load balancing;
and only threads without further attachments are considered for migration.
So balancing works on top of the placement of communication groups, which is
possibly already imbalanced.
The load balancer must even out these imbalances and be aware of cache usage of
these threads as well.
Also, if only parts of the system are managed by the load balancing service,
the base load needs to be considered during placement.
All this leads to a difficult environment for the load balancer and complicates
the test for the overall system balance.

The core accounting needs to keep track of unaccounted load, accounted but
not migratable load, and the migratable load.
\Gls{mpc} measurements are only accessible for managed threads, meaning not
migratable load and migratable load.
Therefore, \gls{mpc} aware placement and balancing depends on these two
groups.
\\

As mentioned in section \ref{design:load}, balance is two dimensional.
Spatial balance takes priority over temporal balance, to allow better cache usage.
But at the end, temporal balance is equally important.

A simple spatial balance metric is the following: Sort the migratable threads by
\gls{mpc} and assign the top four to one core each.
The number of threads chosen depends on the number of available cores.

To adapt preexisting imbalances regarding \gls{mpc},  the number of chosen
threads can vary; for example if the difference between the core with the
highest \gls{mpc} and the core with the second highest \gls{mpc} is larger than
the \gls{mpc} heaviest thread, assigning another thread to it makes no sense.

Temporal balancing is different, as it has to assign all remaining threads.
Thereby, the absolute core load is relevant, not only the load of the managed
threads.
Sorting the threads by load, and assigning them one by one to the
core with the least load, leads to a balanced system.

When imbalances in time are detected, the temporal balancer computes the difference
and searches for a thread to migrate.
Thereby, the ideal thread generates load that accounts for half the difference.
If no such thread is found, the thread generating the next higher amount of
load is migrated.


\paragraph{MPC-IPC-Placement.}
This algorithm creates two copies of the list of all threads and sorts one copy
by \gls{mpc} and one by \gls{instpc}.
The first item of the \gls{mpc}-list is deleted from the
\gls{instpc}-list.
Then, the algorithm delets the first item of the  \gls{instpc}-list from the
\gls{mpc}-list.
Both lists are processed item by item, until the two lists are disjoint.

As long as both lists contain more threads than cores available, each core
is assigned one thread from each list.
If there are less threads than available cores, the algorithm assigns the
remaining thread to the core with the least \gls{mpc} and \gls{instpc},
respectively.

I assume this works rather well for SMT, as each physical core gets a mixed
workload.
However, this approach does not account for execution time.
A possible improvement is to assign the remaining threads by
exexcution time instead of \gls{instpc} and \gls{mpc}.


\paragraph{First-Last-Placement.}
The list of threads without group affiliation is sorted by \gls{mpc}.
Then, the algorithm assigns the first and the last element of the list to the
same core.
The core is selected based on \gls{mpc}.


\paragraph{Group placements.}
% todo on what basis is first last placement used?
If the group is of type \texttt{distr}, first-last-placement is used for its
members.

\texttt{Clsvr} type groups are handled in two ways: First, it is checked which
group members belong to the same tasks.
Of each task one thread is assigned to the same core as the server thread.
% todo on what basis? why not by load? would make more sense.
The others threads are distributed by first last placement.

\texttt{Sec} and \texttt{rt} type groups get their private physical core, hence
all threads of these groups are assigned to the same core.


\paragraph{Simple Load Distribution.}
The algorithm assigns incoming threads to the core with the least load.
If several cores have the same load, e.g. in an overload scenario, the
algorithm assigns the thread to the core with the least threads.

A core's load is determined by the amount of idle time since the last
placement.
When threads leave, the system becomes unbalanced, but no extra balancing
actions are taken.

To prove that behaviour analysis brings any benefit, it has to beat
this simple algorithm.


\paragraph{Missing Logic for Shared Memory.}
\todo{where is a better place for this?}
Tasks running several threads which use shared memory to communicate will have
a worse performance using these algorithms.
No communication parameter tells the load balancer that these tasks should be
grouped together.
Consider two threads with shared memory on two separated cores.
The only connection is the \gls{llc}.
The cache line holding the shared memory will bounce back and forth between
the two cores, as one writes and invalidates the cache of the other.
This means additional L1 and L2 cache-miss costs, each time they communicate.

If the number of threads is equal to the number of cores, the cache line will
bounce between four different L1 and L2 caches, impairing the overall
performance of the threads.

An additional shared memory configuration parameter is necessary to help the
load balancer diminish this issue.
Then two threads could be placed, so that they run co-scheduled on the same
caches.



\begin{comment}
\paragraph{Pseudo-code of placement algorithm}
  \begin{verbatim}
  from all threads:
    select #core highest LLC miss rate
    select #core highest exec-time
    intersection of both are critical threads
    if threads placed on different cores
      then do nothing
    else
      move higher LLC miss rate thread to other core
    do accounting

  forall threads left do:
    bin by priority levels
    sort each bin by miss rate

  forall prio-bin in prio-bin-list do:
    while threads in prio-bin
      dequeue highest miss rate
      sort cores by lowest accounted miss rate
      place max(#core, #threads left in bin) threads RR on cores;
  \end{verbatim}

  \paragraph{\gls{smt} abstraction code}
  \begin{verbatim}
  if SMT is enabled
    sort threads once by exec time and once by LLC miss
    while duplication:
      look at next LLC-miss thread and dequeue it from exec-time
      look at next exec-miss thread and dequeue it from LLC-miss

    while threads unassigned && queue not empty:
      dequeue one thread from LLC miss list for SMT#0
      dequeue one thread from LLC-miss list for SMT#1
      dequeue one thread from exe-time list for SMT#0
      dequeue one thread form exec-time list for SMT#1
  \end{verbatim}

  \paragraph{Minimize migration pseudo-code}
  \begin{verbatim}
  sort all threads by LLC-miss
  sliding window size #threads with less than 5% LLC miss difference
  if at least 2 threads in the current window are migrated
    if two threads are swaping cores
      don't do the migration
    ALTERNATIVELY
    if the from-core-to-core-matrix has entries on opposing fields
      swap the to-values of both entries
  \end{verbatim}

\end{comment}

\section{Summary}
\label{impl:summary}


\cleardoublepage

%%% Local Variables:
%%% TeX-master: "diplom"
%%% End:
