% vim:set ft=tex:
\section{Internal Data Management}
\label{impl:internal}

%\paragraph{Thread representation \& management}
If a thread calls \texttt{scheduler()->run\_thread(cap, sched\_params)} passing
a capability on itself and its scheduling parameters, the proxy scheduler
receives the call.
After adding the task's configuration, the proxy forwards the run call to the
central load balancer by invoking \texttt{run\_thread(cap, sched\_params, config)}.
The load balancer then ask the thread management to look up the capability in
its list of known threads.
If the thread is unknown, a new internal thread object of type
\texttt{Thread\_t} is created, encapsulating the capability, the scheduling
parameters, and the configuration.
Besides these three values, \texttt{Thread\_t} stores previous and current
performance values and execution times.

Next, the group configuration is evaluated and the \texttt{Thread\_t} instance
is assigned to all configured groups.

Besides creating and searching, the thread management also validates the
presence of threads and in the case of deletion, it removes the corresponding
\texttt{Thread\_t} instance from all groups, management lists, and deletes empty
groups, before the object is destroyed.


\paragraph{Interval Cycle}
The interval cycle design introduced in section \ref{design:balancer} is
implemented as follows:
Measurements and prediction are stored in the threads \texttt{Thread\_t}
instance.
The decision component evaluates the balance of the system and selects threads
to migrate.
The new scheduling parameters are also stored in the threads \texttt{Thread\_t}
instance.
If the scheduling parameters changed, the enforce component informs the kernel.
\texttt{Thread\_t} is the central structure connecting all these components.

Threads in groups have the benefit to be placed separately, maintaining their
group properties.
Also, when the system is unbalanced, threads without group associations are
considered first for migration.
Isolation groups are not considered here, as their core assignment is static.
Then \texttt{distr} groups threads are distributed to all cores, before
\texttt{clsvr} type groups are placed.
The client-server groups are placed on the core with the least \gls{mpc}.
If a task provides several threads to the group, its threads are evenly
distributed to each core.

Afterwards, the load balancer assigns the threads without group affiliation.
First, the cache-heavy threads are distributed to each cache, depending on
\gls{mpc}; they are balanced in space.
Second, the remaining threads balance the system in time.
The load balancer tries to even out all remaining imbalances regarding
computation time between the cores.
The following section describes this in more detail.

% todo normally, after a fwd reference a topic switch
