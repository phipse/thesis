\chapter{Design}
\label{sec:design}

% Ist das zentrale Kapitel der Arbeit. Hier werden das Ziel sowie die
% eigenen Ideen, Wertungen, Entwurfsentscheidungen vorgebracht. Es kann
% sich lohnen, verschiedene Möglichkeiten durchzuspielen und dann
% explizit zu begründen, warum man sich für eine bestimmte entschieden
% hat. Dieses Kapitel sollte - zumindest in Stichworten - schon bei den
% ersten Festlegungen eines Entwurfs skizziert werden.
% Es wird sich aber in einer normal verlaufenden
% Arbeit dauernd etwas daran ändern. Das Kapitel darf nicht zu
% detailliert werden, sonst langweilt sich der Leser. Es ist sehr
% wichtig, das richtige Abstraktionsniveau zu finden. Beim Verfassen
% sollte man auf die Wiederverwendbarkeit des Textes achten.

% Plant man eine Veröffentlichung aus der Arbeit zu machen, können von
% diesem Kapitel Teile genommen werden. Das Kapitel wird in der Regel
% wohl mindestens 8 Seiten haben, mehr als 20 können ein Hinweis darauf
% sein, daß das Abstraktionsniveau verfehlt wurde.

\ldots design \ldots

\todo{write design}

\section{Assumptions and Goals}
\label{design:assump}

The research presented in \ref{state:related} and the constraints due to
the microkernel environment mentioned in \ref{state:env} lead to the following
assumptions:

\begin{description}
  \item[interacting threads]
  \item[Thread groups]
  \item[security core concept] think someone about real time
  \item[thread priority levels]
  \item[priority guarantees] accross core boundaries at placement time?
    options: one prio bin on one core; RR on all cores by prio bin; high prio
    first, place depending on LLC miss rate || exec time accounting?
  \item[]
  \item[partially symmetric multicore processor]
  \item[fixed architecture] Haswell with HT en-/disabled
  \item[Performance] thread placement algorithm using hardware topology and
    thread behaviour information
  \item[Energy efficency] In case of overload equal to performance algorithm,
    otherwise, race to idle, typical: race to idle is best
\end{description}



 % ---------------------------------------------------------------------------

* Table with data sources for prediction and analysis


\subsection{Scratchpad}
\paragraph{Idea:}
One question of the thesis is which information can I get
from the system and where does this information come from.
There are two kinds of information sources, static ones and dynamic ones.

\textbf{Static information sources} are things that can be questioned by the load
balancer and will return mostly the same answer.
E.g. scheduling parameters of a thread (prio, quantium, affinity),
the process\_ID of POSIX threads of the same task is equal,
\ldots{}.

\textbf{Dynamic information sources} are dependent on measurement and behaviour of the
threads and the system.
E.g. hardware performance counters,
usage of time during the last epoch,
\ldots{}.

Uncategorized: shared memory/dataspace sharing, can I even get this info?
IPC: let the threads pass a cap list of frequent comm partners:

\paragraph{Minimal Design}
The load balancing service must maintain its own representation for all threads
in the system and also for the hardware configuration, to be able to place
threads on different cores.
After initialization, where the service should query the kernel scheduler to
set up its data structures, it has to replace the Scheduler capability, to
transparently intervene in the scheduling process.
Hence, the service has to implement the L4\dots{}Scheduler Protocol.
However, newly created threads are started by calling
\begin{verbatim}
L4::Env::env()->Scheduler()->run_thread(
	      L4::Cap<L4::Thread>,
	      l4_sched_param_t);
\end{verbatim}
So the Scheduler-Cap in Env needs to be the capability of the load balancing
service.
To provide a scheduler to a task started by Ned, the start-up parameters
overwrite the default scheduler, which points
to an object providing the scheduler interface: run\_thread, info, idle\_time.



\section{Modules}
\label{design:modules}

\subsection{Hardware performance monitoring}

\subsection{Analyse}
  \paragraph{Hardware}
    * Use dedicated hardware --> hardware analysis future work
    * Haswell: per core: L1 \& L2 cache; shared L3 cache between all cores

  \paragraph{Measurement}
    * Measure the load on each core during a time interval \textbf{TI};


\subsection{Predict}

  * Predict load on each core for the next time intervall;
    * IF thread {A, B, \ldots} is migrated to core {1,2,3,4};


\subsection{Decide}

  * Decide on a thread distribution to cores, based on predicitons;


\subsection{Enforce}

  * Enforce the thread-to-core assignment




% vim:set ft=tex:
\section{Features}
\label{design:features}

* First goal: Distribute several threads to different cores
* L4Re integration
* HyperThread awareness

\paragraph{Issues with security core idea:}
The idea to reduce attack surface for e.g. openssh cache attacks was to provide
a dedicated core for security critical applications.
This idea spawned from the assumption that L1D \& L2 cache was solely present
on a core, but the L3 cache content is a superset of all cores L1D \& L2 cache.
Hence, cache side channel attacks are still possible, although more
complicated, as the attacker not present on the core, must determine, which
cache lines correspond to the L1D \& L2 cache of the ``security core''.
Attacks presented in \cite{yarom_recovering_2014} and
\cite{bernstein_cache-timing_2005} rely on the fact, that only one application
is using cache lines on the core.
On a multicore, only observing the \gls{llc}, other running applications are
expected to disturb the observations, effectivly reducing the side channel
throughput.



\section{Goal Workload}

* CPU bound: While (1)
$* Memory bound: while (1) read_random_sequential_memory ()$


\cleardoublepage

%%% Local Variables:
%%% TeX-master: "diplom"
%%% End:
