\chapter{Design}
\label{sec:design}

% Ist das zentrale Kapitel der Arbeit. Hier werden das Ziel sowie die
% eigenen Ideen, Wertungen, Entwurfsentscheidungen vorgebracht. Es kann
% sich lohnen, verschiedene Möglichkeiten durchzuspielen und dann
% explizit zu begründen, warum man sich für eine bestimmte entschieden
% hat. Dieses Kapitel sollte - zumindest in Stichworten - schon bei den
% ersten Festlegungen eines Entwurfs skizziert werden.
% Es wird sich aber in einer normal verlaufenden
% Arbeit dauernd etwas daran ändern. Das Kapitel darf nicht zu
% detailliert werden, sonst langweilt sich der Leser. Es ist sehr
% wichtig, das richtige Abstraktionsniveau zu finden. Beim Verfassen
% sollte man auf die Wiederverwendbarkeit des Textes achten.

% Plant man eine Veröffentlichung aus der Arbeit zu machen, können von
% diesem Kapitel Teile genommen werden. Das Kapitel wird in der Regel
% wohl mindestens 8 Seiten haben, mehr als 20 können ein Hinweis darauf
% sein, daß das Abstraktionsniveau verfehlt wurde.

\ldots design \ldots

\todo{write design}

\section{Assumptions and Goals}
\label{design:assump}

The research presented in \ref{state:related} and the constraints due to
the microkernel environment mentioned in \ref{state:env} lead to the following
assumptions:

\begin{description}
  \item[Thread groups]
  \item[Performance]
  \item[Energy efficency] In case of overload equal to performance algorithm
\end{description}



 % ---------------------------------------------------------------------------

* Table with data sources for prediction and analysis


* Design a sound model for the predict and design modules first

\section{Modules}
\label{design:modules}

\subsection{Hardware performance monitoring}

* Thread description as central data storage

% ----------------------------------------------------------------------------

\subsection{Analyse}
  \paragraph{Hardware}
    * Use dedicated hardware --> hardware analysis future work \\
    * Haswell: per core: L1 \& L2 cache; shared L3 cache between all cores

  \paragraph{Measurement}
    * Measure the load on each core during a time interval \textbf{TI}; \\
    * measure LLC misses during last \textbf{TI};


% ----------------------------------------------------------------------------

\subsection{Predict}

  * Predict load and LLC misses of each thread for the next time intervall; \\


% ----------------------------------------------------------------------------

\subsection{Decide}

  * Decide on a thread distribution to cores, based on predictions;

  \subsubsection{Placement Generator}

  \paragraph{Pseudo-code of placement algorithm}
  \begin{verbatim}
  from all threads:
    select #core highest LLC miss rate
    select #core highest exec-time
    intersection of both are critical threads
    if threads placed on different cores
      then do nothing
    else
      move higher LLC miss rate thread to other core
    do accounting

  forall threads left do:
    bin by priority levels
    sort each bin by miss rate

  forall prio-bin in prio-bin-list do:
    while threads in prio-bin
      dequeue highest miss rate
      sort cores by lowest accounted miss rate
      place max(#core, #threads left in bin) threads RR on cores;
  \end{verbatim}

  \paragraph{\gls{smt} abstraction code}
  \begin{verbatim}
  if SMT is enabled
    sort threads once by exec time and once by LLC miss
    while duplication:
      look at next LLC-miss thread and dequeue it from exec-time
      look at next exec-miss thread and dequeue it from LLC-miss

    while threads unassigned && queue not empty:
      dequeue one thread from LLC miss list for SMT#0
      dequeue one thread from LLC-miss list for SMT#1
      dequeue one thread from exe-time list for SMT#0
      dequeue one thread form exec-time list for SMT#1
  \end{verbatim}

  \paragraph{Minimize migration pseudo-code}
  \begin{verbatim}
  sort all threads by LLC-miss
  sliding window size #threads with less than 5% LLC miss difference
  if at least 2 threads in the current window are migrated
    if two threads are swaping cores
      don't do the migration
    ALTERNATIVELY
    if the from-core-to-core-matrix has entries on opposing fields
      swap the to-values of both entries
  \end{verbatim}


% ----------------------------------------------------------------------------

\subsection{Enforce}

  * Enforce the thread-to-core assignment




% vim:set ft=tex:
\section{Features}
\label{design:features}

* First goal: Distribute several threads to different cores
* L4Re integration
* HyperThread awareness

\paragraph{Issues with security core idea:}
The idea to reduce attack surface for e.g. openssh cache attacks was to provide
a dedicated core for security critical applications.
This idea spawned from the assumption that L1D \& L2 cache was solely present
on a core, but the L3 cache content is a superset of all cores L1D \& L2 cache.
Hence, cache side channel attacks are still possible, although more
complicated, as the attacker not present on the core, must determine, which
cache lines correspond to the L1D \& L2 cache of the ``security core''.
Attacks presented in \cite{yarom_recovering_2014} and
\cite{bernstein_cache-timing_2005} rely on the fact, that only one application
is using cache lines on the core.
On a multicore, only observing the \gls{llc}, other running applications are
expected to disturb the observations, effectivly reducing the side channel
throughput.



\section{Goal Workload}

* CPU bound: While (1)
$* Memory bound: while (1) read_random_sequential_memory ()$


\cleardoublepage

%%% Local Variables:
%%% TeX-master: "diplom"
%%% End:
