\chapter{Design}
\label{sec:design}

% Ist das zentrale Kapitel der Arbeit. Hier werden das Ziel sowie die
% eigenen Ideen, Wertungen, Entwurfsentscheidungen vorgebracht. Es kann
% sich lohnen, verschiedene Möglichkeiten durchzuspielen und dann
% explizit zu begründen, warum man sich für eine bestimmte entschieden
% hat. Dieses Kapitel sollte - zumindest in Stichworten - schon bei den
% ersten Festlegungen eines Entwurfs skizziert werden.
% Es wird sich aber in einer normal verlaufenden
% Arbeit dauernd etwas daran ändern. Das Kapitel darf nicht zu
% detailliert werden, sonst langweilt sich der Leser. Es ist sehr
% wichtig, das richtige Abstraktionsniveau zu finden. Beim Verfassen
% sollte man auf die Wiederverwendbarkeit des Textes achten.

% Plant man eine Veröffentlichung aus der Arbeit zu machen, können von
% diesem Kapitel Teile genommen werden. Das Kapitel wird in der Regel
% wohl mindestens 8 Seiten haben, mehr als 20 können ein Hinweis darauf
% sein, daß das Abstraktionsniveau verfehlt wurde.

\todo{write design}

This chapter describes the problems that need to be solved, to achieve the
overall goal of a behaviour aware load balancer on the Fiasco.OC microkernel.
From chapter \ref{sec:state} four main areas can be deduced: \gls{smt},
cogestion awareness, communication, and isolation.
\todo{congestion awareness -> behaviour observation?}
Additionally, the scheduler proxy and distribution algorithms must be designed.
Each of these areas brings with it its own set of problems, which will be
discussed in the following sections.

% vim:set ft=tex
\section{SMT/\gls{ht}}
\label{design:smt}

\newacronym{eu}{EU}{execution units}

The performance gains of \gls{ht} stems from the conversion of thread level
parallelism to instruction level parallelism, by providing more instructions to
the \gls{eu} to fill their pipelines.
But if two threads issue the same group of instructions, no performance gain is
possible, as the number of \gls{eu} has not increased.
Hence, two threads with good \gls{ht} performance use different \gls{eu}s.

The term co-schedule, coined in \cite{snavely_symbiotic_2000}, describes two
threads running at the same time on two corresponding logical cores.
The two big questions for this work are: how can the performance of a co-schedule be
measured? And: is such a measurement comparable to previous ones?

From a hardware perspective the question is: How high is the \gls{eu}
utilization of a co-schedule?
A performance counter observing all executed
micro-ops\footnote{UOPS\_RETIRED.ALL} could answer this question.
But it is still questionable, if this measure reflects the performance
degradation each thread suffers due to sharing execution resources.
To get the degradation percentage, knowledge about solo execution is necessary,
which is either gained offline beforehand, or by scheduling each thread alone
for some time and extrapolating the measurement to the thread's lifetime.
The former is excluded due to the scope of this thesis, the latter harms
overall system performance and is imprecise, if the thread workload changes
heavily.

Besides fine granular performance measure and relying on estimations, a third
problem with rating \gls{ht} performance is the length of a balancing interval
and the possibility of running more than two threads during this interval on a
pair of logical cores.
If more than two threads are running during an interval, the load balancer can
only rate the group performance, because it is impossible to deduce which
of the possible co-schedules ran for which amount of time.
Consequently, the group rating describes the performance of a physical core.

The same applies to performance comparison between hyper-threads. As the
performance of a single thread is influenced by its co-runner, it cannot be
compared to logical cores of different physical cores.

All this necessitates an abstraction from logical cores.
This abstraction allows for the comparison of physical core performance and
lets the load balancer to act independently of the presence of \gls{ht}.

% vim:set ft=tex
\section{Isolation}
\label{design:isolation}

A microkernel provides strong isolation and also real-time guarantees.
A requirement aware thread to core mapping, can support these guarantees.
The following discusses security and real-time needs of applications and how to
manage these.

\paragraph{Security}
As presented in section \ref{state:related} even in virtualized environments
CPU cache is a valid side channel to attack secret keys.
To reduce the channel's throughput, security critical applications should share
as little cache as possible with unknown applications.
In Haswell's inclusive cache hierarchy, the \gls{llc} is shared among all
cores, but the L1 and L2 caches are core local.
However, their cache lines are included in the shared \gls{llc}.

As shown in \cite{inci_seriously_2015} and \cite{liu_last-level_2015}
the impact of a ``security core'' design is limited.
Still, a security core increases the difficulty for PRIME+PROBE attacks.
\\

The security requirement of a task is defined by the system design.
Therefore, a way to provide this static information to the load balancing
service is necessary.
It is also preferable to define specific security groups to separate different
security critical tasks or to place a group of related security tasks.
Hence, the configuration scheme must allow to define groups and identify the
same group in the configurations of different tasks.

This can be achieved by a per task configuration parameter identifying
different groups by name.


\paragraph{Real-Time}
Very similar isolation guarantee is necessary for real-time tasks.
The main difference is the possibility to assign more than one core to a
real-time task group.
If this is necessary depends on the group size and the tasks' requirements.
\todo{reason for multicore support}

The configuration parameter parameter must identify different groups by name
and their physical core requirements.




\paragraph{Configuration}
% move to implementation?
The configuration parameters for security and real-time tasks are the
following:
\begin{lstlisting}
  SEC = <Name>
  RT = <Name>
\end{lstlisting}
Multicore support for real-time tasks is not provided at the current state of
the implementation.
Also, if the number of RT and SEC groups is higher than the number of physical
cores, or if core 0 is used the system will output an error message.

% vim:set ft=tex
\section{Behaviour Observation}
\label{design:behaviour}

The key element of this project is behaviour observation.
The threads behaviour is displayed through its resource usage.
Cache misses and instructions show how memory and computation intense a thread
is.

To acquire these information, the kernel needs to run performance counter on
each core and provide per thread accounting.
Via a system call interface the account values can be requested and reseted for
each thread.

As shown in \cite{zhuravlev_addressing_2010} and \cite{knauerhase_using_2008}
\gls{llc} misses are a good indicate for the memory usage of a thread.
A high \gls{llc} miss value indicates the thread is using a lot of memory and
due to the misses it waits often for the memory request to be served.
During this waiting phases, the thread cannot proceed, hence the executed
instruction count is low.

So a memory intensive thread differs in \gls{llc} misses and instructions per
cycle from a computation intense thread.


\todo{what else to write here?}


% vim:set ft=tex
\section{Communication}
\label{design:comm}

A less attended behaviour property of threads is communication.
It is difficult to observe which threads are communicating with each other and
how often.
This section explores kernel and user land approaches to acquire this
information static and dynamically.


\paragraph{Communication types}
Two types of communication exist.
\citeauthor{hofmeyr_load_2010} assumed a program running several threads which
compute a chunk of work and then communicate their results to each other.
This cycle repeats until all work is executed, where the time executed is the
major part of the cycle and only a small part is used for communication.
Nevertheless, the overall program performance is best, if each thread reaches
the communication synchronisation at the same time to spend as little time as
possible waiting for others.
This connection was also note in \cite{hofmeyr_load_2010}.
The compute-communicate pattern is typical for distributed systems and is named
distributed communication in the following.

The other type is client-server communication.
To achieve the best possible performance, the communication latency, meaning
the time difference between sending and receiving the message, must be minimal.
The latency is minimal, if both communication partners are on the same core.
Additionally, time slice donation, where the server executes on the time slice
of the client, is only possible if both execute on the same core.
Furthermore, the server will not execute unless it has a client request, so a
server typically has long idle times.
Therefore, it is preferable to execute both communication partners on the same
core.

But this conclusion is wrong, if there are several clients of the same task.
If the task runs several threads, then it most likely wants these threads to
execute in parallel.
Running all threads close to the server on the same core, would heavily impact
their performance.
In the worst case, the performance drops below single threaded performance, due
to context switch overhead.

The same argumentation applies, if the server has a high number of clients from
different tasks.
To execute all clients on the server's core, would be a waste of resources.
Hence, the distribution of the tasks threads to different cores takes priority
over low communication latencies.


\subsection{Kernel}
Naturally, acquiring information at kernel level is easier, as more information
context is available.
But leakage of kernel information to user land must be pondered carefully.
Hence, the following approaches are marked as debug feature.


\paragraph{Communication Matrix}
Of interest are communication partners and the rate of communication during a
time frame.
It is important to reduce the latency of intense connections.
To model the communication relationships, each thread is assigned a global
identifier and a matrix is created, where each field contains a counter,
counting each communication event, between the two corresponding global IDs.
This matrix is symmetric, hence to save memory, only the upper half is used
and communication between two threads uses the counter in field
(lowerID, higherID).
The upper bound for the memory requirement is $1/2 * (\#threads)^2$.
As not all threads will communicate with each other, usage of a sparse matrix
will further lower the memory requirement.
A sparse matrix allocates only memory for fields, which are actually used.

\paragraph{Map}
Another approach can be a thread local map, which contains the global
identifier of the communication partner and a counter.
Each communication attempt increases the counter by one.
Similar to the sparse matrix, this approach only allocates memory for used
counters.
Also, the used memory is directly accounted to the threads which needs it.
However, as this is a thread local solution, both communication partners will
maintain their separate counters.
\\

The user land load balancer can use the system call interface to query
and reset these counters.
The answer can contain an array of all counters for this thread, to reduce the
system call overhead.
In case of the matrix, the answer must be created from the thread ID column and
the thread ID row.


\subsection{User land}
At user level, information gathering is much more complicated.
The strong isolation guarantees make it infeasible to work with global
identifiers across all threads.
But comparable capabilities already provide an identifier.

\paragraph{Cooperative communication capabilities}
To communicate with a partner, a thread needs a capability in its own
capability table, pointing to the receiver.
Together with a counter value, this capability can be passed to the load
balancer and the load balancer can then identify communication partners
by comparing this received capability, with the capabilities of the threads
it balances.
The kernel provides a system call which tells, if two passed capabilities point
to the same object.
The load balance can then build a relationship graph between threads and adapt
its assignment.

This approach requires changes to the scheduler interface in L4Re and
cooperating threads.
But it has the benefit, that each thread can decide on its own, if the
communication relationship is intense enough to benefit from increased
locality.
Unfortunately, this requires changes to each thread managed by the balancer and
regular intensity updates, which itself adds communication overhead.
Together with the interface changes, this dynamic approach is deemed infeasible
for this project.

\paragraph{Configuration parameter}
Less invasive, but static, is the usage of task specific configuration parameters.
The two communication types yield two configuration parameters to represent the
different behaviours.
Together with an identifier thread groups can be configured.
This extends the set of isolation parameters for the thread configuration
introduced in \ref{design:isolation}.

This approach requires careful system design, but no change of program source
code or thread cooperation.

\paragraph{Implementation}
The two communication types are named ``DISTR'' and ``CLSVR''.
A group is identified via ``<name>'' and either distributed or clustered.
To identify the server in the client-server relationship, the server task's
group name is assigned the ``\_S'' postfix.
Client tasks use the ``\_C'' postfix.

\begin{lstlisting}
  distr = <Name>
  clsvr = <Name>_{C|S}
\end{lstlisting}

These parameters are optional and are used in the startup script interpreted by
Ned.

% vim:set ft=tex:
\section{Energy Model}
\label{design:energy}

Energy consumption is important in modern systems, as battery powered devices
become more popular.
Measuring energy consumption is not part of this thesis.
However, I will derive a model for energy consumption on x86 from work presented
in section \ref{state:related} and design an energy efficient load balancing
strategy, based on this model.
\\

A critical question for an energy model is:
When is the least amount of energy consumed?
In a race-to-idle scenario, where work is done fast and the cores sleep for a
longer time, or if the work is done at lower frequencies for a longer time.
In \cite{le_sueur_slow_2011} \citeauthor{le_sueur_slow_2011} assume that
quality of service is maintained, even at low frequencies.
They show for their three applications that the
energy efficiency is best for low frequencies, but depending on the
application, at the cost of throughput.
So it is highly application-specific to find the right frequency for the
running application.
The complexity increases, if more than one application is present in the
system.
Furthermore, a load balancer cannot decide on core frequencies, because its
view on the system is not necessarily complete.

\citeauthor{le_sueur_slow_2011} note that turbo boost should be avoided at
all cost, as the performance gain is little compared to the increased energy
consumption.
The hardware does not use turbo boost as long as all cores are loaded, as the
energy budget does not allow it.
Also, it is possible to disable the feature completely.


Another relevant observation in \cite{le_sueur_slow_2011} is, that the usage
of deep sleep states and the related restart costs only slightly increase the
system load.
If a core goes into a deep sleep state, the system load is low anyway, so the
additional wakeup-load poses no problem.
\\

\cite{imes_poet_2015} states that different models are necessary for different
platforms.
A mobile Haswell generation processor consumes the least amount of energy if
the race-to-idle scheme is used, whereas an embedded Exynos5 generation CPU is
designed for constant low frequency operation.
The target hardware for this work is a Haswell generation processor, so the
race to idle model seems promising.
\\

The difference in energy consumption between low frequency and normal
frequency, however, is little compared to a deep sleep state, due to the
already high base energy consumption.
So following \cite{imes_poet_2015}, it is best to increase sleep time and use
the normal frequency to execute the arising work quickly.

The model parameters for best energy efficiency are:
No turbo boost, either execute at normal frequency or sleep, and use as many
cores as possible, to reduce cache misses and consequently execution time.

A reduction in runtime leads to larger sleep times or more throughput, which is
the goal of a load balancer all along.
Therefore, an algorithm maximizing throughput also increases energy efficiency
on the x86 architecture.


% todo Can POET be used in conjunction with a load balancer? Do they benefit each
% other?

% vim:set ft=tex:
\section{Load Balancer}
\label{design:balancer}

All the techniques introduced in the previous sections of this chapter
are now united into the central load balancing component.
The section describes the components the load balancer is build of.
% todo chapter structure


\paragraph{Load Balancer \& Task Proxies}
To achieve good load balancing performance, the load balancer needs information
about the underlying hardware, as well as the amount of work in the system.
Thereby, it is also possible to build hierarchical load balancing, where
only a partial view on the hardware is available and only a subset of tasks run
on this specific load balancer.
However, while leading to isolation between different subhierarchies,
the overall performance can suffer, as the system is effectively partitioned.
A surplus of computing resources in one partition is unusable by another
partition leading to wasted resources.
The partition size is only adaptable at core granularity, so meeting the
precise resource needs may not be possible, leading to overprovisioning and
consequently, resource waste.

Isolation between parts of the system can also be provided by a central load
balancing component.
A central load balancer as well as a hierarchical approach is part of the chain
of trust of each balanced application, so both approaches are equally
trustworthy.
But the central component can react dynamically to changes in the workloads and
shift the partition borders dynamically.
Also, if tasks with security properties, as discussed in section
\ref{design:isolation}, enter or leave the system, the dynamic approach can
repurpose a core.
Redistribution of cores between hierarchical load balancer would require
additional reasoning, which load balancer the free core is assigned to, or which
load balancer must release a core.
Either the load balancer must communicate their reasoning between each other or
with a central instance further up the hierarchy.
The former weakens the isolation and increases overhead, the latter means more
logic in a central instance.

Due to the easier implementation of a central load balancing component,
I choose to implement the centralized service.
\\

Next to the load balancer, task specific proxy scheduler acts as information
source.
Each task can have isolation (\ref{design:isolation}) and communication
properties (\ref{design:comm}).
The task proxy stores this task specific configuration and provides it to the
load balancer.

The factory creating the load balancer and the task proxies, processes the
configuration parameters and instantiates the configuration structure exchanged
between proxies and balancer.
The configuration structure comprises the priority band and a list of
groups derived from the parameters.


\paragraph{Thread Management}
All threads known to the load balancer are managed by the thread management.
It creates a internal administration class for each thread and identifies a
thread via its capability.
The load balancer evaluates the configuration passed by the task proxy and
if isolation or communication groups are set, the thread is assigned to these
groups.
The management provides a consistent view on all managed, grouped, and not
grouped threads for other components and is also in charge of cleanup.


\paragraph{CPU Topology \& Accounting}
Besides the management of threads, the load balancer must also know the
topology of the underlying hardware.
The topology scanner analyses topology via CPUID and stores it in a tree like
data structure.
A package describes all physical cores of one socket.
If \gls{ht} is present, each physical core contains data structures for
two logical cores.

Each physical core is associated with an accounting object.
Core accounting stores all assigned threads and computes the accumulated
load on its assigned core.
Additionally, it provides information on which thread to migrate in case of an
imbalance.
The accounting object connects topology, physical cores, and threads, therefore,
it is the place for the SMT abstraction introduced in \ref{design:smt}.
Each thread assigned to the physical core, is distributed to its hyper-threads
by an algorithm, e.g. round robin, \gls{llc} weight, IPC, or load.
If no \gls{ht} is available, no further assignment is necessary.
The availability of \gls{ht} is deduced from the topology enumeration results.


\paragraph{Information Gathering \& Balancing}
% each interval measure predict decide enforce cycle
% measure: for each thread, gather performance counter values and execution
  % time from kernel and store in thread administration class.
% prediction: compute a performance forecast for the next interval
% decide: select a thread to core assignment based on the prediction for the
  % next interval.
% enforce: configure the core affinity for each thread and inform the kernel
The logic of the load balancer is partitioned into four consecutive modules:
measurement, prediction, decision, and enforcement.
Each balancing interval, they run once and adapt the thread to core
assignment to the current circumstances.

First, the information about the current situation is updated by gathering
performance counter and execution time values of the last interval for each
thread.
Based on these measurements, the prediction module generates a forecast for
each thread's resource needs.
This forecast forms the input for the decision module.
Together with the topology information and the accounting values for each core
from last interval, the decision module checks, if the system is still balanced.
If not, selected threads must be migrated to establish balance again.
After deciding which threads must be migrated, the enforce module applies the
migrations by modifying each thread's affinity and informing the kernel.

These steps repeat every interval. A high interval duration reduces the
load balancing overhead, but also leaves the system longer in a bad state, when
thread behaviour changes or threads enter and leave the system.
A short interval duration increases the measurement overhead and may increase
the amount of migrations.
The ideal interval size increases system performance to outweigh the interval's
own runtime overhead.
\\

The benefit of dividing the interval cycle into four separate components lies
in its flexibility.
If other information sources should be used, the measurement component is the
place to add them.
If different performance prediction algorithms should be tested, only the
prediction component must be changed.
The decision component allows to easily test balancing and placement algorithms
and also incorporating multi socket systems is neatly encapsulated in it.

Of course if information sources arise, changes down the line are necessary.
But this approach provides a concise description of the different components.
Also extending the load balancer to heterogeneous architectures is assumed to
work well, as the design is similar to the approach in
\cite{sarma_smartbalance_2015}.


\begin{comment}

\paragraph{Topology description}
Represents the cache and core layout of the CPU.
On Haswell it consist of a package description containing core descriptions.
A core description consists of the hardware assigned APIC\_id, the kernel
assigned fiasco\_id, and the SMT abstraction.
It can be expanded to include multi-socket systems by adding more package
descriptions.
It maps the fiasco view on the cores onto the actual HW topology and uses CPUID
to determine corresponding logical cores.

\paragraph{Thread\_t}
is the administrative representation of a L4-thread.
It contains not only of the thread parameters passed via run\_thread(), but
also measurements for the current and last interval.
LLC-misses, execution time, an identifier, and the cores it currently runs on
and will run on in the next interval.

\end{comment}

% vim:set ft=tex:
\section{Summary}
\label{design:summary}

During this chapter I discuss the difficulties with determining the
performance of corresponding hyper-thread cores without offline knowledge about
the running threads.
The \gls{ht} abstraction achieves two things: the load balancer compares the
performance of physically equal units of hardware, and it becomes independent
of the actual presence of \gls{ht}.

The load balancing service provides isolation for application with real-time
or security needs.
Also, I present different approaches to gain knowledge about communication
relationships within a task or between a group of tasks.

The discussion about an energy model leads to the conclusion, that on x86
processors a race-to-idle scheme without usage of turbo-boost features is the
most promising model.
A reduction in computation time leads to larger idle periods for the different
cores, which in turn leads to more time spent in deep sleep states, where
energy usage is minimal.

Subsequently, I present my design of the central load balancing service and its
modules.
Measurement and prediction of thread behaviour provides the knowledge to decide
on balancing options and to enforce this decision.

The chapter finishes with a discussion about different definitions for load
and depending on the character of the load, which system state is considered to
be in balance.
\\

The following chapter presents changes to the Fiasco.OC kernel to provide
access to hardware performance counters; how a system designer informs the load
balancer about isolation and communication groups; and discusses different
algorithms to decide on thread placement.


%\section{Assumptions and Goals}
\label{design:assump}

The research presented in \ref{state:related} and the constraints due to
the microkernel environment mentioned in \ref{state:env} lead to the following
assumptions:

\begin{description}
  \item[interacting threads]
  \item[Thread groups]
  \item[security core concept] think someone about real time
  \item[thread priority levels]
  \item[priority guarantees] accross core boundaries at placement time?
    options: one prio bin on one core; RR on all cores by prio bin; high prio
    first, place depending on LLC miss rate || exec time accounting?
  \item[]
  \item[partially symmetric multicore processor]
  \item[fixed architecture] Haswell with HT en-/disabled
  \item[Performance] thread placement algorithm using hardware topology and
    thread behaviour information
  \item[Energy efficency] In case of overload equal to performance algorithm,
    otherwise, race to idle, typical: race to idle is best
\end{description}



%\todo{move assumptions to end of chapter}
 % ---------------------------------------------------------------------------
\begin {comment}
\begin{itemize}
  \item SMT
  \item Isolation
  \item Behaviour observation
  \item Communication
  \item ThreadMapper aka Scheduler Proxy \& TaskProxies
  \item Algorithms
\end{itemize}

* Table with data sources for prediction and analysis


\subsection{Scratchpad}
\paragraph{Idea:}
One question of the thesis is which information can I get
from the system and where does this information come from.
There are two kinds of information sources, static ones and dynamic ones.

\textbf{Static information sources} are things that can be questioned by the load
balancer and will return mostly the same answer.
E.g. scheduling parameters of a thread (prio, quantium, affinity),
the process\_ID of POSIX threads of the same task is equal,
\ldots{}.

\textbf{Dynamic information sources} are dependent on measurement and behaviour of the
threads and the system.
E.g. hardware performance counters,
usage of time during the last epoch,
\ldots{}.

Uncategorized: shared memory/dataspace sharing, can I even get this info?
IPC: let the threads pass a cap list of frequent comm partners:

\paragraph{Minimal Design}
The load balancing service must maintain its own representation for all threads
in the system and also for the hardware configuration, to be able to place
threads on different cores.
After initialization, where the service should query the kernel scheduler to
set up its data structures, it has to replace the Scheduler capability, to
transparently intervene in the scheduling process.
Hence, the service has to implement the L4\dots{}Scheduler Protocol.
However, newly created threads are started by calling
\begin{verbatim}
L4::Env::env()->Scheduler()->run_thread(
	      L4::Cap<L4::Thread>,
	      l4_sched_param_t);
\end{verbatim}
So the Scheduler-Cap in Env needs to be the capability of the load balancing
service.
To provide a scheduler to a task started by Ned, the start-up parameters
overwrite the default scheduler, which points
to an object providing the scheduler interface: run\_thread, info, idle\_time.






\section{Goal Workload}

* CPU bound: While (1)
$* Memory bound: while (1) read_random_sequential_memory ()$

\end {comment}

\cleardoublepage

%%% Local Variables:
%%% TeX-master: "diplom"
%%% End:
