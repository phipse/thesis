\chapter{Design}
\label{sec:design}

% Ist das zentrale Kapitel der Arbeit. Hier werden das Ziel sowie die
% eigenen Ideen, Wertungen, Entwurfsentscheidungen vorgebracht. Es kann
% sich lohnen, verschiedene Möglichkeiten durchzuspielen und dann
% explizit zu begründen, warum man sich für eine bestimmte entschieden
% hat. Dieses Kapitel sollte - zumindest in Stichworten - schon bei den
% ersten Festlegungen eines Entwurfs skizziert werden.
% Es wird sich aber in einer normal verlaufenden
% Arbeit dauernd etwas daran ändern. Das Kapitel darf nicht zu
% detailliert werden, sonst langweilt sich der Leser. Es ist sehr
% wichtig, das richtige Abstraktionsniveau zu finden. Beim Verfassen
% sollte man auf die Wiederverwendbarkeit des Textes achten.

% Plant man eine Veröffentlichung aus der Arbeit zu machen, können von
% diesem Kapitel Teile genommen werden. Das Kapitel wird in der Regel
% wohl mindestens 8 Seiten haben, mehr als 20 können ein Hinweis darauf
% sein, daß das Abstraktionsniveau verfehlt wurde.

\ldots design \ldots

\todo{write design}

* Table with data sources for prediction and analysis


* Design a sound model for the predict and design modules first

\section{Modules}
\label{design:modules}

\subsection{Hardware performance monitoring}

\subsection{Analyse}
  \paragraph{Hardware}
    * Use dedicated hardware --> hardware analysis future work
    * Haswell: per core: L1 \& L2 cache; shared L3 cache between all cores

  \paragraph{Measurement}
    * Measure the load on each core during a time interval \textbf{TI};


\subsection{Predict}

  * Predict load on each core for the next time intervall;
    * IF thread {A, B, \ldots} is migrated to core {1,2,3,4};


\subsection{Decide}

  * Decide on a thread distribution to cores, based on predicitons;


\subsection{Enforce}

  * Enforce the thread-to-core assignment




\section{Features}
\label{design:features}

* First goal: Distribute several threads to different cores
* L4Re integration
* HyperThread awareness



\section{Goal Workload}

* CPU bound: While (1)
$* Memory bound: while (1) read_random_sequential_memory ()$


\cleardoublepage

%%% Local Variables:
%%% TeX-master: "diplom"
%%% End:
