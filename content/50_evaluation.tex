% vim:set ft=tex:
\chapter{Evaluation}
\label{sec:evaluation}

% Zu jeder Arbeit in unserem Bereich gehört eine Leistungsbewertung. Aus
% diesem Kapitel sollte hervorgehen, welche Methoden angewandt worden,
% die Leistungsfähigkeit zu bewerten und welche Ergabnisse dabei erzielt
% wurden. Wichtig ist es, dem Leser nicht nur ein paar Zahlen
% hinzustellen, sondern auch eine Diskussion der Ergebnisse
% vorzunehmen. Es wird empfohlen zunächst die eigenen Erwartungen
% bezüglich der Ergebnisse zu erläutern und anschließend eventuell
% festgestellte Abweichungen zu erklären.

Numerous questions regarding different aspects of load balancing arise
throughout this thesis.
What is the impact on performance of good balancing at \gls{smt}-level?
How does the prototype perform under little, normal, and heavy load compared to
Linux's CFS?
Load balancing is not about average performance, it is also about consistent
run-time behaviour of the whole system throughout several repetitions. How do
the balancing algorithms perform in this regard?
How beneficial is the knowledge of communication relationships?
How does a preferential treatment of those impact the performance of other
applications in the system?

These are the questions I delve into in this chapter.
However, due to time constraints, I am not able to measure, analyse, and answer
all of them in detail.

% vim:set ft=tex:
\section{Benchmark and Measurement Setup}

In the following, I describe the benchmark applications and the
system environment, the benchmarks run in.

\paragraph{Benchmarks.}
I use four applications from the \emph{SPEC CPU2006} benchmarking
suit\footnote{\url{http://spec.org/cpu2006/}}:
namely, 403.gcc, 416.gamess, 429.mcf, and 470.lbm.
They will display the algorithms performance under load.

416.gamess is a floating point benchmark written in Fortran executing a quantum
chemistry simulation.
It is a compute-bound benchmark and generates close to zero \gls{mpc}.
The 403.gcc integer benchmark compiles a fixed amount of source code.
It has nine different input workloads, from which I selected the 166.i
workload.
The benchmark is also compute-bound and generates a low number of \gls{mpc}.
470.lbm simulates fluid dynamics and stresses the floating point units.
It generates a high amount of \gls{mpc}, therefore it is memory-bound.
429.mcf generates the highest number of \gls{mpc}.
It computes a combinatorial optimization using the integer units.
The benchmark set consists of two compute-bound and two memory-bound
benchmarks, with one integer and one floating point benchmark each.

Besides the SPEC benchmarks, I employ two other applications to proof the
group configuration awareness of the load balancer one the one side, and the
performance benefits for such a configuration on the other side.
To that end, I use a multi-threaded matrix multiplication for a matrix with
one thousand entries and a client-server application printing the number of the
core it currently runs on.

The latter benchmark shows that client and server are migrated together, as
long as there is only one client per task.
If a task spawn several threads, the balancer distributes the additional
threads to other cores.


\paragraph{Algorithms.}
I will measure the performance of four different algorithms.
The simple load distribution algorithms introduced in \ref{impl:algos},
distributes each new thread to the core with the largest amount of idle time.
It does no further migrations.

The space-time-balance algorithm, also introduced in \ref{impl:algos},
observes temporal and spatial thread behaviour.
It distributes the threads with the highest memory usage to different caches
and balances the rest depending on their execution time.

The third algorithm is a small adaption of the space-time-balance algorithm.
It just omits the spatial balancing part and only performs temporal balancing.

The fourth and last algorithm different from the above, as it runs in
conjunction with the scheduler at kernel level; it is Linux's \gls{cfs}.


\paragraph{Time Measurement.}
For each application I measure the time it takes to start the application,
execute it until completion, and exit it again.
I repeat the measurement one hundred times, or if several benchmarks run
concurrently, until the slowest benchmark has repeated one hundred times.
On L4Re, I use the Ned startup script to measure and output the times for
each application.
On Linux, I use a ruby script to measure the durations and to store them in a
file for each application.

Furthermore, I do not use the time stamp counter (TSC), as this counter depends
on the core the time stamp is taken on.
Instead I use \texttt{clock\_gettime(CLOCK\_MONOTONIC)} on Linux and
\texttt{l4\_kip\_clock(l4re\_kip())} on L4Re.
 

\paragraph{64bit Issue.}
Running the benchmarks on 64bit, raises an issue in the memory subsystem of L4Re.
Memory allocation and deallocation increases the runtime of 403.gcc by several
seconds for each repetition, when it runs alone in the system.
It is irrelevant if a simple load distribution, space-time-balancing,
or no load balancing runs the benchmark.
The task is run to completion and then restarted, so I expect each allocated
page to be free after completion.
The L4Re internal memory management should now merge free chunks of memory.
However, the size of the list of free memory areas increases constantly and
shows allocated pages in between, preventing compaction.
I suspect the time spent to search for possible merges and to iterate through
the list to be the cause for the increase in runtime.

This issue does not affect the runtime of 416.games, but the other memory-bound
benchmarks are affected as well.
429.mcf shows slight but constant increase with each run.
429.mcf allocates all memory at the beginning of the benchmark, so the memory
subsystem is not used constantly, as is the case for 403.gcc.

When 416.gamess, 429.mcf, and 470.lbm run together on L4Re the runtimes
increase occasionally between 2.5 and 4 times.
This increase cannot solely be explained by bad load balancing decisions.
Also, the average runtime for the first 50 iterations of 470.lbm is 60 seconds
lower than the average runtime of the second 50 iterations.
Minimum, maximum and median runtime show a similar difference.

Providing a different memory allocator to each task created via the ned script,
did not eliminate this problem.

As this issue did not present itself for 32bit, all benchmarks are executed
in a 32bit environment.


\begin{comment}
\paragraph{Benchmarks.}
SPEC progs:
solo runs: SLD, STB, MIPC, CFS
group runs: SLD, STB, MIPC, CFS
--> degradation of median compared  to solo runs

Group runs for different SMT algos: RR, load,
Group runs for different MIPC assignments: load, mpc-ipc

pingpong  clsvr group;
openmp-mmul distribution group; -- load generated by other mmul or fractal



\paragraph{Group Configuration Benchmarks}
The first is a simple \gls{ipc} ping-pong, where the server answers a call from
the client with the number of the core it is currently running on.
The client measures the time for the \gls{ipc} call and prints the answer of
the server together with its own core number and the \gls{ipc} duration.

\end{comment}

% vim:set ft=tex:
\section{Measurements}
\label{eval:measurements}

I measure three different constellations of the SPEC applications: solo, load,
and heavy load.
The solo execution time of each application serves as baseline to compare the
performance of different algorithms in a low load situation, and to enable the
computation of the slowdown when run in parallel with other applications.
The slowdown of the execution time is computed for the best and worst
case and the median.
I use the terms execution time and run-time synonymous and they describe the
duration in wall clock time units between two measurements.
I describe the measurement process in the previous section.

I compare the algorithms using box plots to visualize
the run-time for the major part of the measurements together with outliers.
However, due to the nature of the box plots, data points marked as outlier need
to be taken with a grain of salt.
Box plots have six components: median, lower and upper quartile, lower and
upper whisker, and outliers.
Lower quartile, median and upper quartile are actual values present in the data
set and mark 25\%, 50\%, and 75\% of the sorted data values.
The distance between the lower and upper quartile values is the \emph{inter
quartile distance} or IQD.
Depending on the IQD, the lower and upper whiskers are computed as follows:
%
\begin{align*}
  \text{lower whisker} &= \text{lower quartile} - 1.5 * \text{IQD}\\
  %
  \text{upper whisker} &= \text{upper quartile} + 1.5 * \text{IQD}
\end{align*}
%
The actual value of the lower whisker is the value of the next data point,
that is higher than the computed value.
For the upper whisker its the next lower-value data point.
Box plots consider every value beyond the value of the lower or upper whisker
as outlier.

Now the issue arises, if the inter quartile distance is low and therefore,
the whisker values are close to their respective quartile value, a lot of
data points can be marked as outliers.
While they deviate from the main amount of run-time values, the actual distance
is not necessarily high.
So a high number of marked outliers is not necessarily a sign of bad performance.

\paragraph{Cost of Balancing.}
Balancing is can be expensive run-time wise, if it migrates threads with a
large cache working set to another physical core, as the threads need to reload
their working set into the core local caches.
Furthermore, the measure-predict-decide-enforce (MPDE) cycle needs processor time,
although only on core zero in the current implementation, but the housekeeping
during the cycle can be expensive.
Because threads can leave at any time, the load balancer must check their
presence multiple times and act on failing system calls.
I believe improvements to the source code can reduce the time, but currently,
the MPDE cycle takes normally around $170\mu{}s$ when managing ten threads or around
$55\mu{}s$ when managing one thread.
However, I observed that this cycle can take up two to orders of magnitude
longer due to housekeeping issues.
The normal cycle duration depends on the number of threads the load
balancer manages and not on the length of a balancing interval.
A shorter balancing interval, however, raises the total overhead of the load
balancer, as the MPDE cycle is executed more often.

In the following, I analyse, how many cache misses must occur, to explain a one
second run-time increase of an application, when running alone on the load
balancer.
Intel documentation(\cite{intel_perf_analysis_2009}) states $60ns$ for a
\gls{llc} miss, independent
measurements\footnote{\url{http://www.7-cpu.com/cpu/Haswell.html}}
suggest $70ns$, which accounts on a $3\textit{GHz}$ CPU for a difference of $3$ to $4$ cycles
compared to the Intel statement.

The maximum amount of misses per seconds is $10^9/70 \approx 14285714$ misses.
During one interval of $100ms$ $10^8/70 \approx 1428571$ or $10ms$ $10^7/70 \approx
142857$ misses can be generated.

The memory-heavy mcf application runs $90$ seconds.
If its run-time is increased by one second due to bad balancing decisions and
subsequent cache misses, it suffers on average an additional
$14285714 / 90 \approx 158730$ cache misses per second during its run-time.
Relative to an interval length of $100ms$ that are $15873$ misses per interval.
The cache size of the core local L1 and L2 caches is $64 + 256 = 320$KB or
$5120$ cache lines \`a $64$Byte.
Producing $15873$ misses during an 100ms interval would mean to flush the
whole cache $3.1$ times.
One cache miss takes $70ns$, hence flushing the whole cache $3.1$ times takes
$5120 * 3.1 * 70ns = 1111040ns$ or $1.11104ms$.
So, if mcf suffers a slowdown of one second it has to spend $1.1ms$ each balancing
interval on additional cache misses.

However, forcing a core switch every $100ms$ might cause losing a whole cache set
once, but not thrice.
This discussion is simplified, as I discuss last level cache misses
and only account for core local caches.
When a core switch occurs, it is likely that the data can be fetched from the
\gls{llc} as it is an inclusive cache.
A \gls{llc} access needs approximately $13ns$. Assuming reuse of all $5120$ cache
lines, fetching them needs $66560ns$ or  $0.06656ms$.
That leaves around $1.05ms$ per balancing interval to account for.

I suspect occasional spikes in the MPDE cycle duration to be the reason for
this.

% todo SMT measurement issue: need micro benchmark to determine increase in
% application performance, due to smt balancing decisions.

\begin{figure}[!ht]
  \begin{subfigure}{\textwidth}
  \includegraphics[width=\textwidth]{images/finalPlots/boxplots/pdf/boxplot_gcc_mcf_solo}
  \end{subfigure}
  \begin{subfigure}{\textwidth}
  \includegraphics[width=\textwidth]{images/finalPlots/boxplots/pdf/boxplot_gamess_lbm_solo}
  \end{subfigure}
  \caption{Solo run-times visualized using one box plot for each balancing
    algorithm; STB and TB use a 100ms balancing interval.
    From top left to bottom right: SPEC gcc, mcf, gamess, and lbm.}
  \label{eval:fig:box_solo}
\end{figure}

\paragraph{Solo Execution.}
The solo run-times for each benchmark provide insight into the minimal
execution time and the program behaviour throughout different runs.
However, the benchmark is not the program under test, the balancing algorithm
is.
SLD achieves the minimal possible run-times, because it does not
migrate any running application and produces no additional overhead through
measurement and balancing action.
The green box plots in figure \ref{eval:fig:box_solo} show that nearly all SLD
run-times lie within a tenth of a second of each other.
The very deterministic run-time suggests, that SLD schedules the benchmark on
the same core each run.
Also, compared to the violet and blue plots of STB and TB, the SLD run-times
are either as good as the minimal run-time of the other two or better.
\\

\begin{figure}[!ht]
  \begin{subfigure}{\textwidth}
  \includegraphics[width=\textwidth]{images/10msPlots/boxplots/pdf/boxplot_gcc_mcf_solo_10ms}
  \end{subfigure}
  \begin{subfigure}{\textwidth}
  \includegraphics[width=\textwidth]{images/10msPlots/boxplots/pdf/boxplot_gamess_lbm_solo_10ms}
  \end{subfigure}
  \caption{Solo run-times visualized using one box plot for each balancing
    algorithm; STB and TB use a 10ms balancing interval.
    From top left to bottom right: SPEC gcc, mcf, gamess, and lbm.}
  \label{eval:fig:box_solo_10ms}
\end{figure}

%todo STB an TB solo performance discussion with 100/10 ms differences

\Gls{cfs} on the other hand is significantly faster.
The gap for the memory-bound benchmarks is smaller, than the gap for
the two compute-bound benchmarks.
The algorithm dynamically adjusts the time slices of the running programs to reduce
the scheduling overhead.
Furthermore, the consistently worse run-time of all other load balancing
algorithms suggests, that this slowdown can also be accounted to the performance
differences between monolithic kernels and micro-kernels.
\\

The results of STB and TB are good for gcc and gamess, mcf and lbm, however,
raises question about the amount of migrations and the overhead of the load
balancer.
Between figure \ref{eval:fig:box_solo} and figure \ref{eval:fig:box_solo_10ms}
lies a ten fold increase in overhead, as the interval duration changes from
$100ms$ to $10ms$.
In case of mcf and lbm, STB got a little bit slower, TB on the other hand got a
little bit faster, so the overhead ten fold increase is not necessarily the
reason for slower performance.
Due to the low number of repetitions, the shown improvements might not be
representative for lbm and mcf.
Gamess's performance is unchanged and gcc sees a little improvement under TB.

Solo runs give insights into the behaviour of the algorithms and provide
a base line to compare load runs to.


\paragraph{Concurrent Execution.}
As large as the gap between \gls{cfs} and the other balancing algorithms is in
the solo execution case, the concurrent case provides better results.
All measurements in figure \ref{eval:fig:box_all} are taken during the same
benchmarking run of the specific balancer.
Thus, the execution times of the different applications are connected, but
determining if worst case time of one benchmark correlates with best case times
of another, needs further analysis, which, again, the time budget does not
allow for.

At this point, I want to recall the above mentioned connection between outliers
and the inter quartile distance in box plots.

Where the performance of SLD for a single application was quite good, it is
no competitor if the system is loaded, as is the case in figure
\ref{eval:fig:box_all}.
The inner quartile distance for mcf, gamess, and lbm covers nearly the whole
deviation span of the other algorithms, and for gcc SLD shows an already large
IQD and upper whisker value, but still a high number of bad performing outliers.

The median run-time is as good as the other algorithms', but SLD's worst case
is more than two times worse than its median.
Furthermore, in all benchmarks is the distance between median and upper quartile
much larger than the distance between lower quartile and median.
This asymmetry shows the unpredictability of the SLD placement.
In summary, SLD provides indeterministic performance for loaded systems with a
high risk to generate a bad performing thread-to-core mapping.
%
\begin{figure}[h!]
  \begin{subfigure}{\textwidth}
  \includegraphics[width=\textwidth]{images/finalPlots/boxplots/pdf/boxplot_gcc_mcf_all}
  \end{subfigure}
  \begin{subfigure}{\textwidth}
  \includegraphics[width=\textwidth]{images/finalPlots/boxplots/pdf/boxplot_gamess_lbm_all}
  \end{subfigure}
  \caption{Concurrent run-times visualized using one box plot for each balancing
    algorithm; STB and TB use a 100ms balancing interval and all time
    measurements are taken from the same concurrent run.
    From top left to bottom right: SPEC gcc, mcf, gamess, and lbm.
    }
    \label{eval:fig:box_all}
\end{figure}
%
\begin{figure}[!ht]
  \begin{subfigure}{\textwidth}
  \includegraphics[width=\textwidth]{images/10msPlots/boxplots/pdf/boxplot_gcc_mcf_all_10ms}
  \end{subfigure}
  \begin{subfigure}{\textwidth}
  \includegraphics[width=\textwidth]{images/10msPlots/boxplots/pdf/boxplot_gamess_lbm_all_10ms}
  \end{subfigure}
  \caption{Concurrent run-times visualized using one box plot for each balancing
    algorithm; STB and TB use a 10ms balancing interval and all time
    measurements are taken from the same concurrent run.
    From top left to bottom right: SPEC gcc, mcf, gamess, and lbm.
    }
  \label{eval:fig:box_all_10ms}
\end{figure}
\\


\Gls{cfs} on the other hand, performs well, but it is not as superior as during
the solo benchmarks.
Regarding minimum and maximum run-times, it performs better than TB and STB in
case of gcc and lbm and equally well for mcf and gamess.
Gamess has a lot of outliers, which partly can be accounted to the computation
of box plots.
However, the marks above 75 seconds are more than 1.5 times slower than the
median run-time, which points to other issues with this workload.
% todo why are they a problem; where do they come from; conjunction with other
% benchmarks;
% heavy migration, possible, but close to no cache working set, therefore no
% high miss rates after migration;
% Guess: coscheduled with lbm, stress on floating point unit; -- unlikely
% Guess: coscheduling issue, as stb has less worst case and slightly better best case
%  . . .

I do not consider the printed outliers in case of gcc as a problem, because
they are within the 1.5 times median run-time range and compared to the other
algorithms, they are below their upper quartile run-time.
However, CFS is not the focus of this thesis, hence, I do not further
investigate its issues with gamess and gcc.
\\

The focus of this thesis lies on behaviour awareness, which the STB and TB
algorithms implement.
In difference to SLD, both perform a MPDE cycle every balancing interval.
The interval length in figure \ref{eval:fix:box_all} is $100ms$, whereas figure
\ref{eval:fix:box_all_10ms} shows a $10ms$ interval.

In both figures, TB's gamess shows a behaviour similar to CFS, while the
outliers in the $10ms$ case are less and not as high.
Looking at all benchmarks, TB performs better with a short interval duration:
less deviation, lower worst-case execution times, and the memory-bound
applications show also better best-case run-times.

STB is more of a mixed picture: less deviation as well, but the median
slightly decreased for all applications and also the worst-case run-times
increased.
However, when compared to TB, deviation of all benchmarks is less providing a
more deterministic result.
STB provides each application with a fair share of resources depending on the
application's run-time behaviour.
As the measurements of all applications were taken during the same run, the
best-case run-time of one is possibly connected to the worst-case run-time of
another, hence a fair load balancer provides an equal share of resources
during each execution.

However, the system load can be increased.


\paragraph{Heavy Load.}
On a quad-core machine, the four SPEC benchmarks are sufficient to fully
utilize the processor.
However, the quad-core machine with \gls{ht} has still some underutilized
resources, thus I double the workload by running each benchmark twice.
\cite{zhuravlev_addressing_2010} also turns to this approach to fully utilize
the Xeon server processor with eight physical cores, where each two cores share
one \gls{llc}.

Figure \ref{eval:fig:box_heavy} shows the results for a $100ms$ load balancing
interval for STB and TB. SLD is not part of this measurement, as the
performance shown under less load is not competitive.

Although I run two instances of each benchmark, I only use the numbers of one
instance.
The results in figure \ref{eval:fig:box_heavy} are consistent with figure
\ref{eval:fig:box_all} in the following two ways:
First, the run-time deviation is much lower for STB; second, the median
run-time is better in case of gcc, but worse for the other three applications,
however, the worst-case performance of STB is consistently better.
It is not evident from the bottom-left plot if STB is better; the
execution time of the top mark is 373.0.

Because the execution times of all applications is connected, the worse
best-case performance of one application benefits the better worst-case
performance of the other.
It is preferable to improve on the median execution time of all benchmarks and
I believe there is room for algorithmic improvements, however, a high
worst-case execution time in one long running application like mcf or lbm, allows for a
several good run-times in the short running gcc or gamess; Lbm's median is
three times as high as the gcc's or gamess's median.
So the goal besides the numbers is to provide an environment, where each
application uses as much resources as needed to perform well, while not causing
severe performance degradation to other applications.
%
\begin{figure}[h!]
  \begin{subfigure}{\textwidth}
  \includegraphics[width=\textwidth]{images/finalPlots/boxplots/pdf/boxplot_gcc_mcf_heavy}
  \end{subfigure}
  \begin{subfigure}{\textwidth}
  \includegraphics[width=\textwidth]{images/finalPlots/boxplots/pdf/boxplot_gamess_lbm_heavy}
  \end{subfigure}
  \caption{Concurrent run-times visualized using one box plot for each balancing
    algorithm; from top left to bottom right: SPEC gcc, mcf, gamess, and lbm.
    Time measurements are taken from the same concurrent run.}
    \label{eval:fig:box_heavy}
\end{figure}

\paragraph{Gamess}
Before I discuss the overall performance, I want to focus on the performance of
gamess.
On Linux the benchmark spawns one thread and runs on one core, on L4Re gamess
spawns four threads: one to work, one for the L4Re-kernel, and two unaccounted
for.
In difference to Linux, gamess prints a lot of data to console on L4Re, but as
the application is written in Fortran, maintaining it is an issue.
The measured solo run-times are consistently $55$ seconds independent of the
load balancing algorithm.
However, if gamess runs concurrently with other applications, the best-case
run-time is $47.2$ seconds, so gamess executed nearly eight seconds faster.
$47.2$ would $0.6$ seconds slower than CFS, which is a reasonable distance, but
where does this eight second gap come from?
Dynamic voltage and frequency scaling? Turned off.
Turbo Boost? Turned off.
Looking at the numbers, over 60 of 291 measurements are below 50 seconds under
load, so this is no coincidence either.
Scheduled on core zero or its companion core? Already unlikely for STB and TB,
but an additional tests showed, that running it without a load balancer on core
zero also takes $55$ seconds.
Also, using the load balancer to pin all threads to one core neither improves
the run-time.
The reason behind the execution time decrease under load remains an open
question.

\paragraph{Overall Performance.}
I compute the overall performance of each load balancing scheme as average
run-time increase relative to average solo run-time.
Figure \ref{eval:fig:exec_inc} shows a group of bars for each application
executing under each balancing scheme and the overall average.
On average CFS and STB perform equally well, whereas TB outperforms both,
however in the high load scenario CFS's performance is significantly worse
than TB's or STB's.
However, these numbers do incorporate all measurements, including the extreme
run-times of Gamess under CFS displayed in figure \ref{eval:fig:box_heavy}.
They also do not reflect the run-time deviation across multiple repetitions.
%
\begin{figure}[h!]
  \begin{subfigure}{.49\textwidth}
  \includegraphics[width=\textwidth]{images/runtimeIncrease/barchart_load_inc}
  \end{subfigure}
  \begin{subfigure}{.49\textwidth}
  \includegraphics[width=\textwidth]{images/runtimeIncrease/barchart_heavy_inc}
  \end{subfigure}
  \caption{Average increase in run-time per application and overall; for
    normal load on the left and high load on the right.}
  \label{eval:fig:exec_inc}
\end{figure}

% vim:set ft=tex:
\section{Summary}
\label{eval:summary}

The evaluation shows that topology- and behaviour-aware load balancing has
benefits regarding fairness, run-time deviation, and worst-case run-time.
The solo execution baseline shows issues with measurement spikes of the
behaviour-aware implementation.
However, as the system load increases, space-time-balancing shows less deviation
and better worst-case execution times, which suggests fair distribution of
execution resources among all managed applications.
Both time-balancing and space-time-balancing are superior in terms of
deviation and worst-case run-time compared to the
simple-load-distribution approach that does not adapt to changing
circumstances or application behaviour.

The evaluation also raised issues of the implementation.
The measure-predict-decide-enforce cycle shows high run-time spikes, which
significantly slow down the executing applications.
Besides these spikes, the MPDE cycle overhead depends on the number of managed
threads and a reasonable duration of around $170$\textmu{}s for ten managed
threads.

Some questions were not answered during this evaluation.
The impact of \gls{ht}-aware balancing remains an open question.
Nevertheless, different implementation stages showed, that bad \gls{ht} balancing visibly
impacts the performance.
The theory of a positive effect of communication groups was not supported with empirical
data, as well as the performance of other load balancing algorithms
like MPC-IPC-Placement (section \ref{impl:algos}).

The following section explores different directions for future work with this
implementation and on this topic in general.


\begin{comment}

Is behaviour aware load balancing on L4Re better than behaviour unaware
balancing?
What differences are observable between different SMT assignment schemes?

When compared with Linux \gls{cfs} scheduler, how different is the
performance?

Of interest are performance degradation compared to solo execution, when
executed in parallel with other applications.
Also, the related work suggests that the execution time is much more
deterministic, hence the maximum and minimum execution times are close
together.


\end{comment}


\cleardoublepage

%%% Local Variables:
%%% TeX-master: "diplom"
%%% End:
