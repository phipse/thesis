\section{Terminology}
\label{state:term}

In the following I make the reader familiar with the terminology of the field.
Readers may continue with section \ref{state:related} and use this section
as reference, if a term is ambiguous or unknown.

% todo Item sorting

\begin{description}
  \item[\Gls{utcb}] A \gls{utcb} is an user-level memory section containing
    information about a thread, the thread's communication registers and
    communication buffers. It is primarily used for message transfer.

  \item[\Gls{sc}] A \gls{sc} contains information for the scheduler consisting
    of a priority level, execution time, and remaining execution time.

  \item[Thread] A thread is a unit of execution, consisting of a kernel and
    \gls{utcb}, a \gls{sc}, and further interfaces.

  \item[Task] A task is a unit of isolation. It manages different address
    spaces; for example memory, capability, and I/O port spaces.
    Threads run within tasks, but tasks do not necessarily contain a thread.

  \item[Capability] A capability is a rights management concept.
    The system allows access to an object only if the caller provides a
    capability containing appropriate rights.
    It is orthogonal to access control lists, where the object knows, who may
    access it.

  \item[\Gls{ipc}] \Gls{ipc} is one way for threads in L4Re/Fiasco.OC
    to communicate with one another.
    It is synchronous communication with the following modes of operation:
    send, open wait, closed wait, reply \& open wait, and call.
    A thread initiating an open wait, waits for an arbitrary thread to contact
    him, whereas a closed wait, expects a communication intent of a specific
    thread.
    A call is normally used for communication with a service, where a reply is
    awaited, hence it combines a send and a closed wait operation.
    To initiate an \gls{ipc} in L4Re/Fiasco.OC the sending thread needs to hold
    a capability on a channel going to its communication partner.

  \item[\Gls{instpc}] \Gls{instpc} describe the number of instructions executed
    per processor cycle.

% todo Make context specific description lists?

  \item[Core] A physical CPU core is referred to as core.

  \item[Cache] Generally, a CPU cache is used to store frequently accessed data
    close to the CPU core to reduce access times.
    As a cache has size restrictions, depending on the type of cache different
    evocation policies can be applied, e.g. \gls{fifo} or \gls{lru}.
    There are multiple levels of caches in modern CPUs, creating a hierarchy;
    three levels are common.
    If modified data is evicted from the highest level of the hierarchy, it is
    written back to memory.
% todo device memory caches: fully-cached, not-cached, buffered

  \item[Cache Coherence] Data must be kept consistent between different CPU
    cache levels to provide a strict order of writes to the same memory
    relocation.
    Multi-socket x86 system employ a snooping protocol to keep caches coherent.
    Each cache broadcasts every update over the address bus and all other
    caches update their the status of their copy of the data.
    Haswell generation processors use the MESI protocol to keep their data
    coherent (\cite[2-22]{intel_optimization_manual_2015}).

  \item[Migration] The process of moving a thread from one core to another core
    is called migration.
    The cores do not necessarily need to be on the same chip, nor do they need
    to be in the same machine. Only the thread is migrated, cached data remains
    on the old core.

  \item[\Gls{eu}] EU refers to hardware components inside a
    processor core which perform a computation; for example integer or
    floating-point units.

  \item[Hardware Thread] A hardware thread describes an architectural state
    which consists among others, general-purpose, control, and machine state
    registers.
    The operating system multiplexes the software threads to the hardware
    thread.
    Only a hardware thread can access \gls{eu}s of the processor.

  \item[Simultaneous Muli-Threading (SMT)] \gls{smt} is a technique to
    multiplex several hardware threads on the same physical core.
    Thus, sharing all the core's resources -- integer and floating point units,
    caches, prefetcher, etc. -- among the core's hardware threads.
    The goal is to improve the utilization of the \gls{eu}s by increasing
    the number of instructions available to them.
    \gls{smt} transforms thread level parallelism to instruction level
    parallelism.
    I reference a \gls{smt}-core as logical core.

  \item[\Gls{ht}] \gls{ht} is the \gls{intel} specific implementation of the
    \gls{smt}-concept.

  \item[\Gls{cmp}  \& \Gls{smp}]
    A \Gls{smp} architecture describes a set of completely equal physical
    cores sharing the \gls{llc}.
    \Gls{cmp} architectures on the other hand, consist of two equal processors
    soldered on one die, thus sharing nothing.
    In both cases, all cores follow a cache-coherency protocol.

  \item[\Gls{isa}]\gls{isa} describes the set of hardware instructions executed
    by a processor.

  \item[Intervall] Describes a pre-defined amount of time.
  \item[Wall clock time] Describes the time passed in real life.
  \item[Execution time] Describes the time a thread spends executing on a
    core.
  \item[Turnaround time] Time between submission of a job and its completion.




\end{description}
