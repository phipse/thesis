\section{Terminology}
\label{state:term}

The following makes the reader familiar with the terminology of the field.
Advanced readers may skip this section and use it as reference, if a term
used in later sections is ambiguous unknown.

% list of glossary entries
%\newacronym{intel}{}{INTEL\textregistered{}}
\newglossaryentry{intel}{name=Intel\textregistered{},
description={Intel\textregistered{} Corp.} }

% list of acronyms
\newacronym{ht}{HT}{hyper-threading}
\newacronym{smt}{SMT}{simultaneous multi-threading}
\newacronym{cmp}{CMP}{chip multiprocessor}
\newacronym{smp}{SMP}{symmetric multiprocessor}
\newacronym{utcb}{UTCB}{user thread control block}
\newacronym{sc}{SC}{scheduling context}
\newacronym{fifo}{FIFO}{First-in-First-out}
\newacronym{lru}{LRU}{least-recently-used}
\newacronym{llc}{LLC}{last-level-cache}
\newacronym{ipc}{L4-IPC}{L4 inter-process communication}
\newacronym{instpc}{IPC}{instructions per cycle}
\newacronym{isa}{ISA}{Instruction Set Architecture}

\todo{Item sorting}

\begin{description}
  \item[\Gls{utcb}] A \gls{utcb} is a user-level memory section containing
    information about a thread, the thread's communication registers and
    communication buffers. It is primarily used for message transfer.

  \item[\Gls{sc}] A \gls{sc} contains information for the scheduler consisting
    of a priority level, execution time, and remaining execution time.

  \item[Thread] A thread is a unit of execution, consisting of a \gls{utcb}
    and a \gls{sc}.

  \item[Task] A task is a unit of isolation. It manages different address
    spaces; for example memory, capabilities, and ioports.
    A task runs at least one thread.

  \item[\Gls{ipc}] \Gls{ipc} is one way for threads in L4Re/Fiasco.OC
    to communicate with one another.
    It is synchronous communication with the following modes of operation:
    send, open wait, closed wait, reply \& open wait, call.
    A thread initiating an open wait, waits for an arbitrary thread to contact
    him, whereas a closed wait, expects a communication intend of a specific
    thread.
    A call is normally used for communication with a service, where a reply is
    awaited, hence it combines a send and a closed wait operation.
    To initiate an \gls{ipc} in L4Re/Fiasco.OC the sending thread needs to hold
    a capability on a channel going to its communication partner.

  \item{\Gls{instpc}] \Gls{instpc} describe the number of instructions executed
    per processor cycle.

  \todo{Make context specific description lists?}

  \item[Core] A physical CPU core is referred to as core.

  \item[Cache] Generally, a cache is used to store frequently accessed data
    in the CPU core to reduce access times.
    As a cache has size restriction, depending on the type of cache different
    evocation policies can be applied, e.g. \gls{fifo} or \gls{lru}.
    There are multiple levels of caches in modern CPUs, creating a hierarchy;
    three levels are common.
    If data is evicted from the highest level of the hierarchy, it is written
    back to memory.
    \todo{Write-through and write-back cache?}

  \item[Migration] The process of moving a thread from one core to another core
    is called migration.
    The cores do not necessarily need to be on the same chip, nor do they need
    to be in the same machine. Only the thread is migrated, cached data remains
    on the old core.

  \item[\Gls{smt}] \gls{smt} is a technique to multiplex several hardware
    threads on the same physical core.
    Thus, sharing all the cores resources -- ALU, FP units, caches, prefetcher,
    etc. -- among the core's hardware threads.
    The goal is to improve the utilization of the execution units by increasing
    the number of instructions available to them.
    \gls{smt} transforms thread level parallelism to instruction level
    parallelism.
    A \gls{smt}-thread is referenced as logical core or hardware thread.

  \item[\Gls{ht}] \gls{ht} is the \gls{intel} specific implementation of the
    \gls{smt}-concept.

  \item[\Gls{cmp}  \& \Gls{smp}] Regarding cache coherence, \gls{cmp} and
    \gls{smp} architectures equal.
    However, \gls{cmp} style CPU consists of several CPU chips soldered on one
    die.
    This leads to physically separated caches, limiting the amount of cache
    available to a single thread.
    In constrast, modern \gls{smp} architectures share the \gls{llc} among all
    cores on one die.

  \item[\Gls{isa}]\gls{isa} describes the set of hardware instructions executed
    by a processor.

  \item[Intervall] Describes a pre-defined number of clock ticks.

  \item[Hardware Performance Counters] \todo{newacronym?} granularity: (per
    cpu, per core, per logical core);
    update interval -- 1ms?;
    security risks?

  \item[Job] In scheduling theory a job is a specific unit of work.
    In real-time systems, the execution time of a job is known beforehand.
    \todo{difference between jobs and threads?}

  \item[Turnaround time] Time between submission of a job and its completion.

  \item[Makespan] Time to execute a set of jobs from beginning to end.




\end{description}
