\section{Terminology}
\label{state:term}

The following makes the reader familiar with the terminology of the field.
Advanced readers may skip this section and use it as reference, if a term
used in later sections is ambiguous unknown.

% list of glossary entries
%\newacronym{intel}{}{INTEL\textregistered{}}
\newglossaryentry{intel}{name=Intel\textregistered{},
description={Intel\textregistered{} Corp.} }

% list of acronyms
\newacronym{ht}{HT}{hyper-threading}
\newacronym{smt}{SMT}{simultaneous multi-threading}
\newacronym{cmp}{CMP}{chip multiprocessor}
\newacronym{smp}{SMP}{symmetric multiprocessor}
\newacronym{utcb}{UTCB}{user thread control block}
\newacronym{sc}{SC}{scheduling context}
\newacronym{fifo}{FIFO}{First-in-First-out}
\newacronym{lru}{LRU}{least-recently-used}
\newacronym{llc}{LLC}{last-level-cache}
\newacronym{ipc}{IPC}{inter-process communication}

\todo{Item sorting}

\begin{description}
  \item[\Gls{utcb}] A \gls{utcb} is user-level memory section containing
    information about a thread and the thread's communication registers and
    buffers.
    \todo{UTCB description}

  \item[\Gls{sc}] A \gls{sc} contains information for the scheduler consisting
    of a priority level, execution time, and remaining execution time.
    \todo{What other items are in a scheduling context}

  \item[Thread] A thread is a unit of execution, consisting of a \gls{utcb}
    and a \gls{sc}.

  \item[Task] A task consists of a one or more threads and provides memory and
    memory management for it's threads.

  \item[\Gls{ipc}] \Gls{ipc} is the only way for threads in L4Re/Fiasco.OC
    to communicate with one another.
    It is synchronous communication with the following modes of operation:
    send, open wait, closed wait, reply \& open wait, call.
    A thread initiating an open wait, waits for an arbitrary thread to contact
    him, whereas a closed wait, expects a communication intend of a specific
    thread.
    A call is normally used for communication with a service, where a reply is
    awaited, hence it combines a send and a closed wait operation.
    \todo{atomic?}
    To initiate an \gls{ipc} in L4Re/Fiasco.OC the sending thread needs to hold
    a capability on the receiving thread.
    \todo{IPC -> instructions per cycle}

  \todo{Make context specific description lists?}

  \item[Core] A physical CPU core is referred to as core.

  \item[Cache] Generally, a cache is used to store frequently accessed data.
    As a cache has size restriction, depending on the type of cache different
    evocation policies can be applied, e.g. \gls{fifo} or \gls{lru}.
    The hardware cache present on a CPU uses \gls{lru}.
    A cache in general can have an arbitrary count of levels, creating a cache
    hierarchy.
    In current CPUs three cache levels are common.
    If data is evicted in a lower hierarchy level, it moves to a higher level.
    If there is no higher level, the data is written back to memory.
    \todo{Write-through and write-back cache?}

  \item[Migration] The process of moving a thread from one core to another core
    is called migration.
    The cores do not necessarily need to be on the same chip, nor do they need
    to be in the same machine.

  \item[\Gls{smt}] \gls{smt} is a technique to multiplex several hardware
    threads on the same physical core.
    Thus, sharing all the cores resources (ALU, FP units, caches, prefetcher,
    etc.) among the core's hardware threads.
    The goal is to improve the utilization of the core, by increasing the
    available instructions and, hence, the possibility that each computation
    unit is utilized.
    This transfers thread level parallelism to instruction level parallelism.
    A \gls{smt}-thread is referenced as logical core or hardware thread.

  \item[\Gls{ht}] \gls{ht} is the \gls{intel} specific implementation of the
    \gls{smt}-concept.

  \item[\Gls{cmp}] \gls{cmp} is a category of processor architectures, where
    the CPU consists of two equal CPU chips soldered together on one die.
    As a result the \gls{llc} is not shared among all cores.
    Consider a quad-core CPU build out of two dual-core CPUs.
    Hence, cores 0 \& 1 share their \gls{llc} and cores 2 \& 3, respectively.

  \item[\Gls{smp}] \gls{smp} is a category of processor architectures, where
    all cores are equal and the \gls{llc} is either shared among all of them or
    each core has it's own \gls{llc}.

  \item[Intervall] Describes a pre-defined number of clock ticks.

  \item[Hardware Performance Counters] \todo{newacronym?} granularity: (per
    cpu, per core, per logical core);
    update interval -- 1ms?;
    security risks?

  \item[Job] In scheduling theory a job is a specific unit of work. Normally,
    the execution time of a job is known beforehand.
    \todo{difference between jobs and threads?}

  \item[Turnaround time] Time between submission of a job and its completion.

  \item[Makespan] Time to execute a set of jobs from beginning to end.




\end{description}
