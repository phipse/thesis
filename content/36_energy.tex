% vim:set ft=tex:
\section{Energy model}
\label{design:energy}

Energy consumption is important in modern systems, as battery powered devices
become more popular.
But observing energy consumption is not part of this thesis.
However, I will derive a model for energy consumption on x86 from work presented
in section \ref{state:energy} and design an energy efficient load balancing
strategy, based on this model.
\\

The critical question for an energy model is:
When is least energy consumed?
If the work is done fast and the cores sleep for a longer time or if the work
is done at lower frequencies for a longer time.

Assume the system is only at half utilization at the lowest possible frequency.
In \cite{le_sueur_slow_2011} it is assumed, that quality of service is
maintained, even at low frequencies.
They showed, for their three applications that the energy efficiency is best
for low frequencies, but depending on the application, at the cost of
throughput.
So it is highly application specific to find the right frequency for the
running application.
The complexity increases, if more than one application is present in the
system.
Furthermore, a load balancer cannot decide on core frequencies.

\citeauthor{le_sueur_slow_2011} noted that turbo boost -- raising the frequency
of one core above standard, if other cores are idle -- should be avoided at
all cost, as the performance gain is little compared to the increased energy
consumption.
Usage of turbo boost can be avoided, if all cores are loaded, so the
prerequisites are not met, or by completely disabling it.

Another relevant observation in \cite{le_sueur_slow_2011} was, that the usage
of deep sleep states and the related restart costs only slightly increase the
system load.
If deep sleep states are used, the system load is low anyway, so the additional
load poses no problem.
\\

\cite{imes_poet_2015} states that different models are necessary for different
platforms.
A mobile Haswell generation processor consumes least energy, if the race to
idle scheme is used, whereas an embedded Exynos5 generation CPU is designed for
constant low frequency operation.
The target hardware for this work is a Haswell generation processor, so the
race to idle model looks promising.
\\

The difference in energy consumption between low frequency and normal
frequency, however, is little compared to a deep sleep state, due to the
already high base power.
So following \cite{imes_poet_2015}, it is best to increase sleep time and use
the normal frequency to execute the arising work quickly.

The model parameters for best energy efficiency are:
No turbo boost, either execute at normal frequency or sleep, and use as many
cores as possible, to reduce cache misses and consequently execution time.

A reduction in runtime leads to larger sleep times or more throughput, which is
the goal of a load balancer all along.
Therefore, a algorithm maximizing throughput also increases energy efficiency
on the x86 architecture.


% todo Can POET be used in conjunction with a load balancer? Do they benefit each
% other?
