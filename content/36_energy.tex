% vim:set ft=tex:
\section{Energy model}
\label{design:energy}

Energy consumption is important in modern systems, as battery powered devices
become more popular.
But observing energy consumption is not part of this thesis and energy
measurement itself consumes energy.
Hence, a model describing energy consumption is derived from related work
presented in section \ref{state:related}.

\cite{imes_poet_2015} showed that different models are necessary for different
platforms.
A mobile Haswell generation processor consumes least energy, if the race to
idle scheme is used, whereas an embedded Exynos5 generation CPU is designed for
constant low frequency operation.
The target hardware for this work is a Haswell generation processor, so the
race to idle model looks promising.
However, the additional energy cost of turbo boost should be avoided, as noted
in \cite{le_sueur_slow_2011}.
Turbo boost can be avoided by either completely disabling the feature or by
distributing work to all cores.

\cite{le_sueur_slow_2011} also suggest, to use deep sleep states as much as
possible, as the overhead of reactivating the core is small and if a deep sleep
state can be considered, the overall system load is low enough to cope with
this additional load.
% least energy consumption in deep sleep states
% energy account = working time energy + sleeping time energy
% overhead through deep sleep states is small
%
