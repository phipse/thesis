% vim:set ft=tex:
\section{Energy Model}
\label{design:energy}

Energy consumption is important in modern systems, as battery powered devices
become more popular.
Measuring energy consumption is not part of this thesis.
However, I will derive a model for energy consumption on x86 from work presented
in section \ref{state:related} and design an energy efficient load balancing
strategy, based on this model.
\\

A critical question for an energy model is:
When is the least amount of energy consumed?
In a race-to-idle scenario, where work is done fast and the cores sleep for a
longer time, or if the work is done at lower frequencies for a longer time.
In \cite{le_sueur_slow_2011} \citeauthor{le_sueur_slow_2011} assume that
quality of service is maintained, even at low frequencies.
They show for their three applications that the
energy efficiency is best for low frequencies, but depending on the
application, at the cost of throughput.
So it is highly application-specific to find the right frequency for the
running application.
The complexity increases, if more than one application is present in the
system.
Furthermore, a load balancer cannot decide on core frequencies, because its
view on the system is not necessarily complete.

\citeauthor{le_sueur_slow_2011} note that turbo boost should be avoided at
all cost, as the performance gain is little compared to the increased energy
consumption.
The hardware does not use turbo boost as long as all cores are loaded, as the
energy budget does not allow it.
Also, it is possible to disable the feature completely.


Another relevant observation in \cite{le_sueur_slow_2011} is, that the usage
of deep sleep states and the related restart costs only slightly increase the
system load.
If a core goes into a deep sleep state, the system load is low anyway, so the
additional wakeup-load poses no problem.
\\

\cite{imes_poet_2015} states that different models are necessary for different
platforms.
A mobile Haswell generation processor consumes the least amount of energy if
the race-to-idle scheme is used, whereas an embedded Exynos5 generation CPU is
designed for constant low frequency operation.
The target hardware for this work is a Haswell generation processor, so the
race to idle model seems promising.
\\

The difference in energy consumption between low frequency and normal
frequency, however, is little compared to a deep sleep state, due to the
already high base energy consumption.
So following \cite{imes_poet_2015}, it is best to increase sleep time and use
the normal frequency to execute the arising work quickly.

The model parameters for best energy efficiency are:
No turbo boost, either execute at normal frequency or sleep, and use as many
cores as possible, to reduce cache misses and consequently execution time.

A reduction in run-time leads to larger sleep times or more throughput, which is
the goal of a load balancer all along.
Therefore, an algorithm maximizing throughput also increases energy efficiency
on the x86 architecture.


% todo Can POET be used in conjunction with a load balancer? Do they benefit each
% other?
