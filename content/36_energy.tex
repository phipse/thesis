% vim:set ft=tex:
\section{Energy model}
\label{design:energy}

Energy consumption is important in modern systems, as battery powered devices
become more popular.
But observing energy consumption is not part of this thesis.
However, based on a energy consumption model, an energy efficient load
balancing strategy is designed.
This model is derived from related work presented in section
\ref{state:related}.

The critical question for an energy model is:
When is least energy consumed?
If the work is done fast and the cores sleep for a longer time or if the work
is done at lower frequencies for a longer time.

Assume the system is only at half utilization at the lowest possible frequency.
In \cite{le_sueur_slow_2011} it is assumed, that quality of service is
maintained, even at low frequencies.
They showed, for their three applications that the energy efficiency is best
for low frequencies, but depending on the application, at the cost of
throughput.
So it is highly application specific to find the right frequency for the
running application.
The complexity increases, if more than one application is present in the
system.
Furthermore, it is not the job of the load balancer to decide on core
frequencies.

\citeauthor{le_sueur_slow_2011} noted that turbo boost -- raising the frequency
of one core above standard, if other cores are idle -- should be avoided at
all cost, as the performance gain is little compared to the increased energy
consumption.
Usage of turbo boost can be avoided, if all cores are loaded, so the
prerequisites are not met, or by disabling it in the BIOS.

Another relevant observation in \cite{le_sueur_slow_2011} was, that the usage
of deep sleep states and the related restart costs only slightly increase the
system load.
If deep sleep states are used, the system load is low anyway, so the additional
load poses no problem.
\\

\cite{imes_poet_2015} states that different models are necessary for different
platforms.
A mobile Haswell generation processor consumes least energy, if the race to
idle scheme is used, whereas an embedded Exynos5 generation CPU is designed for
constant low frequency operation.
The target hardware for this work is a Haswell generation processor, so the
race to idle model looks promising.
\\

It is assumed, that a good energy efficient model makes use of race to idle on all
cores, avoiding turbo boost and maximizing the time for deep sleep states.

% todo Can POET be used in conjunction with a load balancer? Do they benefit each
% other?
