\section{Summary}
\label{state:summ}
\todo{rename to Summary, do summary of chapter}

\begin{itemize}
  \item introduction to relevant terminology
  \item SMT, symbiotic scheduling, co-schedule, measurement difficulties
  \item Congestion awareness of shared resources; Goal minimal slowdown for all
    applications; fairness, cache weights, pain metric, reliance on default
    scheduler, performance counters
  \item communication awareness: compute-communicate cycle, distribution of one
    application
  \item don't haves: No CMP architecture -> SMT-SMP combination; client server
    communication model;
  \item Fiasco.OC: no logic, just mechanisms in the kernel, hence no balancing;
  \item L4Re: no automatic, behaviour based balancing.
  \item do: communication models: HPC, CL-SVR; behaviour analysis: perf
    counters; SMT-SMP processor awareness, reduced conflicts in shared hardware
    resources, only on-line measurements;
  \item secure/real-time/exclusive core design.
\end{itemize}


\begin{itemize}
  \item Current processor architectures don't use CMP processors any more, therefore
    the cache layout is different. My work evaluates the research results on the
    new HW layout. But AMD Opteron Barcelona had a similar cache layout.
  \item User level scheduling on a $µ$-kernel. But user level scheduling was
    done before.
  \item More scheduling parameters \textit{or} less assumptions about threads.
  \item No off-line measurements, only on-line information gathering.
  \item Thread interaction possible (communication partner, security flag)
  \item Designated cores for security critical applications.
\end{itemize}


% -----------------------------------------------------------------------------
