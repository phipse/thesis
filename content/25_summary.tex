\section{Summary}
\label{state:summary}

\begin{comment}
After I introduce the reader to relevant terminology, I present related
research about symbiotic \gls{smt} co-schedules and the problem of accurately
measuring the performance of a specific co-schedule.
The prime are of application for load balancing is resource congestion. Research by
\citeauthor{knauerhase_using_2008}, \citeauthor{banikazemi_pam_2008}, and
\citeauthor{zhuravlev_addressing_2012} shows that separating application
competing for cache and memory decreases and stabilises their execution times.
The research presents \gls{llc} misses of an application as indicator for
memory usage and stress put on the memory subsystem.

\cite{hofmeyr_load_2010} balances a single application with several threads.
When all threads reach a synchronization barrier and they communicate their
results.
Hence, the slowest thread lets all other threads wait.
The authors develop a model to provide each thread with equal progress
opportunity.
The knowledge of this communication relationship determines the decision of
the load balancer.

Computations are reflected in cache usage of an application.
PRIME+PROBE attacks analyse the usage of cache lines of the targeted
application and can rebuild a cryptographic secret from the observations.
Placing security critical tasks on other cores than possible attackers, reduces
the amount of cache shared, hence raises the bar for an attack.
\\

In difference to the related work, the goal in this thesis is to design and
implement a load balancer, which mainly uses online measurements to decide on a
thread-to-core mapping.
Also, it should consider communication relationships and security requirements.
While previous research uses \gls{cmp} style processors, I use a \gls{smp} type
processor.
\Gls{cmp} is more similar to a multi-socket system, due to the separated
\gls{llc} between two processors on one die.
\end{comment}

I present separate approaches to increase SMT performance through symbiotic
co-schedules, to reduce congestion on shared caches through analysis of
\gls{llc} misses, to improve application runtime through knowledge about the
communication behaviour of its threads, and to determine cryptographic secrets
through the cache side-channel attacks.

My goal in this thesis is to unite all these topics in a load balancing
service for Fiasco.OC/L4Re.
The load balancer should: provide applications with security requirements with
a separate core and cache, to reduce the attack surface for cache side-channel
attacks;
consider communication relationships to reduce communication latencies;
determine memory usage through hardware performance counters and provide cache
intense applications with as much cache as possible;
achieve good co-schedule performance on two corresponding logical cores.

I implement the prototype service for one specific architecture, an \gls{intel}
Haswell Core i7-4660K.

% todo not mentioned: energy, linux cfs, details about each single topic


\begin{comment}

\begin{itemize}
  \item introduction to relevant terminology
  \item SMT, symbiotic scheduling, co-schedule, measurement difficulties
  \item Congestion awareness of shared resources; Goal minimal slowdown for all
    applications; fairness, cache weights, pain metric, reliance on default
    scheduler, performance counters
  \item communication awareness: compute-communicate cycle, distribution of one
    application
  \item don't haves: No CMP architecture -> SMT-SMP combination; client server
    communication model;
  \item Fiasco.OC: no logic, just mechanisms in the kernel, hence no balancing;
  \item L4Re: no automatic, behaviour based balancing.
  \item do: communication models: HPC, CL-SVR; behaviour analysis: perf
    counters; SMT-SMP processor awareness, reduced conflicts in shared hardware
    resources, only on-line measurements;
  \item secure/real-time/exclusive core design.
\end{itemize}


\begin{itemize}
  \item Current processor architectures don't use CMP processors any more, therefore
    the cache layout is different. My work evaluates the research results on the
    new HW layout. But AMD Opteron Barcelona had a similar cache layout.
  \item User level scheduling on a $µ$-kernel. But user level scheduling was
    done before.
  \item More scheduling parameters \textit{or} less assumptions about threads.
  \item No off-line measurements, only on-line information gathering.
  \item Thread interaction possible (communication partner, security flag)
  \item Designated cores for security critical applications.
\end{itemize}

\end{comment}

% -----------------------------------------------------------------------------
