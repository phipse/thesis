% vim:set ft=tex:
\section{Summary}
\label{eval:summary}

The evaluation shows, that topology and behaviour-aware load balancing has its
benefits regarding fairness, run-time deviation, and worst-case run-time.
While the solo execution baseline shows overhead issues of the behaviour-aware
implementation, it also shows slower execution on top of L4Re compared to
GNU/Linux.
However, as the system load increases space-time-balancing shows less deviation
and better worst-case execution times, which suggests fair distribution of
execution resources among all managed applications.
Both, time-balancing and space-time-balancing are superior than the
simple-load-distribution approach that does not adapt to changing
circumstances or application behaviour.

The evaluation also raised issues of the implementation.
The measure-predict-decide-enforce cycle shows enormous run-time spikes, which
significantly slow down the executing applications.
Besides these spikes the MPDE cycle overhead depends on the number of managed
threads and a reasonable duration of around $170\mu{}s$ for ten managed
threads.

Some question are not answered during this evaluation.
The impact of \gls{ht}-aware balancing remains an open question, however,
different implementation stages showed, that bad \gls{ht} balancing visibly
impacts the performance.
The reason for Gamess's five to seven second faster execution under load
remains unknown, as well as the performance of other load balancing algorithms
like MPC-IPC-Placement (section \ref{impl:algos}).

The following section explores different directions for future work with this
implementation and on this topic in general.
