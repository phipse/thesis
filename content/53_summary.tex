% vim:set ft=tex:
\section{Summary}
\label{eval:summary}

The evaluation shows that topology- and behaviour-aware load balancing has
benefits regarding fairness, run-time deviation, and worst-case run-time.
The solo execution baseline shows issues with measurement spikes of the
behaviour-aware implementation.
However, as the system load increases, space-time-balancing shows less deviation
and better worst-case execution times, which suggests fair distribution of
execution resources among all managed applications.
Both time-balancing and space-time-balancing are superior in terms of
deviation and worst-case run-time compared to the
simple-load-distribution approach that does not adapt to changing
circumstances or application behaviour.

The evaluation also raised issues of the implementation.
The measure-predict-decide-enforce cycle shows high run-time spikes, which
significantly slow down the executing applications.
Besides these spikes, the MPDE cycle overhead depends on the number of managed
threads and a reasonable duration of around $170$\textmu{}s for ten managed
threads.

Some questions were not answered during this evaluation.
The impact of \gls{ht}-aware balancing remains an open question.
Nevertheless, different implementation stages showed, that bad \gls{ht} balancing visibly
impacts the performance.
The theory of a positive effect of communication groups was not supported with empirical
data, as well as the performance of other load balancing algorithms
like MPC-IPC-Placement (section \ref{impl:algos}).

The following section explores different directions for future work with this
implementation and on this topic in general.
