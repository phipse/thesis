% vim:set ft=tex:
\chapter{Future Work}
\label{sec:futurework}

\newglossaryentry{arm}{name=ARM\textregistered{},
description={ARM\textregistered{}} }

After evaluating the benefits and constraints of behaviour-aware load
balancing, I present different opportunities for improvements.
This thesis provides a basis for further experiments, be it to improve SMT
utilization, better isolation between task groups, or specialized load
balancing algorithms.
The modular structure allows to change specific parts and accommodate changes.

\paragraph{Precise Event Based Sampling}
PEBS allows the user to register an interrupt coupled to a specific counter.
If the counter is increased by an event or overflows, the interrupt triggers
and allows to read the hardware state from a prespecified memory region.
This feature helps a developer to pinpoint, for example, memory accesses which
resulted in a cache miss.

PEBS overflow interrupt could eliminate the overflow check in the thread
accounting code path of the kernel.
The interrupt handler must then increase the account of the currently running
thread.

\paragraph{\gls{intel} Cache Allocation Technology (CAT)}
The most promising improvement is CAT\footnote{
  \url{http://www.intel.com/content/dam/www/public/us/en/documents/white-papers/cache-allocation-technology-white-paper.pdf}}.
It allows the user to partition the \gls{llc} or more precisely to decide
which core is able to write to which cache lines in the \gls{llc}.
However, a cache hit can also be happen outside of this area.
This feature allows to configure the system, such that security or latency
critical applications run on a exclusive core, which uses a portion of the
\gls{llc} no other core has write access to.
Only the core with write access to this memory area can evict cache lines.

I expect this feature to eliminate the cache as a side-channel for PRIME+PROBE
attacks.
Even if the attacker application runs on the same processor and shares the same
\gls{llc}, it cannot write to the attacked memory.
Also, real-time applications can now rely on a set of cache lines being
present, if the application does not evict the lines itself.

The load balancer presented in  this thesis can now not only provide temporal
separation, with the help of CAT it can also provide spatial isolation on
shared hardware.
This hardware feature is present in Broadwell or later generation CPUs.

\paragraph{Multi-socket systems}
Besides CAT, usage of multi-socket systems can also prevent PRIME+PROBE attacks
or provide a real-time task group with an environment with less interferences.
Currently, the load balancer does not support multiple sockets.
First, the CPUID based topology analysis must be extended to support multiple
packages and second, the CoreAccounting must be adapted to use this new
topology.

\paragraph{Heterogeneous Architectures}
A more complex extension is the support for heterogeneous architectures, like
\gls{arm} big.LITTLE.
\citeauthor{sarma_smartbalance_2015} use a predefined matrix to predict the
performance change to an application, if it is migrated to another core type.
But they do not consider communication or isolation groups.
How should isolation groups be placed on a heterogeneous architecture?
Does a group only get one big or little core?
Does a group get one core of each type?
When the group is dynamically migrated depending on its needs, how fast can a
core be cleared for this group?
How large is the performance impact the other applications suffer from such a
clear operation?

%todo revise this paragraph
How are communication groups distributed?
When a client is migrated to a big core, is the server migrated, too?
Should \texttt{distr} groups only use big cores?

The energy consumption is also very different compared to a x86 system.
If performance per watt is important, \texttt{distr} groups should maybe be
prevented from using big cores.

In this discussion I focus on a heterogneous architecture with two different
core sizes, but the questions regarding isolation and communication groups
already become difficult to answer.



\paragraph{Behaviour Categories}
An approach I did not explore during this work are behaviour categories.
Consider four categories: very high \gls{mpc}, high \gls{mpc}, low \gls{mpc},
very low \gls{mpc}.
Each thread belongs to one category and the algorithm selects threads of
different categories to run together on one core.
The assignment of threads to categories could also take a behaviour history
into account and categorize the thread regarding its general behaviour.

I also expect behaviour categories to help decide on difficult questions
arising from heterogeneous architectures.


\paragraph{Workload-aware balancing}
The modular approach allows quick replacement and testing of different
algorithms in the various components.
If very specific workloads run on the system, tuned balancing algorithms for
this specific workload can be used to increase the performance of the system.


\cleardoublepage

%%% Local Variables:
%%% TeX-master: "diplom"
%%% End:
