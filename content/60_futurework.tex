% vim:set ft=tex:
\chapter{Future Work}
\label{sec:futurework}

After evaluating the benefits and constraints of behaviour-aware load
balancing, I present different opportunities for improvements.
This thesis provides a basis for further experiments, be it to improve SMT
utilization, better isolation between task groups, or specialized load
balancing algorithms.
The modular structure allows to change specific parts and accommodate changes.

\paragraph{Precise Event Based Sampling}
PEBS allows the user to register an interrupt coupled to a specific counter.
If the counter is increased by an event or overflows, the interrupt triggers
and allows to read the hardware state from a prespecified memory region.
This feature helps a developer to pinpoint for example memory accesses which
resulted in a cache miss.

PEBS overflow interrupt could eliminate the overflow check in the thread
accounting code path of the kernel.
The interrupt handler must then increase the account of the currently running
thread.

\paragraph{\gls{intel} Cache Allocation Technology (CAT)}
The most promising improvement is CAT.
It allows the user to partition the \gls{llc} or more precisely to decide
which share of the \gls{llc} corresponds to which core.
This allows to provide security or latency critical applications to run on a
specific core with an exclusive share of the \gls{llc}.
Processes running concurrently on other cores cannot write to exclusive
cache areas of other cores, hence the cache lines residing in the cores share
are only evicted, if the application running on the core does so.

I expect this feature to eliminate the cache as a side-channel, even if the
attacker application runs on the same processor with the same \gls{llc}.
The load balancer presented in  this thesis, can now not only provide temporal
separation, with the help of CAT is can also provide spatial isolation on
shared hardware.
This hardware feature is present in Broadwell or later generation CPUs.

\paragraph{Multi socket systems}
Besides CAT, usage of multi socket systems can also prevent PRIME+PROBE attacks
or provide a real-time task group with an environment with less interferences.
Currently, the load balancer does not support multiple sockets.
First, the CPUID based topology analysis must be extended to support multiple
packages and second, the CoreAccounting must be adapted to use this new
topology.

\paragraph{Workload-aware balancing}
The modular approach allows to quickly replace and test different algorithms in
different components.
If very specific workloads run on the system, tuned balancing algorithms for
this specific workload can be used to increase the performance of the system.


\paragraph{Behaviour Categories}
An approach I did not explore during this work are behaviour categories.
Consider four categories: very high \gls{mpc}, high \gls{mpc}, low \gls{mpc},
very low \gls{mpc}.
Each thread belongs to one category and the algorithm selects threads of
different categories to run together on one core.
The assignment of threads to categories could also take a behaviour history
into account and categorize the thread regarding its general behaviour.


\cleardoublepage

%%% Local Variables:
%%% TeX-master: "diplom"
%%% End:
