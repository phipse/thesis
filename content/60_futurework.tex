% vim:set ft=tex:
\chapter{Future Work}
\label{sec:futurework}

\newglossaryentry{arm}{name=ARM\textregistered{},
description={ARM\textregistered{}} }

After evaluating the benefits and constraints of behaviour-aware load
balancing, I present different opportunities for improvements.
This thesis provides a basis for further experiments, be it to improve
isolation between task groups, expansion to multi-socket systems, or
heterogeneous architectures.

\paragraph{Precise Event Based Sampling (PEBS).}
PEBS is a feature on Intel-based processors that allows the user to register
an interrupt coupled to a specific counter.
If the counter is increased by an event or overflows, the interrupt triggers
and allows to read the hardware state from a prespecified memory region.
This feature helps a developer to pinpoint, for example, memory accesses which
resulted in a cache miss.

PEBS overflow interrupt could eliminate the overflow check in the thread
accounting code path of the kernel.
The interrupt handler must then increase the account of the currently running
thread by the maximum counter value and clear the overflow flag.

\paragraph{\gls{intel} Cache Allocation Technology (CAT).}
The most promising improvement on Intel processors is CAT (\cite{intel_cat}).
It allows the user to partition the \gls{llc} or, more precisely to decide
which core is able to write to which cache lines in the \gls{llc}.
However, the core can still read the entire cache.
This feature allows to configure the system, such that security or latency
critical applications run on an exclusive core, which uses a portion of the
\gls{llc} no other core has write access to.
Only the core with write access to this memory area can evict cache lines.

I expect this feature to eliminate the cache as a side-channel for PRIME+PROBE
attacks.
Even if the attacker application runs on the same processor and shares the same
\gls{llc}, it cannot write to the attacked memory.
Also, real-time applications can now rely on a set of cache lines being
present, if the application does not evict the lines itself.

The load balancer I present in  this thesis can now not only provide temporal
separation, but with the help of CAT it can also provide spatial isolation on
shared hardware.

However, this hardware feature is currently only present in Broadwell or later
generation Xeon\texttrademark{} server CPUs.

\paragraph{\gls{intel} Cache Monitoring Technology (CMT).}
CMT allows the monitoring of individual or a group of threads, applications, or
virtual machines by using a resource monitoring ID (RMID)\footnote{
  \url{https://software.intel.com/en-us/blogs/2014/12/11/intel-s-cache-monitoring-technology-software-visible-interfaces}}.
It monitors the cache usage of all objects with the same RMID and provides
the values via control and data MSRs\footnote{Model Specific Register}.
As this feature measures actual cache usage, I expect it to provide better
information for spatial balancing than \gls{llc} misses, but \gls{llc} misses still
indicate the load on prefetchers and the memory interface, which is not
displayed by CMT.

However, this feature is only available in Haswell or later generation
Xeon\texttrademark{} server CPUs.

\paragraph{Multi-Socket Systems.}
Besides CAT, usage of multi-socket systems can also prevent PRIME+PROBE attacks
or provide a real-time task group with an environment with less interference.
Currently, the load balancer does not support multiple sockets.
First, the CPUID based topology analysis must be extended to support multiple
packages and second, the core accounting must be adapted to use this new
topology.

\paragraph{Heterogeneous Architectures.}
A more complex extension is the support for heterogeneous architectures, like
\gls{arm} big.LITTLE.
\citeauthor{sarma_smartbalance_2015} (\cite{sarma_smartbalance_2015}) use a
predefined matrix to predict the performance change of a thread,
if it is migrated to another core type.
However, the authors do not consider communication or isolation realtionships.
Additionally, thread groups raise numerous questions: Which core type do isolation groups
need?
If an isolation group is migrated to another core type, how large is the
performance  impact to the rest of the system?
Which core type uses a server task with multiple distributed clients?

The energy consumption is also very different compared to an x86 system.
If performance per energy is important, is it better to run \texttt{distr} groups
constantly on the little core type?


\paragraph{Behaviour Categories.}
An approach I did not explore during this work are behaviour categories.
Consider four categories: very high \gls{mpc}, high \gls{mpc}, low \gls{mpc},
very low \gls{mpc}.
Each thread belongs to one category and the algorithm selects threads of
different categories to execute on the same core; the SMT abstraction can also
use these categories.
The assignment of threads to categories could also take a behaviour history
into account and categorize each thread regarding its past behaviour.

I also expect behaviour categories to help decide on difficult questions
arising from heterogeneous architectures.

\paragraph{Hierarchy of Load Balancers.}
I discuss hierarchical load balancing in chapter \ref{design:balancer} and
decided to implement a central component.
The logic in the central component does not allow to implement a
balancer on top of it, which follows different balancing policies.
Currently, all those decisions would be overruled.
Therefore, the central load balancer approach contradicts the general L4Re
idea, to implement different policies on top of each other.
A future project can explore designs allowing a hierarchy of balancers and
evaluate the performance differences compared to the central load balancing I
present in this work.

\begin{comment}
\paragraph{Workload-Aware Balancing.}
The modular approach allows quick replacement and testing of different
algorithms in the various components.
If very specific workloads run on the system, tuned balancing algorithms for
this specific workload can be used to increase the performance of the system.
\end{comment}

\paragraph{\large{}\scshape Conclusion.\newline}

{\itshape
This thesis demonstrated the benefit of topology- and behaviour-aware load balancing.
It explored static and dynamic information sources to enable informed balancing
decisions.
The designed load balancer connects hardware topology with balancing on
different levels, while maintaining communication relationships and isolation
properties.
The implementation proved the positive effect of informed decisions,
through reduced run-time deviation in all applications running concurrently.
Fair distribution of resources renders the system's throughput predictable for
known task sets.

While the evaluation lacks an analysis of all information sources, their
theoretical benefit was argued for.
The evaluation also uncovered implementation deficits which disturb the
measurements of heavily loaded systems.
I provided ideas to improve on these deficits.
However, the results present a clear tendency for behaviour-information-based
load balancing to decrease influences between threads and provide each thread
with a requirement-based resource share.

This thesis challenges future research to improve on cache isolation and usage
analysis, but also to explore load balancing on heterogeneous hardware, where
performance prediction is of even greater importance.
\/}


\cleardoublepage

%%% Local Variables:
%%% TeX-master: "diplom"
%%% End:
