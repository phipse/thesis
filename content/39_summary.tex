% vim:set ft=tex:
\section{Summary}
\label{design:summary}

During this chapter I discussed the difficulties with determining the
performance of corresponding hyper-thread cores without offline knowledge about
the running threads.
The \gls{ht} abstraction achieves two things: the load balancer compares the
performance of physically equal units of hardware, and it becomes independent
of the actual presence of \gls{ht}.

The load balancing service provides isolation for applications with real-time
or security requirements.
Also, I presented different approaches to gain knowledge about communication
relationships within a task or between a group of tasks.

The review of energy models led to the conclusion, that on x86
processors a race-to-idle scheme without usage of turbo-boost features is the
most promising model.
A reduction in computation time leads to larger idle periods for the different
cores, which in turn leads to more time spent in deep sleep states, where
energy usage is minimal.

Subsequently, I presented my design of the central load balancing service and its
modules.
Measurement and prediction of thread behaviour provides the knowledge to decide
on balancing options and to enforce this decision.

The chapter finished with a discussion about different definitions for load
and depending on the character of the load, which system states are considered
to be in balance.
\\

The following chapter presents changes to the Fiasco.OC kernel to provide
access to hardware performance counters; how a system designer informs the load
balancer about isolation and communication groups; and discusses different
algorithms to decide on thread placement.
