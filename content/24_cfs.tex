\section{Linux Completely Fair Scheduler}
\label{state:cfs}

\newacronym{cfs}{CFS}{Completely Fair Scheduler}

Since Linux 2.6.23 the \gls{cfs} runs Linux.
It is completely fair, in regard to execution time provided to each task.
To achieve fairness, it measures so called virtual runtime of each task and
selects the task with the lowest account to run next.
Internally, \gls{cfs} uses a red-black tree to sort the tasks regarding virtual
runtime, where the leftmost task in the tree is the one with the least virtual
runtime.

\gls{cfs} also supports group scheduling to provide fairness between tasks with one
thread and tasks with many threads.
If a task spawns many threads, they all execute on the account of the main
task, preventing one task to acquire more execution time than others.

Conceptually, \gls{cfs} runs the currently selected task until the virtual runtime is
higher than the virtual runtime of any other task.
Then a task switch occurs.
As this would lead to constant task switching, two measures are used to counter
act this: Target scheduling latency and minimum task runtime granularity.

The target scheduling latency (TSL) describes an amount of time shared between a
number of tasks.
If two tasks are runnable, and the TSL is 20ms, then each task runs for 10ms.
However, if more tasks become runnable, the amount of time per task shrinks and
eventually the switching overhead is higher than the time a task spends
executing.
To prevent this situation the minimum task runtime granularity defines a
minimum amount of time, a task needs to execute.
As the number of tasks grow, the TSL increases to be able to schedule each
runnable task once for the minimum task runtime.

Priorities are accommodated by weighting the virtual runtime accounted to each
task depending on the priority level.
The virtual runtime of a high priority task increases slower than the virtual
runtime of a low priority task, leading to more actual execution time for the
high priority task.

\paragraph{Domains, Groups, \& load balancing}
A domain is a unit of physical hardware.
A physical core is a domain and the two hyper-threads running inside are each a
group within this domain.
Therefore, a group is a set of domains in the same level of the hierarchy.
Each group within a domain is comparable to other groups in the same domain.

\gls{cfs} does active load balancing only groups in the same domain.
Their load is compared and if imbalances are found, \gls{cfs} tries to migrate a task
from one group to another to establish balance again.
Thereby, a cache affinity flag indicates, if the task ran recently and might
still have valid cache lines.
If the cache affinity flag is set, the task is not considered for migration.
