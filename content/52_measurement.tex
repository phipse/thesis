% vim:set ft=tex:
\section{Measurements}
\label{eval:measurements}

I measure each of the four SPEC benchmarks twice; once running alone in the
system and once, running under load in parallel with the three other benchmarks.
The solo execution time of each application serves as baseline to compare the
performance of different algorithms in a low load situation, and to enable the
computation of the slowdown when run in parallel with other applications.
The slowdown of the computation time is computed for the best case and worst
case, the lower and upper quartile, and the median.
The result provides a complete picture of the algorithms' performance.

I also compare the algorithms using box plots to visualize
the runtime for the major part of the measurements together with outliers.
However, due to the nature of the box plots, data points marked as outlier need
to be taken with a grain of salt. 
Box plots have six components: median, lower and upper quartile, lower and
upper whisker, and outliers.
Median and lower and upper quartile are actual values present in the data set,
and mark 50\%, 25\%, and 75\% of the sorted data.
The distance between the lower and upper quartile values is the \emph{inter
quartile distance} or IQD.
The lower and upper whiskers are computed as follows:
%
\begin{align*}
  \text{lower whisker} &= \text{lower quartile} - 1.5 * \text{IQD}\\
  %
  \text{upper whisker} &= \text{upper quartile} + 1.5 * \text{IQD}
\end{align*}
%
The actual value of the lower whisker is the value of the next data point,
which is higher than the computed value.
For the upper whisker its the next lower value data point.
Box plots consider every value beyond the value of the lower or upper whisker
as outlier.

Now the issue arises, if the inter quartile distance is low and therefore,
the whisker values are close to the respective quartile value, which leads to
lots of data points to be visualized as outliers.


% todo SMT measurement issue: need micro benchmark to determine increase in
% application performance, due to smt balancing decisions.

\begin{comment}
\begin{figure}[h!]
  \begin{subfigure}{.49\textwidth}
    \includegraphics[width=\textwidth]{images/finalPlots/gcc_simple_balancer.pdf}
    \caption{SPEC GCC}
    \label{baseline:gcc}
  \end{subfigure}
  \begin{subfigure}{.49\textwidth}
    \includegraphics[width=\textwidth]{images/finalPlots/gamess_simple_balancer.pdf}
    \caption{SPEC GAMESS}
    \label{baseline:gamess}
  \end{subfigure}
  \begin{subfigure}{.49\textwidth}
    \includegraphics[width=\textwidth]{images/finalPlots/lbm_simple_balancer.pdf}
    \caption{SPEC LBM}
    \label{baseline:lbm}
  \end{subfigure}
  \begin{subfigure}{.49\textwidth}
    \includegraphics[width=\textwidth]{images/finalPlots/mcf_simple_balancer.pdf}
    \caption{SPEC MCF}
    \label{baseline:mcf}
  \end{subfigure}
  \caption{Distribution of runtimes with simple balancer}
\end{figure}
\end{comment}
