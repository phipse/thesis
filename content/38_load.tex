% vim:set ft=tex:
\section{Load and Balance}
\label{design:load}

\newacronym{mpc}{MPC}{misses per cycle}

What is load?
And depending on the answer to this question: What is balance?

\gls{cfs} defines load as amount of execution time used by threads.
But a core's load can also be defined as
the cache and memory usage, the sum of \gls{instpc}, or simply as the number of
assigned threads.
Using a mix of the above as load measure is also imaginable.

The answer to the question: What is balance? -- depends on the definition of
load.
Assuming the \gls{cfs} definition, a system is balanced, if the groups within a
domain carry an even amount of load.
If this assumption holds for each level in the domain hierarchy, the system is
in balance.
But load only displays consumed computation time, it balances time and
disregards space.

The opposite approach measures only \gls{mpc} and ignores computation
resources.
So the load of a domain is the sum of \gls{mpc} of all assigned threads.
A system is in balance, if each group within a domain has an equal number of
\gls{mpc}.
Or better, if no thread-to-core assignment reduces the imbalances between cores.
But this measure suffers because of its disregard for remaining execution time.
While memory intense threads may perform better, computation intense threads
suffer due to migration and possibly high thread count imbalances between
cores.
For example, if one threads produces a high number of \gls{mpc} it is assigned a
core alone, to meet its cache needs.
If four other threads are computation intense and the sum of their \gls{mpc} is
still lower than the high \gls{mpc} threads, they will be sharing one core.
This system is in balance regarding \gls{mpc} but is perceived as heavily
imbalanced.

A third measure is \gls{instpc}.
It shows, how much instructions a thread was able to perform during its
runtime.
But as with \gls{mpc} this number is different for each thread.
CPU bound workloads tend to have a high number of \gls{instpc},
but memory bound workloads tend to have a much lower number, due to cache
misses.
Also, instructions are not equally costly, which leads
\citeauthor{snavely_symbiotic_2000} to their weighted instruction measure.
But at runtime it is not possible to pinpoint the weight of the current
instruction.

Balancing only by \gls{instpc} will lead to a similarly unfair distribution of
threads to cores like balancing by \gls{mpc}.
The disregard for cache usage will likely decrease the number of instructions
a cache sensitive threads is able to execute, as more cache misses are
generated.

The number of threads assigned to a core also impacts the progress of other
assigned threads.
But only, if the newly assigned thread is really executing and not a service,
which mostly idles and executes only on request.
Thread count ignores load, \gls{instpc}, and \gls{mpc} and is therefore
useless.

A mix of execution time, \gls{instpc}, and \gls{mpc} looks promising.
The main drawback of execution time is the ignorance towards cache usage.
Extending the execution time measure with \gls{mpc} knowledge should improve this.
A second approach can combine \gls{instpc} and \gls{mpc} measures: I assume
\gls{instpc} and \gls{mpc} to be two orthogonal measures, hence running a
high \gls{mpc} on the same core as a high \gls{instpc} thread should perform
well.
\\

% todo merit flaw table as summary

After discussing load and balance, it is also necessary to decide on a
threshold, before actively migrating threads.
The cost of a migration must not necessarily be paid to have a perfectly
balanced system.
Slight imbalances decrease the performance compared to a perfectly balanced
system, but as long as this decrease is less than the cost of a migration,
it is practicable to tolerate the imbalance.

Cache-heavy threads, determined by last level cache misses per cycle, can have
two reasons for a high miss count: first, they reuse little memory and use new
memory in a not prefetchable manner, otherwise the prefetcher would not miss;
second, the cache working set is large and parts of the working set were
evicted during the runtime of other programs.
Migrating the former kind does not hurt the program much, migrating the latter
kind however is costly and should be avoided.

While it is generally a bad idea to migrate high \gls{mpc} threads, a
clustering of high \gls{mpc} threads is worse.

So the balancing should take place in time and in space.
While balancing in space takes priority over balancing in time and time 
balancing evens out imbalances.
The space balancing should act, if high imbalances regarding accumulated
\gls{mpc} per core arise, or if the difference of \gls{mpc} heavy threads per
core rises above one or two high \gls{mpc} threads per core.

Regarding time imbalances, the questions are: is there a thread that would even
out imbalances if migrated? And: When does the load balancer start searching
for such a thread?
In the evaluation section I search for a good threshold value.
