\chapter(Motivation)


\section(Security Aspects)

* shared resources --> covert channels --> information leakage


\subsection(Covert / Side Channel Attacks -- Cache)

* Issue: Information leakage through cache accesses - hit/miss behavior
* Attack vectors:
  * Data-Cache / Instruction-Cache
  * Shared functional units
  * Branch predictors
* Attack types:
  * Time driven 
    * Cache Collision attacks
    * D.J. Bernstein attack (2005)
  * Access driven


\subsubsection(OpenSSL cache collision attacks)

* CLFLUSH instruction flushes the whole cache hierachy; local and remote
  caches. MOESI propagates cache flush.
  ==> FLUSH+RELOAD attacks still possible, even if attack process runs on
  different CORE with no shared HW-CACHE.
  (s. Intel ArchSoftDev Manual Vol. 2A 3-111)
==> lock whole system to execute encryption/signing operations (similar to
    Flicker (see  Trusted Computing Lesson))


\subsection(Mitigation)

* Grouping, security levels


\section(Performance inprovements)

* Stable throughput
  * memory intensive \& menory non-intensive workload on cores with shared
    cache; HyperThreads
* Performance per watt
  * TurboBoost vs. sleep-state
* Gang scheduling


\section(Miscellaneous)

* no changes to HW or applications
* user-space program; minimal kernel changes
* Real-Time?
 
