% vim:set ft=tex:
\section{Kernel Performance Counter}
\label{impl:perfcounter}

\gls{intel} provides hardware performance counters for various CPU
architectures.
Their intention is to enable software developers to observe the behaviour of
their software and examine bottlenecks or performance hotspots.

I presented the performance monitoring capabilities of the targeted Haswell
architecture in section \ref{state:haswell}.

The present kernel implementation for architectural performance monitoring is
extended to use the fixed-function counters for unhalted core cycles and
instructions retired.
Additionally, a general purpose counter for \gls{llc} misses is used,
leaving three general purpose counters per logical core for future use.

\paragraph{Configuration \& Start}
Each general purpose performance counter consists of a counter register and a
configuration register.
To specify the event the counter should observe, the \texttt{UMask} and
\texttt{Event Select} fields in the configuration register must be filled.
Additionally, user and operating system mode bits must be set, to count events
happening in privileged and unprivileged mode.
With the edge detect bit set, every new occurrence of the selected event is
counted.
To start counting, two enable bits must be set.
The first in the counter's configuration register and the second in the global
control register.
If one of these two bits is not set, the counter is not enabled.
\\

Fixed-function counters are a bit different.
They consist of one counter register and one control register shared among all
fixed-function counters, to enable and disable each one.
Like general purpose counters, the fixed-function counters must also be enabled
in the global control register.

Fiasco.OC already contains the boot parameter \texttt{-loadcnt} enabling the
unhalted core cycle counter on each core.
Although a fixed-function counter is available for this event, the
implementation still used a general purpose counter.
Using this flag, each core's performance counters are configured and enabled to
count unhalted core cycles, instructions retired, and \gls{llc} misses.

\paragraph{Per Thread Accounting}
% todo find out, where and when thread perf counting is enabled?
The \texttt{-loadcnt} flag only enables the hardware counters.
The accounting must be enabled for each thread separately, by calling
\texttt{l4\_thread\_perf\_reset}, which also resets the counter value to zero.
Only then, Fiasco.OC adds the counter values on each context switch to the
thread's account and resets the hardware counter to zero for the next thread.

I extended the \texttt{L4::Thread} interface to access the thread's performance
account.
\texttt{perf\_read} provides the interface to retrieve a thread's current
account.
A \texttt{perf\_counter\_read\_t} struct must be passed by reference, which
is filled with the values provided by the kernel.

Listing \ref{impl:lst:perfcntrinterface} shows the user-land implementation of
this interface.

\begin{lstlisting}[language=c++,
  caption={User land interface for retrieving and resetting a thread's
    performance counters; UTCB-pointer left out for brevity.},
  label={impl:lst:perfcntrinterface}]
enum L4_thread_ops {
  ...
  L4_THREAD_PERF_READ_OP	      = 10UL,
  /**< Read thread's performance counter */
  L4_THREAD_PERF_RESET_OP	      =	11UL,
  /**< Reset thread's performance counter */
  ...
}

struct perf_counter_read_t {
  l4_uint64_t core_cycle_cnt;
  l4_uint64_t instr_retired_cnt;
  l4_uint64_t llc_miss_cnt;
};
typedef struct perf_counter_read_t perf_counter_read_t;

L4_INLINE l4_msgtag_t
l4_thread_perf_read(l4_cap_idx_t thread,
                    perf_counter_read_t *cntr) L4_NOTHROW;

L4_INLINE l4_msgtag_t
l4_thread_perf_reset(l4_cap_idx_t thread) L4_NOTHROW;
\end{lstlisting}
