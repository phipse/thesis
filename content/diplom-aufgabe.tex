\selectlanguage{ngerman}
\begin{centering}
\large
\textit{Aufgabenstellung für die Diplomarbeit}
\\
\end{centering}

\vspace{1cm}

\begin{flushright}
\normalsize\noindent
\textbf{Design \& Implementierung eines Lastverteilungssystems\\ für
Mehrprozessor-Mikrokernsysteme}\\
\end{flushright}
\vspace{3mm}

Aktuell unterstützt L4Re keine dynamische Lastverteilung auf mehrere
Prozessorkerne. Ein Thread muss dem Scheduler selbst mitteilen auf welchem Kern
er ausgeführt werden will, ansonsten wird er auf Kern \#0 ausgeführt.

Um Mehrprozessorsysteme besser auszulasten kann eine Lastverteilungskomponente
eingesetzt werden, die Threads selbstständig auf mehrere Prozessoren verteilt.
Hierbei können verschiedene Vorgehensweisen zum Eisatz kommen.
Das L4Re-System hat Ausführungsanforderungen bezüglich Sicherheit sowie
Echtzeit, die durch Isolation von Subsystemen auf unterschiedlichen Prozessoren
umgesetzt werden müssen. Einzelne Applikationen haben aber auch die Anforderung
mehrere Threads prozessorlokal auszuführen.
Der Energieverbrauch eines Gesamtsystems ist ein weiterer Gesichtspunkt und
kann durch eine Lastverteilungskomponente beeinflusst werden, indem nicht
benötigte Prozessoren potentiell abgeschaltet werden. Die
Lastverteilungskomponente muss hierbei für entsprechend leere Prozessoren
sorgen.

Ziel der Arbeit ist es, die Informationen zu identifizieren und Mechanismen
bereitzustellen, anhand derer eine Lastverteilungskomponente mit  ihren
Algorithmen Entscheidungen treffen kann. Dabei muss im Zusammenspiel zwischen
Anwendungen und Betriebssystem die Last des Systems analysiert, eventuell
zukünftige Last vorausgesagt und anhand von Einstellungen
Lastbalancierungsentscheidungen getroffen werden.
Die Lastverteilungskomponente soll in das bestehende L4Re-System integriert
werden.
Für die Lastvorhersage sollen ausgewählte Heuristiken benutzt, aber auch
Sicherheitskriterien beachtet werden.
Die Heuristiken sollen anhand verwandter Arbeiten ermittelt werden.

Die Lösung soll exemplarisch umgesetzt und mit einem multi-threaded Benchmark
evaluiert werden.
Dabei sollen mindestens zwei Lastbalancierungsalgorithmen zum Einsatz kommen: ein
Energieoptimierender und ein Latenzoptimierender sowie soll die Ausführung von
sicherheits- und echtzeitkritischen Applikationen beachtet werden.

Die zentrale Fragestellung ist welche Informationen lassen sich effizient durch
Nutzermechanismen sammeln und welche sind nötig, jedoch nur im
Betriebssystemkern vorhanden. Hierbei sollen entsprechende Schnittstellen
vorgeschlagen werden, die außerdem Sicherheits- und Effizienzkriterien
unterliegen.
