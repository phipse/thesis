% vi:ft=tex
\documentclass[utf8,10pt,aspectratio=169]{beamer}

\usepackage[english]{babel}
\usepackage{etex}
\usepackage{pgf}
\usepackage[absolute,overlay]{textpos}
\usepackage{tikz}
\usepackage{listings}
\usepackage{graphicx}
\usepackage{pgfplots}
\usepackage{standalone}
\usepackage{enumitem}
%\usepackage[bitstream-charter]{mathdesign}
\usepackage{berasans}
\usepackage[backend=biber,style=alphabetic]{biblatex}
\usepackage{multirow}
\usepackage{mathtools}
\usepackage{ragged2e}
\usepackage{minted}
\usepackage{verbatim}
\usepackage{textcomp}

\usetikzlibrary{arrows, shapes, positioning, trees,calc,intersections,patterns}
\mode<presentation>

% 45 degree rotated column descriptor
\usepackage{adjustbox}
\usepackage{array}

\usetheme[noheader,smallrightmargin,smallleftmargin,nosectionnum,heavyfont,pagenum]{tud}
\addbibresource{../Related_Work.bib}


\setbeamertemplate{footnote}
{
  %\scriptsize
  \tiny
  \noindent%
  \insertfootnotemark~\insertfootnotetext
}

\setbeamertemplate{enumerate items}[default]
\setbeamertemplate{itemize items}[triangle]

\title[]{Behaviour-Aware Load Balancing\newline for a Microkernel}

\author{Philipp Eppelt}

\date{04.03.2015}
\datecity{Dresden}

\einrichtung{Fakult\"a{}t Informatik}

\institut{Institut f\"u{}r Systemarchitektur}
\professur{Lehrstuhl f\"u{}r Betriebssysteme}


\begin{document}

\maketitle

\large

\newcommand{\ft}[1]{\frametitle{\hfill #1}}

% motivation
%%%%%%%%%%%%%%%%%%%%%%%%%%%%%%%%%%%%%%%%%%%%%%%%%%%%%%%%%%%%%%%%%%%%%%%%%%%%%%

\begin{frame}
  \frametitle{behaviour awareness \& load balancing}
  \centering
  \begin{itemize}
  \setlength{\itemsep}{6pt}
    \item What describes thread behaviour?
    \begin{itemize}
      \item isolation \& communication properties
      \item resource usage
    \end{itemize}
    \item
    \item Why do we want automatic load balancing?
    \item How does Fiasco.OC/L4Re balance load?
    %\item Fiasco.OC does no load balancing, but allows user-land policies
  \end{itemize}

\end{frame}

\begin{frame}
  \frametitle{data latency limits execution speed\\ and leaves processor
  underutilized}
  \centering
  \includegraphics[height=5cm,keepaspectratio]{../images/cpu_mem_gap}
  \footnotetext[1]{Hennessy, Patterson, Asanovi\'{c}: \textit{Computer
  Architecture: A Quatitative Approach}, Elsevier, Amsterdam, 2012. p. 73}
\end{frame}


\begin{frame}
  \frametitle{caches reduce number of waits for memory\\ but increase cache
  miss latency}
  \centering
  \includegraphics[height=5cm,keepaspectratio]{../images/mem_access_hierarchy_latency}
  \footnotetext[1]{Hennessy, Patterson, Asanovi\'{c}: \textit{Computer
  Architecture: A Quatitative Approach}, Elsevier, Amsterdam, 2012. p. 72}
\end{frame}


% state of the art
%%%%%%%%%%%%%%%%%%%%%%%%%%%%%%%%%%%%%%%%%%%%%%%%%%%%%%%%%%%%%%%%%%%%%%%%%%%%%%
% service architecture
%%%%%%%%%%%%%%%%%%%%%%%%%%%%%%%%%%%%%%%%%%%%%%%%%%%%%%%%%%%%%%%%%%%%%%%%%%%%%%

\begin{frame}
  \centering
  \Large
  Load Balancing Service
\end{frame}


\begin{frame}
  \frametitle{load \& balance}
  \centering
  \begin{itemize}
  \setlength{\itemsep}{6pt}
    \item consumed CPU time per core or per thread
    \item last-level cache misses per cycle
    \item instuctions per cycle
    %\item thread count per core
    \item
    \item balance levels: SMT, cores, sockets, \ldots
  \end{itemize}
\end{frame}


\begin{frame}
  \frametitle{isolation-, communication-, and cache-aware placement depends on
  topology knowledge}
  \centering
%  \begin{minipage}[l]{.49\columnwidth}
%    \includestandalone[mode=image,width=\columnwidth]{../images/inclusive_caches}
%  \end{minipage}
%  \begin{minipage}[r]{.49\columnwidth}
    \includestandalone[mode=image,height=5cm,keepaspectratio]{../images/haswell_core_layout}
%  \end{minipage}
\end{frame}


\begin{frame}[fragile]
  \frametitle{system architect provides static properties\\ via task configuration}
  \centering
  \begin{minipage}[c]{\columnwidth}
    \begin{minted}[]{lua}
ld:start({
    caps = { },
    scheduler = loadBalancerFactory:create(
      L4.Proto.Scheduler,
      "min_prio = 0", "max_prio = 5", "distr = fract" ),
    log = { "client", "green" }
  },
  "rom/ex_fractal");
    \end{minted}
  \end{minipage}
\end{frame}


\begin{frame}
  \frametitle{task proxy is scheduler for all threads}
  \centering
  \includestandalone[mode=image,height=5cm,keepaspectratio]{../images/threadMapper_layout}
\end{frame}


\begin{frame}
  \frametitle{incomplete system knowledge}
  \centering
  \includestandalone[mode=image,height=5cm,keepaspectratio]{../images/system_layout_with_loadBalancer}
\end{frame}


\begin{frame}
  \frametitle{MPDE-cycle:\\ accommodate workload changes}
  \centering
  \includestandalone[mode=image,height=6cm,keepaspectratio]{../images/statusvortrag/arch_interval_cycle}
\end{frame}

% evaluation
%%%%%%%%%%%%%%%%%%%%%%%%%%%%%%%%%%%%%%%%%%%%%%%%%%%%%%%%%%%%%%%%%%%%%%%%%%%%%
\begin{frame}
  \centering
  \Large
  Evaluation
\end{frame}

\begin{frame}
  \frametitle{measurement setup}
  \begin{itemize}
  \setlength{\itemsep}{6pt}
    \item time measure: task creation until task destruction
    \item SPEC CPU2006: 403.gcc, 416.gamess, 429.mcf, 470.lbm
    \item interval duration 100ms
    \item solo, medium load, heavy load
    \item at least 100 repetitions
    \item load scenarios run-times measured during the same run
  \end{itemize}
\end{frame}


\begin{frame}
  \centering
  \includegraphics[height=6cm,keepaspectratio]{../images/cache_misses_spec.png}
  \footnotetext[1]{Zhuravlev, Blagodurov, Fedorova: \textit{Addressing Shared
  Resource Contention in Multicore Processors via Scheduling} in: Proceedings of
  the Fifteenth Edition of {ASPLOS} on Architectural Support for Programming
  Languages and Operating Systems, ACM, New York, 2010.}
\end{frame}


\begin{frame}
  \centering
    \includegraphics[height=4.3cm,keepaspectratio]{../images/finalPlots/boxplots/pdf/boxplot_gcc_mcf_solo}

    \includegraphics[height=4.3cm,keepaspectratio]{../images/finalPlots/boxplots/pdf/boxplot_gamess_lbm_solo}
\end{frame}

\begin{frame}
  \centering
    \includegraphics[width=\columnwidth,keepaspectratio]{../images/finalPlots/boxplots/pdf/boxplot_gcc_mcf_solo}

    \small
    app running alone in the system
\end{frame}

\begin{frame}
  \centering
    \includegraphics[width=\columnwidth,keepaspectratio]{../images/finalPlots/boxplots/pdf/boxplot_gamess_lbm_solo}

    \small
    app running alone in the system
\end{frame}


\begin{frame}
  \centering
    \includegraphics[height=4.3cm,keepaspectratio]{../images/finalPlots/boxplots/pdf/boxplot_gcc_mcf_all}

    \includegraphics[height=4.3cm,keepaspectratio]{../images/finalPlots/boxplots/pdf/boxplot_gamess_lbm_all}
\end{frame}


\begin{frame}
  \centering
    \includegraphics[height=4.3cm,keepaspectratio]{../images/finalPlots/boxplots/pdf/boxplot_gcc_mcf_heavy}

    \includegraphics[height=4.3cm,keepaspectratio]{../images/finalPlots/boxplots/pdf/boxplot_gamess_lbm_heavy}
\end{frame}

\begin{frame}
  \frametitle{Overhead of concurrent execution\\ over solo execution time}
  \centering
    \includegraphics[height=5cm,keepaspectratio]{../images/runtimeIncrease/barchart_load_inc.pdf}
    \includegraphics[height=5cm,keepaspectratio]{../images/runtimeIncrease/barchart_heavy_inc.pdf}
    %table_load.pdf
\end{frame}


\begin{frame}
  \frametitle{Conclusion}
  \centering
  \begin{itemize}
  \setlength{\itemsep}{6pt}
    \item dynamic load balancing on Fiasco.OC/L4Re
    \item worth the additional cost
    \item supports isolation and communication properties
    \item fair resource distribution and predictable run-times
  \end{itemize}
\end{frame}

% future work
%%%%%%%%%%%%%%%%%%%%%%%%%%%%%%%%%%%%%%%%%%%%%%%%%%%%%%%%%%%%%%%%%%%%%%%%%%%%%
\begin{frame}
  \centering
  \Large
  Future Work
\end{frame}

\begin{frame}
  \frametitle{Future Work}
  \centering
  \begin{itemize}
  \setlength{\itemsep}{6pt}
    \item delayed measurement fix
    \item Intel Cache Allocation Technology
    \item Intel Cache Monitoring Technology
    \item hierarchical load balancing
    \item multi-socket and distributed systems
    \item architectures besides x86
    \item heterogeneous systems
  \end{itemize}
\end{frame}


\begin{frame}
  \centering
  \Large
  Questions?
\end{frame}


% Backup Slides
%%%%%%%%%%%%%%%%%%%%%%%%%%%%%%%%%%%%%%%%%%%%%%%%%%%%%%%%%%%%%%%%%%%%%%%%%%%%%


\begin{frame}
  \centering
  \includegraphics[width=\textwidth]{../images/runtimeIncrease/table_load.pdf}
\end{frame}


\end{document}
